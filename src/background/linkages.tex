\subsection{Linkages}
\begin{definition}[Linkage]\label{def:linkages-1}
A collection of fixed-length 1D segments joined at their endpoints to form a graph.
\end{definition} 
\begin{definition}[Graph]\label{def:linkages-2}
An ordered pair $G = (V, E)$ comprising a set $V$ of vertices or nodes together with a set $E$ of edges or lines
\end{definition} 
\begin{definition}[Cycle]\label{def:linkages-3}
 A closed walk with no repetitions of vertices or edges allowed, other than the repetition of the starting and ending vertex
\end{definition} 
\begin{definition}[Configuration]\label{def:linkages-6}
A specification of the location of all the link endpoints, link orientations and
joint angles.\cite{demaine2008geometric}
\end{definition}
\begin{definition}[Configuration Space]\label{def:linkages-7}
The space of all configurations of a linkage.
\end{definition} 
A configurations space is said to be continuous if for any two configurations, $\mathcal{A}$ and $\mathcal{B}$ of a linkage $L$, $\mathcal{A}$ can be continuously reconfigured to $\mathcal{B}$ such that, the reconfigurations reside in the configuration domain, $L$ remains rigid throughout reconfiguration (i.e. all links' lengths are preserved), and no violations of linkage intersection conditions. 
\begin{definition}[Pinned Joint]\label{def:linkages-8}
A vertex of a graph (or linkage) that is fixed to a position in a plane.
\end{definition} 
\begin{definition}[Free Joint]\label{def:linkages-8}
A vertex of a graph (or linkage) that is not fixed to a position in a plane.
\end{definition} 
