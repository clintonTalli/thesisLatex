

\subsection{Circle Packing}
\begin{definition}[Circle Packing]\label{def:circlePacking}
$P$ of a planar graph $G$ is a set of of circles with disjoint
interiors $\left\lbrace C_v \right\rbrace_{v \in G} $ such that two
circles are tangent if and only if the corresponding vertices form an edge.
\cite{arXiv13113363v1}
\end{definition} 


\begin{thm}[Circle Packing Theorem]\label{thm2-1}
For every connected simple planar graph $G$ there is a circle packing in the
plane whose intersection graph is (isomorphic to) $G$.
\end{thm}
\begin{figure}[h]
\begin{center}
\includegraphics[scale=.5]{graphics/circlePackingTheoremExample.pdf}
\end{center} 
\caption{This figure is an example of a circle packing for the given simple planar graph.}
\end{figure} 
\textbf{Are all linkages simple planar graphs?}  A proof of Theorem \ref{thm2-1} is found in \cite{stephenson2005introduction}.
\subsubsection{Circle Packings and Polygonal Linkages}
Given a circle of radius $r$, we establish the isomorphism to a hexagon by
circumscribing the vertices of the regular hexagon.
\begin{figure}[h]
\begin{center}
\includegraphics{graphics/circumscribedHexagon.pdf}
\caption{A circumbscribed hexagon}
\end{center}
\end{figure}
\subsubsection{Hinged Polygons}
\begin{definition}[Polygonal Chain]\label{def}
A polygonal chain $P = \left( v_0, v_1, \dots, v_{n-1}\right) $ is a sequence of
consecutively joined segments (or edges) $e_i = v_i v_{i+1}$ of fixed lengths
$l_i = \left\vert e_i\right\vert $, in a plane. \cite{Biedl99lockedand}
\end{definition}
A chain is said to be closed if $v_{n-1} = v_1$, otherwise it is said to be
open. Hinged polygons have been researched for decades and related to linkage problems
\cite{Biedl99lockedand,canny1988complexity}.

Consider the locked configuration of figure \ref{figure:7hexLocked}.  We can
 configure the hexagons to be locked by placing hinged points as follows:
\begin{figure}[h]
\begin{center}
\includegraphics[scale=.33]{graphics/7hexLocked.pdf}
\caption{A locked 7 hexagonal configuration.  (needs to modify picture by
placing red points for hing points.)}
\label{figure:7hexLocked}
\end{center} 
\end{figure}
To prove that it is a locked configuration:
\begin{itemize}
 \item[\rn{1}]
 \item[\rn{2}]
 \item[\rn{3}]
 \item[\rn{4}]
 \item[\rn{5}]
 \item[\rn{6}]
 \item[\rn{7}]
 \item[\rn{8}]
 \item[\rn{9}]
 \item[\rn{10}]
 \end{itemize}
\subsubsection{Hinged Hexagons}
\begin{thm}[]\label{thm}
Any finite collection of polygons of equal area has a common hinged dissection.
\cite{abbott2012hinged}
\end{thm}
\paragraph{The Shapes}
Figure \ref{fig:lockingShape} is a locking shape:
\begin{figure}[h]
\begin{center}
\includegraphics{graphics/lockingShape.pdf}
\caption{This is the shape that resides in boundary of the lattice.}
\label{fig:lockingShape}
\end{center}
\end{figure}
Figure \ref{fig:lockingShape} shall reside in the boundary of a lattice and have
a hinge point at one vertex where the locking shape and boundary meet.
\begin{figure}[h]
\begin{center}
\includegraphics{graphics/shapeInChannel.pdf}
\end{center} 
\caption{A locking shape in the lattice boundary's channel.}
\label{fig:lockingShapeInChannel}
\end{figure}
\paragraph{Junctions}
We define junctions to be the point three hexagons meet in a hexagonal lattice,
e.g. Figure \ref{fig:lattice}.
%Radius of regular polygons 
\newdimen\R
\R=4.5cm
\begin{figure}[h] 
\begin{center}
\begin{tikzpicture}
\begin{scope}
\filldraw[pattern=hexagons]  (0:\R) \foreach \x in {60,120,...,359} {
                -- (\x:\R)
            }-- cycle (90:\R);
\end{scope}
\end{tikzpicture}
\caption{A portion of a hexagonal lattice.}
\label{fig:lattice}
\end{center}
\end{figure}
\newpage
\paragraph{Central Scaling}
\paragraph{Junctions in Conjunctive Normal Form}
Explain the configurations we're interested in.
