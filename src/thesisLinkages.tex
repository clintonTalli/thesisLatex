\section{Linkages and Polygonal Linkages}
Given a \it{graph}, an ordered pair $G = (V,E)$, comprising of a set $V$ of vertices or nodes together with a set $E$ of edges or lines, then a \it{linkage} of $G$ is the realization (or embedding) of $G$ in $\bbr^2$. A \it{polygonal linkage} is an ordered pair, $L = (H,P)$,  comprises of a set of polygons, $P$, and a set of hinge points $H$ where each hinge $h \in H$ corresponds to two points on the boundary of two distinct polygons in $P$. Without loss of generality, for this paper, we focus on linkages and polygonal linkages that are simple planar  graphs, i.e.:
\begin{itemize}
\item[\rn{1}] does not have edges (polygons) that cross (intersect),
\item[\rn{2}] have loops (i.e. $(v,v) \in E$), or
\item[\rn{3}] does not have multiple edges between any pair of vertices.
\end{itemize}
We may visit special cases in which we look at planar graphs that satisfy the last two conditions but not the first.  

\subsection{Configuration Spaces of Linkages}
To describe the types of motion that we are interested in linkages we must define the graph isomorphism.  Two graphs $G=(V_1,E_1)$ and $\Gamma = (V_2,E_2) $, a bijection $f: V_1 \mapsto V_2$ such that for any two vertices $u,v \in V_1$ that are adjacent, i.e. $(u, v) \in E_1$, if and only if $(f(u),f(v)) \in E_2$. 
\begin{table}[!ht]
\begin{center}
$$\begin{array}{|c|c|c|}\hline
\text{Graph}&\text{Vertices}&\text{Edges}\\\hline
G&\left\lbrace a,b,c,d,e \right\rbrace & \left\lbrace (a,b),(b,c),(c,d),(d,e),(e,a) \right\rbrace \\\hline
\Gamma&\left\lbrace 1,2,3,4,5 \right\rbrace & \left\lbrace (1,2),(2,3),(3,4),(4,5),(5,1) \right\rbrace \\\hline
\end{array} $$
\caption{Two graphs that are isomorphic with the alphabetical isomorphism $f(a)=1$, $f(b)=2$, $f(c) = 3$, $f(d)=4$, $f(e)=5$.}
\end{center} 
\label{table:linkage-1}
\end{table} 

Next we add restrictions to our graph isomorphisms to narrow our focus:
\begin{itemize}
\item[\rn{1}] We focus on isomorphisms for planar graphs and or polygonal linkages, simple planar graphs, and
\item[\rn{2}] the isomorphism preserves edge lengths (polygonal area), e.g. $d(u,v) = d(f(u),f(v))$.
\end{itemize}  
With these restrictions of our isomorphisms, we can begin to describe a range of motion to transform a linkage.  That range of motion is said to be the configuration space of that linkage.  To expand on this concept, for given linkage, $L=(V,E)$, and for a given vertex $v \in V$, the set of points in which $v$ can be realized in the plane would be the configuration space for that vertex, $C_v$.  Defining some order of the vertices in $L$, i.e. $V = \left\lbrace v_n \right\rbrace_{i=1}^n$, then the \it{configuration space} for $L$ is said to be the cartesion product of the configuration space of vertices:
\begin{equation}\label{eqn:linkages-1}
C(L) = C_{v_1} \cross C_{v_2} \cross \cdots \cross C_{v_n}
\end{equation} 
Some food for thought on configuration spaces and motions on linkages:
\begin{itemize}
\item[\rn{1}] A configuration space is said to be \it{connected} if there is a continuous mapping for any two planar realizations (linkages) of a graph in the plane.  Otherwise it is said to be \it{disconnected}.
\item[\rn{2}] If the configuration space of a vertex, $C_v$, is a singleton set, then the vertex is said to be \it{pinned}. Otherwise it is said to be \it{free}.
\item[\rn{3}] The types of motions (mappings) that we refrain from using on linkages are translations.
\end{itemize}\newpage 
Note that configuration spaces for polygonal linkages are described similarly.
\subsubsection{Realizability of Linkages}
Suppose we had two configurations of a linkage, $\mathcal{A}$ and $\mathcal{B}$.  A question that can be posed is can we reconfigure $\mathcal{A}$ to $\mathcal{B}$ continuously while respecting simple planar graph conditions?  The answer to this question is a yes or no.  If yes, then there must exist a path connected configuration space between $\mathcal{A}$ and $\mathcal{B}$.  It has been shown that this problem can be posed as a planar satisfiability problem \cite{Breu19983,mulzer2008minimum} (Later on in this paper we'll cover satisfiability problems).  This is the type of problem that we face in this paper.  We will continue to explore this in a different manner, with circle packings.