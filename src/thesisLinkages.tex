\section{Linkages}
A \textit{linkage} is an ordered pair $G = (V,E)$ comprising of a set $V$ of vertices or nodes 
together with a set $E$ of edges or lines. This definition is commonly used for graphs.  Mapping 
the linkage $G$ into the plane is said to be the \textit{embedding}, i.e. $L : V \mapsto 
\bbR^{2}$.  A length function correspond to a linkage, $l: E \mapsto \bbr^+$ gives a length to an 
edge in the linkage.  If We consider a \textit{realization} of a linkage is range of $L$, i.e. 
$L(V)$. If for every edge $(u,v) \in E$ such that $l\left( \left(u,v\right) \right) = \left\vert 
L(u) - L(v) \right\vert$ is true, then $L$ is said to be a \textit{proper embedding} of $G$.
Without loss of generality, for this paper, we focus on linkages that are simple planar  graphs, 
i.e.:
\begin{itemize}
\item[\rn{1}] does not have edges that cross,
\item[\rn{2}] have loops, i.e. $(v,v) \in E$, or
\item[\rn{3}] does not have multiple edges between any pair of vertices.
\end{itemize}
We may visit special cases in which we look at planar graphs that satisfy the last two conditions 
but not the first, e.g.:
\begin{figure}[h]
\begin{center}
\includegraphics[scale=1]{graphics/crossingEdgeLinkage.pdf}
\end{center} 
\caption{A linkage where edges cross however it does not contain loops or multiple edges between 
vertices.}
\label{fig:linkage-3}
\end{figure}
\section{Polygonal Linkages}
\begin{figure}[h]
\begin{center}
\includegraphics[scale=1]{graphics/hingeOnThreeDistinctPolygons.pdf}
\end{center} 
\caption{(a) A polygonal linkage with a non-convex polygon and two hinge points corresponding to 
three polygons.  Note that hinge points correspond to two distinct polygons.(b) Illustrating that 
two hinge points can correspond to the same boundary point of a polygon.}
\label{fig:linkage-1}
\end{figure}
%describe how it is a generalization of Linkages.
Polygon linkages are similar to linkages.  They are an ordered pair of sets, with the exception 
that the set of vertices become a set of hinges, $\HH$, and the set of edges become a set of 
polygons, $\PP$.  Formally, a \it{polygonal linkage} is an ordered pair, $G = (\HH,\PP)$,  
comprises of a set of polygons, $\PP$, and a set of hinge points $\HH$ where each hinge $h \in \HH$ 
corresponds to two points on the boundary of two distinct polygons in $\PP$. Mapping the linkage 
$G$ into the plane is said to be the \textit{embedding}, i.e. $L : \HH \mapsto \bbR^{2}$.  We 
consider a \textit{realization} of a polygonal linkage is range of $L$, i.e. $l(\HH)$.

In figure (\ref{fig:linkage-1}), we illustrate that two hinge points can reside on the same point 
in the plane. Figure (\ref{fig:linkage-1}) and figure (\ref{fig:linkage-3}) are examples of special 
cases that we may run into, but do not want to focus heavily on.  They are presented to the reader 
to facilitate understanding of the definitions of polygonal linkages and linkages respectively.  
Without loss of generality, for this paper, we focus on polygonal linkages that are 
equivalent to simple planar graphs.
%Foigr each hinge point $h \in \HH$, there 
%exists some subset $P \subset \PP$ with 
%at least cardinality of 2 such that for each polygon $p \in P$, $h$ corresponds to some point on 
%the boundary of $p$.  When the polygonal linkage is embedded in the plane, the hinge point is 
%where 
%the polygons of $P$ \it{kiss}.  
% show an example of polygonal linkages
\begin{figure}[h]
\begin{center}
\includegraphics[scale=1]{graphics/PolygonalLinkageExamples.pdf}
\end{center} 
\caption{(a) A polygonal linkage with a non-convex polygon and a hinge point corresponding to three 
polygons.  (b) A polygonal linkage with 8 regular polygons.}
\label{fig:linkage-2}
\end{figure}
For the remainder of this thesis, we'll focus on the polygonal linkages with the following 
restrictions:
\begin{enumerate}
 \item  For each hinge point $h \in \HH$, there exists some subset $P \subset \PP$ with 
exactly cardinality of 2, and
\item every polygon in $\PP$ is convex.
\end{enumerate}
Formally, we define a \it{polygonal linkage} as an ordered pair $L = (\HH,\PP)$ comprising of a 
set of hinges, $\HH$, where each hinge $h\in \HH$ corresponds to two points on the boundary of two 
distinct polygons. A \emph{realization} of a polygonal linkage is an interior-disjoint placement of 
congruent copies of the polygons in $\PP$ such that the points corresponding to each hinge are 
identified (Fig. \ref{fig:1}). 

%With these restrictions of our isomorphisms, we can begin to describe a range of motion to 
%transform a linkage.  That range of motion is said to be the configuration space of that linkage.  
%To expand on this concept, for given linkage, $L=(V,E)$, and for a given vertex $v \in V$, the set 
%of points in which $v$ can be realized in the plane would be the configuration space for that 
%vertex, $C_v$.  Defining some order of the vertices in $L$, i.e. $V = \left\lbrace v_n 
%\right\rbrace_{i=1}^n$, then the \it{configuration space} for $L$ is said to be the cartesion 
%product of the configuration space of vertices:

%DESCRIBE THE FOLLOWING:
%1)CONFIGURATION SPACE AS A VECTOR SPACE OF DIMENSION 2^N WHERE EDGE LENGTH IS PRESERVED.
%2)PINNING 1 VERTEX TO ORIGIN AND A NEIGHBOR, ADD MOTIVATION TO PREVENT ROTATION AND TRANSLATIONS.

% \begin{equation}\label{eqn:linkages-1}
% C(L) = C_{v_1} \cross C_{v_2} \cross \cdots \cross C_{v_n}
% \end{equation} 
% Some food for thought on configuration spaces and motions on linkages:
% \begin{itemize}
% \item[\rn{1}] A configuration space is said to be \it{connected} if there is a continuous mapping for any two planar realizations (linkages) of a graph in the plane.  Otherwise it is said to be \it{disconnected}.
% \item[\rn{2}] If the configuration space of a vertex, $C_v$, is a singleton set, then the vertex is said to be \it{pinned}. Otherwise it is said to be \it{free}.
% \item[\rn{3}] The types of motions (mappings) that we refrain from using on linkages are translations.
% \end{itemize}
% Note that configuration spaces for polygonal linkages are described similarly.
% \subsubsection{Realizability of Linkages}
% Suppose we had two configurations of a linkage, $\mathcal{A}$ and $\mathcal{B}$.  A question that can be posed is can we reconfigure $\mathcal{A}$ to $\mathcal{B}$ continuously while respecting simple planar graph conditions?  The answer to this question is a yes or no.  If yes, then there must exist a path connected configuration space between $\mathcal{A}$ and $\mathcal{B}$.  It has been shown that this problem can be posed as a planar satisfiability problem \cite{Breu19983,mulzer2008minimum} (Later on in this paper we'll cover satisfiability problems).  This is the type of problem that we face in this paper.  We will continue to explore this in a different manner, with circle packings.
\newpage 