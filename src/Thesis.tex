\documentclass[10pt]{article}
%%%packages%%%
\usepackage[latin1]{inputenc}
%\usepackage{newcent}
\usepackage{verbatim}
\usepackage{graphicx}
\usepackage{amsmath}
\usepackage{amsfonts}
\usepackage{amssymb}
\usepackage{amsthm}
\usepackage[section]{placeins}
\usepackage{listings}%R Code
\usepackage{color}%R code
\usepackage{tikz}
\usetikzlibrary{patterns}
% \usepackage{tkz-graph}
% \usepackage{tkz-berge}
% \usepackage{tkz-euclide}
% \usepackage{hyperref}
%\usepackage[left=1in, right=1in, top=.9in, bottom=.9in]{geometry}
\usepackage{setspace}
%\usepackage{hyperref}
\usepackage{pdflscape}
\usepackage[english]{babel}
\usepackage{times}
\usepackage[T1]{fontenc}
\usepackage{multirow}
\usepackage{mathptmx}
%\usepackage[hidelinks]{hyperref}
\usepackage{lipsum}
\usepackage{titletoc}
\usepackage[toc,page]{appendix}
\usepackage{appendix}
\usepackage{enumerate}
\usepackage{enumitem}
%\usepackage{subfigure}
\usepackage{caption}
\usepackage{subcaption}

%%CSUN Stuff%%%
\titlecontents{chapter}
  [0em]
   {\bfseries \addvspace{\baselineskip}}
  {\linebreak \rmfamily \chaptername\hspace{.1em} \thecontentslabel\quad \newline  }
  {\rmfamily}
  {\titlerule*[1pc]{.}\contentspage}

\titlecontents{section}
[1em]
   {}
  {\thecontentslabel \hspace{1em}}
   {}
    {\titlerule*[1pc]{.}\contentspage}

\titlecontents{subsection}
[3.3em]
   {}
  {\thecontentslabel \hspace{1.3em}}
   {}
    {\titlerule*[1pc]{.}\contentspage}

% for bibliography, abstract etc

\titlecontents{part}
  [0em]
   {\rmfamily \addvspace{\baselineskip}}
  {\linebreak \rmfamily \chaptername\hspace{.1em} \thecontentslabel\quad \newline  }
  {\rmfamily}
  {\titlerule*[1pc]{.}\contentspage}


\titlecontents{subsubsection}
[0em]
   {{\rmfamily \addvspace{\baselineskip}}}
  {\thecontentslabel \hspace{1em}}
   {}
    {\titlerule*[1pc]{}}
%%%END OF CSUN STUFF%%%
%%%theoerms, etc%%%%
\theoremstyle{plain}% default
\newtheorem{thm}{Theorem}[section]
\newtheorem{prop}{Proposition}[section]
\newtheorem{lem}{Lemma}[section]
\newtheorem{pf}{Proof}[section]
\theoremstyle{definition}
\newtheorem{definition}{Definition}[section]
\newtheorem{example}{Example}[section]
\theoremstyle{remark}
\newtheorem{rmk}{Remark}[section]
\newtheorem{case}{Case}
\newtheorem{prob}{Problem}[section]
%%%% R Code%%%%
\definecolor{dkgreen}{rgb}{0,0.6,0}
\definecolor{gray}{rgb}{0.5,0.5,0.5}
\definecolor{mauve}{rgb}{0.58,0,0.82}
\lstset{frame=tb,
  language=R,
  aboveskip=3mm,
  belowskip=3mm,
  showstringspaces=false,
  columns=flexible,
  basicstyle={\small\ttfamily},
  numbers=none,
  numberstyle=\tiny\color{gray},
  keywordstyle=\color{blue},
  commentstyle=\color{dkgreen},
  stringstyle=\color{mauve},
  breaklines=true,
  breakatwhitespace=true
  tabsize=3
}

%%%%%%%%TikZ Commands

\def\hexagonsize{1cm}
\pgfdeclarepatternformonly
  {hexagons}% name
  {\pgfpointorigin}% lower left
  {\pgfpoint{3*\hexagonsize}{0.866025*2*\hexagonsize}}%  upper right
  {\pgfpoint{3*\hexagonsize}{0.866025*2*\hexagonsize}}%  tile size
  {% shape description
   \pgfsetlinewidth{0.4pt}
   \pgftransformshift{\pgfpoint{0mm}{0.866025*\hexagonsize}}
   \pgfpathmoveto{\pgfpoint{0mm}{0mm}}
   \pgfpathlineto{\pgfpoint{0.5*\hexagonsize}{0mm}}
   \pgfpathlineto{\pgfpoint{\hexagonsize}{-0.866025*\hexagonsize}}
   \pgfpathlineto{\pgfpoint{2*\hexagonsize}{-0.866025*\hexagonsize}}
   \pgfpathlineto{\pgfpoint{2.5*\hexagonsize}{0mm}}
   \pgfpathlineto{\pgfpoint{3*\hexagonsize+0.2mm}{0mm}}
   \pgfpathmoveto{\pgfpoint{0.5*\hexagonsize}{0mm}}
   \pgfpathlineto{\pgfpoint{\hexagonsize}{0.866025*\hexagonsize}}
   \pgfpathlineto{\pgfpoint{2*\hexagonsize}{0.866025*\hexagonsize}}
   \pgfpathlineto{\pgfpoint{2.5*\hexagonsize}{0mm}}
   \pgfusepath{stroke}
  }
%%%%%%%%custom commands
\newcommand{\NN}{\mathbb{N}} %  set of natural numbers
\newcommand{\ZZ}{\mathbb{Z}} %  set of integer number
\newcommand{\RR}{\mathbb{R}} %  set of real numbers
\newcommand{\SH}{\mathbb{S}} %  set of unit vectors
\newcommand{\HH}{{\cal H}} %  Calligraphic H
\newcommand{\PP}{{\cal P}} %  Calligraphic P
\newcommand{\DD}{{\cal D}} %  Calligraphic D
\newcommand{\QQ}{{\cal Q}} %  Calligraphic D
\newcommand{\FF}{{\cal F}} %  Calligraphic D
\newcommand{\bbH}{{\mathbb{H}}}
\newcommand{\bbR}{{\mathbb{R}}}
\newcommand{\bbP}{{\mathbb{P}}}
\newcommand{\bbZ}{{\mathbb{Z}}}
\newcommand{\bbC}{{\mathbb{C}}}
\newcommand{\bbQ}{{\mathbb{Q}}}
\newcommand{\bbA}{{\mathbb{A}}}
\newcommand{\bbF}{{\mathbb{F}}}
\newcommand{\bbh}{{\mathbb{H}}}
\newcommand{\bbr}{{\mathbb{R}}}
\newcommand{\bbp}{{\mathbb{P}}}
\newcommand{\bbz}{{\mathbb{Z}}}
\newcommand{\bbc}{{\mathbb{C}}}
\newcommand{\bbq}{{\mathbb{Q}}}
\newcommand{\bba}{{\mathbb{A}}}
\newcommand{\bbf}{{\mathbb{F}}}
\newcommand{\bbn}{{\mathbb{N}}}
\newcommand{\bbN}{{\mathbb{N}}}
\newcommand{\disteq}{{\overset{D}{=}}}
\newcommand{\cross}{{\times}}
\newcommand{\CBeta}{{  \left( \begin{array}{c}\hat{\beta}_{1,1} - \hat{\beta}_{2,1} \\ \hat{\beta}_{1,2} -
\hat{\beta}_{2,2} \\ \vdots \\ \hat{\beta}_{1,p} - \hat{\beta}_{2,p}    \end{array} \right) }}
\newcommand{\COVW}{{\left[ \begin{array}{cc}\sigma_1^2 \left( X_1 ' X_1\right)^{-1}\\ \sigma_2^2 \left( X_2 '
X_2\right)^{-1} 
\end{array} \right]}}
\newcommand{\MSRES}{{\sigma_1^2 n_1 + \sigma_2^2 n_2 - p \left( \sigma_1^2 + \sigma_2^2 \right) }}
\newcommand{\XX}{{\left(X ' X\right)^{-1} }} 
\newcommand{\xx}{{\left(X ' X\right)^{-1} }} 
\newcommand{\ssres}{{\text{SS}_\text{RES}}}
\newcommand{\inv}[1]{{#1^{-1}}}
\renewcommand{\it}[1]{{\textit{#1}}}
% \newcommand{\iff}{{\Leftrightarrow}}
\newcommand{\comp}[2]{{\left( #1 \circ #2\right) }}
\newcommand{\set}[2]{{\left\lbrace \left.  #1 \left\vert #2  \right.\right.\right\rbrace  }}
\newcommand{\topo}{{\mathcal{T}}}
\newcommand{\powset}[1]{{\mathcal{P}\left( #1 \right) }}
%\newcommand{\vec}[1]{{\overrightarrow{#1} }}
%%%%Spacing commands %%%%%
\newcommand{\tab}{\hspace{.4cm}}
\newcommand{\quadtab}{\hspace{.4cm}}
\newcommand{\matab}{\hspace{1.01600mm}}
\renewcommand{\arraystretch}{1.5}
\newcommand{\RNum}[1]{\lowercase\expandafter{\romannumeral #1\relax}}
\newcommand{\rn}[1]{\lowercase\expandafter{(\romannumeral #1\relax)}}
\newcommand{\floor}[1]{\left\lfloor #1 \right\rfloor}
\newcommand{\ceil}[1]{\left\lceil #1 \right\rceil}
\newcommand{\combo}[2]{\left(\begin{array}{c}#1\\#2\end{array}\right)}
%\renewcommand{\baselinestretch}{1.0}
% 1.0 is for one line space, 2.0 is for double-line space, etc
%%%%Spacing commands %%%%%

%%%%Margins%%%%%% - Clinton Bowen
\setlength{\topmargin}{-.5in} 
\setlength{\textheight}{9.0in}
\setlength{\oddsidemargin}{0.5in} 
\setlength{\evensidemargin}{0.0in}
\setlength{\textwidth}{6.0in}
%%Please refer to http://en.wikibooks.org/wiki/LaTeX/Page_Layout
%%The parameters below are described pictorally on this webpage.  Tinker with the settins as needed.
%\setlength{\evensidemargin}{0cm}
%\setlength{\oddsidemargin}{0pt}
%\setlength{\topmargin}{.5in}
% \setlength{\hoffset}{0in}
% \setlength{\voffset}{0in}
% \setlength{\headheight}{0pt}
% \setlength{\headsep}{0in}%should be 1 inch from header titles
%\setlength{\textheight}{21cm}
%\setlength{\textwidth}{15.5cm}
% \setlength{\marginparsep}{0pt}
% \setlength{\marginparwidth}{0pt}
% \setlength{\footskip}{0pt}
%%%%Margins%%%%%% - Clinton Bowen
\author{Clinton Bowen}
\title{Protein Folding: Planar Configuration Spaces of Disc Arrangements and
Hinged Polygons: \textit{Protein Folding in Flatland}}
\date{April 1, 2014}
\makeindex
\begin{document}
% 
\pagenumbering{roman}
\thispagestyle{empty}
\begin{center}
CALIFORNIA STATE UNIVERSITY, NORTHRIDGE

\vspace{1.5in}
PROTEIN FOLDING: PLANAR CONFIGURATION SPACES\\
\vspace{15pt}
OF DISC ARRANGEMENTS AND HINGED POLYGONS:\\
\vspace{15pt}
PROTEIN FOLDING IN FLATLAND\\
\vspace{100pt}
A thesis submitted in partial fulfillment of the requirements\\
For the degree of Master of Science in Mathematics\\
\vspace{50pt}
By\\
\vspace{20pt}
Clinton Bowen \\
\vspace{50pt}
Spring 2014
\end{center}
\pagebreak

\addcontentsline{toc}{part}{Signature page}

\vspace*{30pt}
The thesis of Clinton Bowen is approved:

\vspace*{70pt}

\textbf{---------------------------------------------- \ \ \
\ \ \ \ \ \ \ \ \ \ \ \ \ \ ------------------}

Dr. John Dye \ \ \ \ \ \ \ \ \ \ \ \ \ \ \ \ \ \ \ \ \ \ \ \ \ \ \ \ \ \ \ \ \ \ \ \ \ \ \ \ \ \ \ \ \ \ \ \ \ \ \ \ \ \ \ \ \     Date

\vspace*{30pt}

\textbf{---------------------------------------------- \ \ \
\ \ \ \ \ \ \ \ \ \ \ \ \ \ ------------------}

Dr. Silvia Fernandez \ \ \ \ \ \ \ \ \ \ \ \ \ \ \ \ \ \ \ \ \ \ \ \ \ \ \ \ \ \ \ \ \ \ \ \ \ \ \ \ \ \ \ \ \  Date

\vspace*{30pt}

\textbf{---------------------------------------------- \ \ \
\ \ \ \ \ \ \ \ \ \ \ \ \ \ ------------------}

Dr. Bernardo Abrego \ \ \ \ \ \ \ \ \ \ \ \ \ \ \ \ \ \ \ \ \ \ \ \ \ \ \ \ \ \ \ \ \ \ \ \ \ \ \ \ \ \ \ \  Date

\vspace*{30pt}

\textbf{---------------------------------------------- \ \ \
\ \ \ \ \ \ \ \ \ \ \ \ \ \ ------------------}


Dr. Csaba Toth, Chair \ \ \ \ \ \ \ \ \ \ \ \ \ \ \ \ \ \ \ \ \ \ \ \ \ \ \ \ \ \ \ \ \ \ \ \ \ \ \ \ \ \ \ Date


\vspace*{225pt}

\begin{center}
California State University, Northridge
\end{center}

\pagebreak

\addcontentsline{toc}{part}{Dedication}
\begin{center}
DEDICATIONS
% \vspace{80pt}
% 
% To my family and friends, \\
% 
% \vspace{30pt}
% Thank you for all your support
\end{center}

\pagebreak

\addcontentsline{toc}{part}{Acknowledgement}
\begin{center}
ACKNOWLEGDEMENTS
\end{center}
\vspace{80pt}

% I would like to thank my thesis advisor Dr. Yomba and  co-chair Dr. Djellouli for their support  
% and encouragement throughout my thesis preparation.  
% Thanks also to my committee members Dr. Zakeri and Dr. Panferov. \\


\pagebreak

\renewcommand\contentsname{Table of Contents}
\tableofcontents

%\listoffigures
%\listoftables

\pagebreak
\addcontentsline{toc}{part}{Abstract}
\begin{center}
ABSTRACT \\
\vspace*{2em}
PROTEIN FOLDING: PLANAR CONFIGURATION SPACES\\
\vspace{15pt}
OF DISC ARRANGEMENTS AND HINGED POLYGONS:\\
\vspace{15pt}
PROTEIN FOLDING IN FLATLAND\\
\vspace{20pt}
By\\
\vspace{30pt}
Clinton Bowen \\
\vspace{30pt}
Master of Science in Mathematics 
\end{center}
\vspace{30pt}

Insert Abstract here


\pagebreak
\pagenumbering{arabic}
% \input{abstract}
% \begin{abstract}
We look into the decidability of whether a hinged configuration locks.
\end{abstract}
\section{Introduction}
We look into the decidability of continuity on planar configuration space using regular, unitary hexagonal polygons.  These polygons can also represent unit disk configurations \cite{Breu19983} 
\begin{figure}[h]
\begin{center}
\includegraphics[scale=.5]{graphics/7ballLocked.pdf}
\caption{A locked 7 ball configuration}
\label{figure:7ballLocked}
\end{center} 
\end{figure}
% We need the following:
% \begin{itemize}
% \item[\rn{1}] discussion on the outer configuration of hexagons
% \item[\rn{2}] discussion on the inner configuration (lattice tesselation) of hexagons the convex hull of \rn{1}
% \item[\rn{3}] discussion on the hinged cells that reside in the boundary of \rn{2}
% \item[\rn{4}] discussion on the lockedness of \rn{3} where it has two modes (boolean variable)
% \item[\rn{5}] the area of interests:
% \begin{itemize}
% \item how big are the cells of \rn{3}?
% \item does cell size matter in \rn{3}?
% \item How to form a 3-sat conjunction from a set of cells at corners of the individual hexagons within the tesselation
% \item how to form a conjunction from a set of cells on the edges of the of the hexagons.
% \end{itemize} 
% \end{itemize}  
\paragraph{Motivation}
Protein folding, graphite, crystalline structures in metallurgy; disc packing;
hexagonal configurations;  Determine whether chemical structures are realizable.
\paragraph{Outline}
Section 2 covers the necessary mathematical concepts to understanding the
problem.  Section 3 explains the problem, Section 4 covers the results and
findings about the problem.  Section 5, the conclusion, offers final remarks on
the problem.

% \section{Background}
Here we review some of the necessary mathematics behind the problem.  The
definitions found in this chapter are those found in 
\cite{kleinberg2006algorithm,
stephenson2005introduction,frederickson1997dissections}.
\section{Linkages}
\begin{figure}[!htbp]
 \begin{center}
  \includegraphics{graphics/HumanTurkeyLinkage.pdf}
  \caption{Here are skeleton drawings of a human and a turkey.  When animating skeletons, one tends to make sure that the lengths of the skeleton segments are kept the same length throught the animation.  Otherwise, the animation may depart from what is ideally understood of skeletal motions.}
 \end{center}
\end{figure}

When graph drawings model physical objects, other qualities avout the graph can be contextualized in a geometric sense.  
Distance, angular relationstips and other geometric qualities of the drawings can be other useful properties of the drawing to perform analysis on.
The \textit{length assignment} of a graph $G=(V,E)$ is $\ell:E \mapsto \bbr^+$. 
For simple graphs, length assignment must be strictly positive, otherwise it may result in two distinct vertices with the same coordinates.
A \textit{linkage} is a graph $G = (V,E)$ with a length assignment $\ell:E \mapsto \bbr^+$.  
Length assignments can be thought of as a metric where $\ell(u,v) = \ell(v,u)>0$.
%Inser linkage here

Consider embeddings of a graph that respects the length assignment.  
A \textit{realization} of a linkage, $(G,\ell)$, is an embedding of a graph, $\Pi$, such that for every edge $\{u,v\} \in E$, $\ell\left( \{u,v\} \right) = \left\vert \Pi(u) - \Pi(v) \right\vert$.  
A \textit{plane realization} is a plane embedding with the property, $\ell\left( \{u,v\} \right) = \left\vert \Pi(u) - \Pi(v) \right\vert$.
First let's define the space of realizations for a corresponding linkage, i.e.:
$$P_{(G,\ell)} = \set{\Pi_{(G,\ell)}}{\forall \{u,v\} \in E\text{, }\ell\left( \{u,v\} \right) = \left\vert \Pi(u) - \Pi(v) \right\vert}$$
With respect to $P$, we can establish the a \textit{configuration space} that allows one to study problems of motion.  For each vertex of $G$, the embedding of the vertex lies in the plane, i.e. $\Pi(v) \in \bbR^2$.  By enumerating each vertex of $G$, e.g. $v_1, v_2, \dots, v_k, \dots, v_{n}$, we can create a projection mapping from $\mu: P \mapsto \bbR^{2\vert V \vert}$ where the corresponding coordinates of $\Pi(v_k)$ are in the $(2k)^\text{th}$ and $(2k+1)^{th}$ coordinates in $\bbR^{2\vert V \vert}$.  The configuration space is $\mu(P)$.  

By using real analysis, we can begin to pose problems about linkages with respect to a corresponding configuration space.  
We define a path $\gamma: [0,1]\mapsto \mu(P)$ where $\gamma(0)$ corresponds the the projection of a realization of a linkage $\Pi_0$ and $\gamma(1)$ corresponds to another realization of a linkage $\Pi_1$.  
If for any two elements $a,b \in \mu(P)$ that there exists a continuous path $\gamma$ such that $\gamma(0)=a$ and $\gamma(1)=b$, $\mu(P)$ is said to be path connected.   
For $\gamma$ to be continuous we would have that for every $\epsilon > 0$, there exists a $\delta >0$ such that if $x,y \in [0,1]$ and $\vert x-y \vert \delta$ then $\vlr{\vlr{\gamma(x)-\gamma(y)}}<\epsilon$.
$\gamma$ can be thought of as an animation of drawings that starts at $\gamma(0)$ and ends at $\gamma(1)$
To ask if $\mu(P)$ is a connected space, is to ask if $\mu(P)$ is connected in $\bbR^{2\vert V \vert}$.

\subsection{Configuration Spaces of Linkages}

\textbf{NOTE THAT THIS SUBSECTION MAY HAVE REPEATED CONTENT}

Let's focus on the space of embeddings of a linkage. If there are $n$ vertices of a linkage, 
the \textit{configuration space} of a linkage is said to be a vector space of dimension $2 \cdot n$ 
where edge length is preserved.  
\begin{figure}[!h]
\begin{center}
\includegraphics{graphics/twoEmbeddingsOfSameLinkage.pdf}
\end{center} 
\caption{(a) and (b) show a linkage in two embeddings.}
\label{fig:configuration-3}
\end{figure}
A \textit{configuration space} for a linkage $G$ and corresponding proper embedding, $L_1$ is said 
to be for any other proper embedding of a linkage $G$, $L_2$, such that the lengths 
of every edge of $G$ is preserved between the two embeddings, i.e.: 
$$l\left( \left(u,v\right) 
\right) = \left\vert 
L_1(u) - L_1(v) \right\vert = \left\vert L_2(u) - L_2(v) \right\vert$$
Equivalent embeddings include translations and rotations about the center of mass on $L(V)$.  We 
further our embeddings by requiring that one vertice is pinned to the point of origin on the plane 
as well as a neighboring vertex.

\begin{thm}[\cite{CDR03,Str05}Carpenter's Rule Theorem]
 Every realization of a linkage can be continuously moved (without 
self-intersection) to any other 
realization. In other words, the realization space of such a linkage is always connected.
\end{thm}

\begin{figure}[!h]
\begin{center}
\includegraphics[scale=.75]{graphics/LockedConnellyLinkage.pdf}
\end{center} 
\caption{A linkage whose complete configuration space is discontinuous.  These two examples above 
are two configurations of the same linkage that cannot continuously transform into the other 
without edge crossing.}
\label{fig:configuration-4}
\end{figure}
A \textit{reconfiguration} of a linkage whose graph is $G=(V,E)$ and length assignment is $\ell$ is 
a continuous function $f: [0,1] \mapsto \bbr^{2 \cdot \vert V \vert}$ specifying a configuration of 
the linkage for every $t \in [0,1]$ where length assignment $\ell$ is preserved, edges do not cross 
and for every $\epsilon > 0$, there exists a $\delta > 0$ such that $\vert t_1 - t_2 \vert < 
\delta$ implies 
$$\left\vert f\left( t_1 \right) - f\left( t_2 \right) \right\vert < \epsilon$$

%(fig 1) insert a table of a graph and define a length assigment 
%(fig 2) insert a realization of (fig 1)
%(fig 3) insert a second realization of (fig 1)



%graph component of the linkage   the plane.  A linkage 
%\textit{embedding} is $L : V \mapsto 
%\bbR^{2}$.
% A \textit{linkage} is an ordered pair $G = (V,E)$ comprising of a set $V$ of vertices or nodes 
% together with a set $E$ of edges or lines. This definition is commonly used for graphs.  Mapping 
% the linkage $G$ into the plane is said to be the \textit{embedding}, i.e. $L : V \mapsto 
% \bbR^{2}$.  A length function correspond to a linkage, $l: E \mapsto \bbr^+$ gives a length to an 
% edge in the linkage.  If We consider a \textit{realization} of a linkage is range of $L$, i.e. 
% $L(V)$. If for every edge $(u,v) \in E$ such that $l\left( \left(u,v\right) \right) = \left\vert 
% L(u) - L(v) \right\vert$ is true, then $L$ is said to be a \textit{proper embedding} of $G$.
% \begin{figure}[h]
% \begin{center}
% \includegraphics[scale=1]{graphics/crossingEdgeLinkage.pdf}
% \end{center} 
% \caption{A linkage where edges cross however it does not contain loops or multiple edges between 
% vertices.}
% \label{fig:linkage-3}
% \end{figure}



\subsection{Circle Packing}
\begin{definition}[Circle Packing]\label{def:circlePacking}
$P$ of a planar graph $G$ is a set of of circles with disjoint
interiors $\left\lbrace C_v \right\rbrace_{v \in G} $ such that two
circles are tangent if and only if the corresponding vertices form an edge.
\cite{arXiv13113363v1}
\end{definition} 


\begin{thm}[Circle Packing Theorem]\label{thm2-1}
For every connected simple planar graph $G$ there is a circle packing in the
plane whose intersection graph is (isomorphic to) $G$.
\end{thm}
\begin{figure}[h]
\begin{center}
\includegraphics[scale=.5]{graphics/circlePackingTheoremExample.pdf}
\end{center} 
\caption{This figure is an example of a circle packing for the given simple planar graph.}
\end{figure} 
\textbf{Are all linkages simple planar graphs?}  A proof of Theorem \ref{thm2-1} is found in \cite{stephenson2005introduction}.
\subsubsection{Circle Packings and Polygonal Linkages}
Given a circle of radius $r$, we establish the isomorphism to a hexagon by
circumscribing the vertices of the regular hexagon.
\begin{figure}[h]
\begin{center}
\includegraphics{graphics/circumscribedHexagon.pdf}
\caption{A circumbscribed hexagon}
\end{center}
\end{figure}
\subsubsection{Hinged Polygons}
\begin{definition}[Polygonal Chain]\label{def}
A polygonal chain $P = \left( v_0, v_1, \dots, v_{n-1}\right) $ is a sequence of
consecutively joined segments (or edges) $e_i = v_i v_{i+1}$ of fixed lengths
$l_i = \left\vert e_i\right\vert $, in a plane. \cite{Biedl99lockedand}
\end{definition}
A chain is said to be closed if $v_{n-1} = v_1$, otherwise it is said to be
open. Hinged polygons have been researched for decades and related to linkage problems
\cite{Biedl99lockedand,canny1988complexity}.

Consider the locked configuration of figure \ref{figure:7hexLocked}.  We can
 configure the hexagons to be locked by placing hinged points as follows:
\begin{figure}[h]
\begin{center}
\includegraphics[scale=.33]{graphics/7hexLocked.pdf}
\caption{A locked 7 hexagonal configuration.  (needs to modify picture by
placing red points for hing points.)}
\label{figure:7hexLocked}
\end{center} 
\end{figure}
To prove that it is a locked configuration:
\begin{itemize}
 \item[\rn{1}]
 \item[\rn{2}]
 \item[\rn{3}]
 \item[\rn{4}]
 \item[\rn{5}]
 \item[\rn{6}]
 \item[\rn{7}]
 \item[\rn{8}]
 \item[\rn{9}]
 \item[\rn{10}]
 \end{itemize}
\subsubsection{Hinged Hexagons}
\begin{thm}[]\label{thm}
Any finite collection of polygons of equal area has a common hinged dissection.
\cite{abbott2012hinged}
\end{thm}
\paragraph{The Shapes}
Figure \ref{fig:lockingShape} is a locking shape:
\begin{figure}[h]
\begin{center}
\includegraphics{graphics/lockingShape.pdf}
\caption{This is the shape that resides in boundary of the lattice.}
\label{fig:lockingShape}
\end{center}
\end{figure}
Figure \ref{fig:lockingShape} shall reside in the boundary of a lattice and have
a hinge point at one vertex where the locking shape and boundary meet.
\begin{figure}[h]
\begin{center}
\includegraphics{graphics/shapeInChannel.pdf}
\end{center} 
\caption{A locking shape in the lattice boundary's channel.}
\label{fig:lockingShapeInChannel}
\end{figure}
\paragraph{Junctions}
We define junctions to be the point three hexagons meet in a hexagonal lattice,
e.g. Figure \ref{fig:lattice}.
%Radius of regular polygons 
\newdimen\R
\R=4.5cm
\begin{figure}[h] 
\begin{center}
\begin{tikzpicture}
\begin{scope}
\filldraw[pattern=hexagons]  (0:\R) \foreach \x in {60,120,...,359} {
                -- (\x:\R)
            }-- cycle (90:\R);
\end{scope}
\end{tikzpicture}
\caption{A portion of a hexagonal lattice.}
\label{fig:lattice}
\end{center}
\end{figure}
\newpage
\paragraph{Central Scaling}
\paragraph{Junctions in Conjunctive Normal Form}
Explain the configurations we're interested in.

\section{Configuration Spaces of Polygonal Chains}
\subsubsection{Configurations and Locked Configurations}
\subsection{Dissections}
\begin{prob}[Polygonal Dissection]\label{def:dissection}
Given two polygons of equal area, $P_1$ and $P_2$, partition $P_1$ into smaller
pieces,$\left\lbrace P_{1,i}\right\rbrace_{i=1}^n $, rearrange the pieces to
form $P_2$. \cite{frederickson1997dissections}
\end{prob}
\begin{figure}[h]
\begin{center}
\includegraphics[scale=1]{graphics/polygonaldissection.png}
\caption{An axample of two polygons of equal area that can be rearranged into
the other by the given partition.\cite{davidEppstienJunkyard}}
\label{fig:polygonaldissection}
\end{center}
\end{figure}
\subsection{SAT Problems}
\begin{prob}[Satisfiability Problem]\label{prob:ncpi-6}%Problem/Question
Let $\left\lbrace x_i \right\rbrace_{i=1}^{n} $ be boolean variables, and $t_i \in \left\lbrace x_i\right\rbrace_{i=1}^{n}  \cup \left\lbrace \bar{x}_i\right\rbrace_{i=1}^{n}   $.  A \textit{clause} is is said to be a disjuction of distinct terms:
$$
t_1 \vee \cdots \vee t_{j_k} = C_k
$$
Then the \textit{satisfiability problem} is the decidability of a conjuction of a set of clauses, i.e.:
$$ \wedge_{i=1}^m C_i$$
\end{prob} \cite{skiena2009algorithm}
A \textit{3-SAT problem} is a SAT problem with all clauses having only three boolean variables. 
\begin{definition}[Planar 3-SAT Problem]\label{def:sat-1}
Given a boolean 3-SAT formula $B$, define the associated graph of $B$ as follows:  
\begin{equation}\label{eqn:sat-1}
G(B) = \left(\set{v_x}{v_x\text{ represents a variable in }B} \cup \set{v_C}{v_C\text{ represent a clause in }B}  , \set{\left( v_x, v_C\right) }{x \in C \text{ or } \bar{x} \in C}  \right) 
\end{equation} 
If $G(B)$ in equation (\ref{eqn:sat-1}) is planar, then $B$ is said to be a \textit{Planar 3-SAT Problem} \cite{mulzer2008minimum}.
\end{definition} 
\begin{figure}[!h]
\begin{center}
\includegraphics{graphics/LeftSwitchBetweenTwoPolygons.pdf}
\caption{left switch between polygons}
\end{center} 
\end{figure} 
\begin{figure}[!h]
\begin{center}
\includegraphics{graphics/RightSwitchBetweenTwoPolygons.pdf}
\caption{right switch between polygons}
\end{center} 
\end{figure} 
% \section{Problem}
\subsection{Problem Statement} text
\subsection{Decidability of Problem} test
\subsection{Locked Configuration}
Test
\begin{figure}[ht]
\begin{center}
\includegraphics{graphics/7ballLocked.pdf}
\caption{A locked 7 ball configuration}
\end{center} 
\end{figure}
\newpage
% \input{solution}
% \section{Conclusion}We conclude\ldots
Complex structures in nature are often composed of elementary pieces that obey simple local composition rules. Molecular biology, nanomanufacturing, and self-assembly are prime examples. Mathematical models for this phenomenon typically rely on rigidity theory and formal languages. In this paper, we study the realizability of complex structures that are given with a local specification. We consider two models in Euclidean plane.

\begin{enumerate}
\item A \textbf{polygonal linkage} is a set $\PP$ of convex polygons, and a set $\HH$ of hinges,
where each hinge $h\in \HH$ corresponds to two points on the boundary of two distinct polygons.
A \emph{realization} of a polygonal linkage is an interior-disjoint placement of congruent copies of the polygons in $\PP$ such that the points corresponding to each hinge are identified (Fig.~\ref{fig:1}, left).
\item A \textbf{disk arrangement} is a set $\DD$ of pairwise interior-disjoint disks in the plane. The contact graph of a disk arrangement $\DD$ is a graph $G=(\DD,E)$ where two vertices are adjacent if the corresponding disks intersect (kiss). A \emph{realization} of a vertex-weighted graph $G$ as a contact graph of disks is a disk arrangement whose contact graph is $G$ and the radius of each disk is the corresponding vertex weight.
\end{enumerate}
\begin{figure}[htbp]
  \centering
 \includegraphics[width=0.95\textwidth]{graphics/fig1}
\caption{\small (a) A set of convex polygons and hinges. (b) A realization of the polygonal linkage from (a).
(c) A graph $G$ with vertex weights $r_1,\ldots, r_8$. (d) A disk arrangement that realizes the 
weighted graph $G$ as a contact graph with radii equal to the corresponding weights.}
  \label{fig:1}
\end{figure}
Each model has two variants, depending on whether \emph{reflection} is allowed for the realization of each piece independently. For polygonal linkages, an \emph{oriented realization} requires translated and rotated copies of the polygons in $\PP$ (i.e., reflection is not allowed). An \emph{ordered contact graph} for a disk arrangement is a \emph{plane graph} $G$, where the circular order of the neighbors of each vertex is specified, and an \emph{oriented realization} is disk arrangement with the given ordered contact graph.


\smallskip\noindent{\bf Related Previous Work.}
Polygonal linkages (or body-and-joint frameworks) are a generalization of classical linkages (bar-and-joint frameworks) in rigidity theory. A linkage is a graph $G=(V,E)$ with given edge lengths. A realization of a linkage is a (crossing-free) straight-line embedding of $G$ in the plane.
Bhatt and Cosmadakis~\cite{BC87} proved that the realizability of linkages is NP-hard.
Their ``logic engine'' method~\cite{SFM+11,BET+99,FHW97,HK01}, has become a powerful tool in graph drawing.
The logic engine is a graph composed of rigid 2-connected components, connected by cut vertices (hinges). The two possible realizations of each 2-connected component (that differ by a single reflection)  represent the truth assignment of a binary variable. This method does not applicable to the \emph{oriented} version of the realizability, where the circular order of the neighbors of each vertex is part of the input. Cabello et al.~\cite{CDR07,EW90} proved that the realizability of 3-connected linkages (where the orientation is unique by Steinitz's theorem) is NP-hard, but efficiently decidable for near-triangulations~\cite{CDR07,BV96}.

Note that every \emph{tree} linkage can be realized in $\RR^2$ (with almost collinear edges). According to the celebrated \emph{Carpenter's Rule Theorem}~\cite{CDR03,Str05}, every realization of a path (or a cycle) linkage can be continuously moved (without self-intersection) to any other realization. In other words, the realization space of such a linkage is always connected. However, there are trees of maximum degree 3 with at few as 8 edges whose realization space is disconnected~\cite{BCD+09}; and deciding whether the realization space of a tree linkage
is connected is PSPACE-complete~\cite{AKR+04}. (Earlier, Reif~\cite{Rei79} showed that it is PSPACE-complete to decide whether a polygonal linkage can be moved from one realization to another among polygonal obstacles in $\RR^3$.) Cheong et al.~\cite{CdG+07} considers the ``inverse'' problems of introducing the minimum number of point obstacles to reduce the configuration space of a polygonal linkage to a unique realization.


Connelly et al.~\cite{CDD+10} showed that the Carpenter's Rule Theorem generalizes to certain polygonal linkages, which are obtained by replacing the edges of a path linkage with special polygons called (\emph{slender adornments}). Our Theorem~\ref{thm:hinge} indicates that if we are allowed to replace the edges of a path linkage with arbitrary convex polygons, then deciding whether the realization space is empty or not is already NP-hard.

Recognition problems for intersection graphs of various geometric object have a rich history~\cite{HK01}. Breu and Kirkpatrick~\cite{BK98} proved that it is NP-hard to decide whether a graph $G$ is the contact graph of unit disks in the plane (a.k.a. recognizing \emph{coin graphs} is NP-hard). A simpler proof was later provided via the logic engine~\cite{BET+99}. It is also NP-hard to recognize the contact graphs of pseudo-disks~\cite{HK01} and disks of bounded radii~\cite{BK95} in the plane, and unit disks in higher dimensions~\cite{Hli97,HK01}. All these hardness reductions produce graphs of high genus, and do not apply to trees. Note that the contact graphs of disks (of arbitrary radii) are exactly the planar graph (by Koebe's circle packing theorem), and planarity testing is polynomial. Consequently, every tree is the contact graph of disks of \emph{some} radii in the plane.

\section{Linkages}
A \textit{linkage} is an ordered pair $G = (V,E)$ comprising of a set $V$ of vertices or nodes 
together with a set $E$ of edges or lines. This definition is commonly used for graphs.  Mapping 
the linkage $G$ into the plane is said to be the \textit{embedding}, i.e. $L : V \mapsto 
\bbR^{2}$.  A length function correspond to a linkage, $l: E \mapsto \bbr^+$ gives a length to an 
edge in the linkage.  If We consider a \textit{realization} of a linkage is range of $L$, i.e. 
$L(V)$. If for every edge $(u,v) \in E$ such that $l\left( \left(u,v\right) \right) = \left\vert 
L(u) - L(v) \right\vert$ is true, then $L$ is said to be a \textit{proper embedding} of $G$.
Without loss of generality, for this paper, we focus on linkages that are simple planar  graphs, 
i.e.:
\begin{itemize}
\item[\rn{1}] does not have edges that cross,
\item[\rn{2}] have loops, i.e. $(v,v) \in E$, or
\item[\rn{3}] does not have multiple edges between any pair of vertices.
\end{itemize}
We may visit special cases in which we look at planar graphs that satisfy the last two conditions 
but not the first, e.g.:
\begin{figure}[h]
\begin{center}
\includegraphics[scale=1]{graphics/crossingEdgeLinkage.pdf}
\end{center} 
\caption{A linkage where edges cross however it does not contain loops or multiple edges between 
vertices.}
\label{fig:linkage-3}
\end{figure}
\section{Polygonal Linkages}
\begin{figure}[h]
\begin{center}
\includegraphics[scale=1]{graphics/hingeOnThreeDistinctPolygons.pdf}
\end{center} 
\caption{(a) A polygonal linkage with a non-convex polygon and two hinge points corresponding to 
three polygons.  Note that hinge points correspond to two distinct polygons.(b) Illustrating that 
two hinge points can correspond to the same boundary point of a polygon.}
\label{fig:linkage-1}
\end{figure}
%describe how it is a generalization of Linkages.
Polygon linkages are similar to linkages.  They are an ordered pair of sets, with the exception 
that the set of vertices become a set of hinges, $\HH$, and the set of edges become a set of 
polygons, $\PP$.  Formally, a \it{polygonal linkage} is an ordered pair, $G = (\HH,\PP)$,  
comprises of a set of polygons, $\PP$, and a set of hinge points $\HH$ where each hinge $h \in \HH$ 
corresponds to two points on the boundary of two distinct polygons in $\PP$. Mapping the linkage 
$G$ into the plane is said to be the \textit{embedding}, i.e. $L : \HH \mapsto \bbR^{2}$.  We 
consider a \textit{realization} of a polygonal linkage is range of $L$, i.e. $l(\HH)$.

In figure (\ref{fig:linkage-1}), we illustrate that two hinge points can reside on the same point 
in the plane. Figure (\ref{fig:linkage-1}) and figure (\ref{fig:linkage-3}) are examples of special 
cases that we may run into, but do not want to focus heavily on.  They are presented to the reader 
to facilitate understanding of the definitions of polygonal linkages and linkages respectively.  
Without loss of generality, for this paper, we focus on polygonal linkages that are 
equivalent to simple planar graphs.
%Foigr each hinge point $h \in \HH$, there 
%exists some subset $P \subset \PP$ with 
%at least cardinality of 2 such that for each polygon $p \in P$, $h$ corresponds to some point on 
%the boundary of $p$.  When the polygonal linkage is embedded in the plane, the hinge point is 
%where 
%the polygons of $P$ \it{kiss}.  
% show an example of polygonal linkages
\begin{figure}[h]
\begin{center}
\includegraphics[scale=1]{graphics/PolygonalLinkageExamples.pdf}
\end{center} 
\caption{(a) A polygonal linkage with a non-convex polygon and a hinge point corresponding to three 
polygons.  (b) A polygonal linkage with 8 regular polygons.}
\label{fig:linkage-2}
\end{figure}
For the remainder of this thesis, we'll focus on the polygonal linkages with the following 
restrictions:
\begin{enumerate}
 \item  For each hinge point $h \in \HH$, there exists some subset $P \subset \PP$ with 
exactly cardinality of 2, and
\item every polygon in $\PP$ is convex.
\end{enumerate}
Formally, we define a \it{polygonal linkage} as an ordered pair $L = (\HH,\PP)$ comprising of a 
set of hinges, $\HH$, where each hinge $h\in \HH$ corresponds to two points on the boundary of two 
distinct polygons. A \emph{realization} of a polygonal linkage is an interior-disjoint placement of 
congruent copies of the polygons in $\PP$ such that the points corresponding to each hinge are 
identified (Fig. \ref{fig:1}). 

%With these restrictions of our isomorphisms, we can begin to describe a range of motion to 
%transform a linkage.  That range of motion is said to be the configuration space of that linkage.  
%To expand on this concept, for given linkage, $L=(V,E)$, and for a given vertex $v \in V$, the set 
%of points in which $v$ can be realized in the plane would be the configuration space for that 
%vertex, $C_v$.  Defining some order of the vertices in $L$, i.e. $V = \left\lbrace v_n 
%\right\rbrace_{i=1}^n$, then the \it{configuration space} for $L$ is said to be the cartesion 
%product of the configuration space of vertices:

%DESCRIBE THE FOLLOWING:
%1)CONFIGURATION SPACE AS A VECTOR SPACE OF DIMENSION 2^N WHERE EDGE LENGTH IS PRESERVED.
%2)PINNING 1 VERTEX TO ORIGIN AND A NEIGHBOR, ADD MOTIVATION TO PREVENT ROTATION AND TRANSLATIONS.

% \begin{equation}\label{eqn:linkages-1}
% C(L) = C_{v_1} \cross C_{v_2} \cross \cdots \cross C_{v_n}
% \end{equation} 
% Some food for thought on configuration spaces and motions on linkages:
% \begin{itemize}
% \item[\rn{1}] A configuration space is said to be \it{connected} if there is a continuous mapping for any two planar realizations (linkages) of a graph in the plane.  Otherwise it is said to be \it{disconnected}.
% \item[\rn{2}] If the configuration space of a vertex, $C_v$, is a singleton set, then the vertex is said to be \it{pinned}. Otherwise it is said to be \it{free}.
% \item[\rn{3}] The types of motions (mappings) that we refrain from using on linkages are translations.
% \end{itemize}
% Note that configuration spaces for polygonal linkages are described similarly.
% \subsubsection{Realizability of Linkages}
% Suppose we had two configurations of a linkage, $\mathcal{A}$ and $\mathcal{B}$.  A question that can be posed is can we reconfigure $\mathcal{A}$ to $\mathcal{B}$ continuously while respecting simple planar graph conditions?  The answer to this question is a yes or no.  If yes, then there must exist a path connected configuration space between $\mathcal{A}$ and $\mathcal{B}$.  It has been shown that this problem can be posed as a planar satisfiability problem \cite{Breu19983,mulzer2008minimum} (Later on in this paper we'll cover satisfiability problems).  This is the type of problem that we face in this paper.  We will continue to explore this in a different manner, with circle packings.
\newpage 
\section{Configuration Spaces}
\subsection{Configuration Spaces of Linkages}
Given a linkage, $G$, consider its corresponding set of embeddings $\FF_G$.  
\subsection{Configuration Spaces of Polygonal Linkages}
%Show example an example of a locked linkage and locked polygonal linkage.

% To describe the types of motion that we are interested in linkages we must define the graph 
% isomorphism.  Two graphs $G=(V_1,E_1)$ and $G_2 = (V_2,E_2) $, a bijection $f: V_1 \mapsto V_2$ 
% such that for any two vertices $u,v \in V_1$ that are adjacent, i.e. $(u, v) \in E_1$, if and only 
% if $(f(u),f(v)) \in E_2$. 
\begin{table}[!ht]
\begin{center}
$$\begin{array}{|c|c|c|}\hline
\text{Graph}&\text{Vertices}&\text{Edges}\\\hline
G_1&\left\lbrace a,b,c,d,e \right\rbrace & \left\lbrace (a,b),(b,c),(c,d),(d,e),(e,a) \right\rbrace 
\\\hline
G_2&\left\lbrace 1,2,3,4,5 \right\rbrace & \left\lbrace (1,2),(2,3),(3,4),(4,5),(5,1) \right\rbrace 
\\\hline
\end{array} $$
\caption{Two graphs that are isomorphic with the alphabetical isomorphism $f(a)=1$, $f(b)=2$, $f(c) 
= 3$, $f(d)=4$, $f(e)=5$.}
\end{center} 
\label{table:linkage-1}
\end{table} 
% show an example of polygonal linkages
\begin{figure}[!h]
\begin{center}
\includegraphics[scale=1]{graphics/graphIsomorphismExample.pdf}
\end{center} 
\caption{This figure depicts the graph isomorphism shown in Table (\ref{table:linkage-1})between 
$V_1$ and $V_2$ in the plane.}
\label{fig:linkage-3}
\end{figure}
Next we add restrictions to our graph isomorphisms to narrow our focus:
\begin{itemize}
\item[\rn{1}] We focus on isomorphisms for planar graphs and or polygonal linkages, simple planar 
graphs, and
\item[\rn{2}] the isomorphism preserves edge lengths (polygonal area), e.g. $d(u,v) = d(f(u),f(v))$.
\end{itemize}  


















\subsubsection{Confining Linkages to a Restricted Space Within a Configuration Space}
So we've covered the idea of linkages within a plane; now let's constrain the plane to a strip and have a linkage that is a \textit{polygon}, i.e. a linkage that forms a closed chain (e.g. Table \ref{table:linkage-1}), hugging the boundaries of the strip:
\begin{figure}[h]
\begin{center}
  ~ %add desired spacing between images, e. g. ~, \quad, \qquad etc.
    %(or a blank line to force the subfigure onto a new line)
  \begin{subfigure}[b]{0.49\textwidth}
	  \includegraphics[width=\textwidth]{graphics/hexagonInChannelWithPinnedJointRight.pdf}
	  \caption{A bounded hexagon that resides in a channel with a pinned vertex}
	  \label{fig:linkage-1-1}
  \end{subfigure}
  \begin{subfigure}[b]{0.49\textwidth}
	  \includegraphics[width=\textwidth]{graphics/hexagonInChannelWithPinnedJointLeft.pdf}
	  \caption{The second realization of the hexagon residing in a channel with a pinned vertex.}
	  \label{fig:linkage-1-2}
  \end{subfigure}
\end{center} 
\caption{Due to the strip in the plane that the hexagon is bounded within the configuration space is limited to just two realizations.}\label{fig:linkage-1}
\end{figure}
So here we have a linkage whose conifguration space is limited to just two realizations.  With just two realizations, we can assign a binary value to them and have the linkage act as a boolean variable.  We will revisit this concept when we cover satisfiability problems later on in the paper.
\begin{figure}[h]
\begin{center}
  ~ %add desired spacing between images, e. g. ~, \quad, \qquad etc.
    %(or a blank line to force the subfigure onto a new line)
  \begin{subfigure}[b]{0.49\textwidth}
	  \includegraphics[width=\textwidth]{graphics/switchTerminalFinalized2.pdf}
	  \caption{A pentagon that is pinned in a channel junction that is formed by the sides of 3 large regular hexagons. It has two possible configurations, much like that of \ref{fig:linkage-1}}
	  \label{fig:linkage-2-1}
  \end{subfigure}
  \begin{subfigure}[b]{0.49\textwidth}
	  \includegraphics[width=\textwidth]{graphics/switchTerminalFinalized3.pdf}
	  \caption{A pinned pentagon residing in a channel junction that is formed by the sides of 3 large regular hexagons with 2 dashed pentagons intersecting it.}
	  \label{fig:linkage-2-2}
  \end{subfigure}
\caption{Suppose the channel formed is a junction of three regular hexagons.  The polygon partially residing in the junction is a regular hexagon with an equalateral triangle appended at an edge.  This polygon would prevent other polygons (i.e. the dashed polygons) of the same shape residing in the center of the channel without intersection. This demonstrates that a the configuration space within a multichannel environment can have concurrency issues, i.e. some configurations cannot be realizable.}
\end{center} \label{fig:linkage-2}
\end{figure}\newpage
Expanding upon the ideo of \ref{fig:linkage-1}, forming channels with junctions as shown in Figure \ref{fig:linkage-2} can be formed as such by evenly spacing the edges of a hexagonal lattice.  Visually, it is shown that only one of three possible pentagons can reside in the channel at one time.  By asserting certain conditions on the lattice, and extending the problem to a greater region of a hexagonal lattice, we will be able to pose a realizability problem of whether a configuration $\mathcal{A}$ can be reconfigured to $\mathcal{B}$ by switching pentagons without violating overlapped polygon conditions.
%Radius of regular polygons 
\newdimen\R
\R=3cm
\begin{figure}[h] 
\begin{center}
\begin{tikzpicture}
\begin{scope}
\filldraw[pattern=hexagons]  (0:\R) \foreach \x in {60,120,...,359} {
                -- (\x:\R)
            }-- cycle (90:\R);
\end{scope}
\end{tikzpicture}
\caption{A hexagonal lattice contained in a hexagon.}
\label{fig:lattice}
\end{center}
\end{figure}
\newpage

% \begin{definition}[Graph]\label{def:linkages-2}
% An ordered pair $G = (V, E)$ comprising a set $V$ of vertices or nodes together with a set $E$ of edges or lines
% \end{definition} 
% \begin{definition}[Linkage]\label{def:linkages-1}
% A collection of fixed-length 1D segments joined at their endpoints to form a graph.
% \end{definition} 
% A linkage can be thought of as a type of path-connected graph, i.e. the segments of a linkage are the edges of a graph, and the endpoints of the segments are the vertices. For this paper, we restrict our self to linkages that are simple planar graphs, i.e. a linkage that:
% \begin{itemize}
% \item[\rn{1}] does not have multiple edges between any pair of vertices,
% \item[\rn{2}] does not have edges that cross, or
% \item[\rn{3}] have loops (i.e. $(v,v) \in E$).
% \end{itemize}  
% \begin{definition}[Cycle]\label{def:linkages-3}
%  A closed walk with no repetitions of vertices or edges allowed, other than the repetition of the starting and ending vertex
% \end{definition} 
% \begin{definition}[Configuration]\label{def:linkages-6}
% A specification of the location of all the link endpoints, link orientations and
% joint angles.\cite{demaine2008geometric}
% \end{definition}
% \begin{definition}[Configuration Space]\label{def:linkages-7}
% The space of all configurations of a linkage.
% \end{definition} 
% A configurations space is said to be continuous if for any two configurations, $\mathcal{A}$ and $\mathcal{B}$ of a linkage $L$, $\mathcal{A}$ can be continuously reconfigured to $\mathcal{B}$ such that, the reconfigurations reside in the configuration domain, $L$ remains rigid throughout reconfiguration (i.e. all links' lengths are preserved), and no violations of linkage intersection conditions. 
% \begin{definition}[Pinned Joint]\label{def:linkages-8}
% A vertex of a graph (or linkage) that is fixed to a position in a plane.
% \end{definition} 
% \begin{definition}[Free Joint]\label{def:linkages-8}
% A vertex of a graph (or linkage) that is not fixed to a position in a plane.
% \end{definition} 
% \begin{figure}[h]
% \begin{center}
% \includegraphics[scale=.5]{graphics/randomLinkage.pdf}
% \end{center} 
% \caption{A linkage with joints.}
% \end{figure} 
% \begin{figure}[h]
% \begin{center}
% \includegraphics{graphics/freeJointPinnedJoint.pdf}
% \end{center} 
% \caption{The cross represents a free joint; the pinned joints are denoted as disks.  The range of motion shown by the arc describes the continous configuration space of the linkage.}
% \end{figure} 
% 
% For illustrations in the remainder of this paper, free joints will be represented as crosses and pinned joints will be represented as disks.

\section{Disk Arrangements}

\subsection{Oriented Realizations}
\subsection{Disk Packing Confinement Problem}
\begin{figure}[h]
\begin{center}
  ~ %add desired spacing between images, e. g. ~, \quad, \qquad etc.
    %(or a blank line to force the subfigure onto a new line)
  \begin{subfigure}[b]{0.1\textwidth}
	  \includegraphics[width=\textwidth]{graphics/degree2arrangement.pdf}
	  \caption{A disk arrangement with two layers of disks}
	  \label{fig:linkage-1-1}
  \end{subfigure}
  \begin{subfigure}[b]{0.4\textwidth}
	  \includegraphics[width=\textwidth]{graphics/degree3arrangement.pdf}
	  \caption{A disk arrangement with three layers of disks}
	  \label{fig:circlePacking1-2}
  \end{subfigure}
  \begin{subfigure}[b]{0.4\textwidth}
	  \includegraphics[width=\textwidth]{graphics/degree4arrangement.pdf}
	  \caption{A disk arrangement with four layers of disks}
	  \label{fig:circlePacking1-2}
  \end{subfigure}
\end{center} 
\caption{The gradual growth of disk arrangements by adding two kissing disks to each of the previously generated disks.  By continuing this arrangement growth, the space needed to contain the kissing disks will exceed the area containing the disk arrangements.}\label{fig:circlePacking-1}
\end{figure}

\subsection{Satisfiability}
\begin{prob}[Satisfiability Problem]\label{prob:Satisfiability-1}%Problem/Question
Let $\left\lbrace x_i \right\rbrace_{i=1}^{n} $ be boolean variables, and $t_i \in \left\lbrace 
x_i\right\rbrace_{i=1}^{n}  \cup \left\lbrace \bar{x}_i\right\rbrace_{i=1}^{n}   $.  A 
\textit{clause} is is said to be a disjuction of distinct terms:
$$
t_1 \vee \cdots \vee t_{j_k} = C_k
$$
Then the \textit{satisfiability problem} is the decidability of a conjuction of a set of clauses, 
i.e.:
$$ \wedge_{i=1}^m C_i$$
\end{prob} \cite{skiena2009algorithm}
A \textit{3-SAT problem} is a SAT problem with all clauses having only three boolean variables. 
\begin{definition}[Planar 3-SAT Problem]\label{def:Satisfiability-2}
Given a boolean 3-SAT formula $B$, define the associated graph of $B$ as follows:  
\begin{equation}\label{eqn:sat-1}
G(B) = \left(\set{v_x}{v_x\text{ represents a variable in }B} \cup \set{v_C}{v_C\text{ represent a 
clause in }B}  , \set{\left( v_x, v_C\right) }{x \in C \text{ or } \bar{x} \in C}  \right) 
\end{equation} 
If $G(B)$ in equation (\ref{eqn:sat-1}) is planar, then $B$ is said to be a \textit{Planar 3-SAT 
Problem} \cite{mulzer2008minimum}.
\end{definition}
\subsubsection{Logic Engine}
The logic engine simulates the well known Not All Equal 3 SAT Problem (NAE3SAT).  
\subsubsection{Not All Equal 3 SAT Problem}
\begin{prob}[Satisfiability Problem]\label{prob:Satisfiability-2}%Problem/Question
Give a set of clauses $C$, each containing three boolean variables, can each clause contain at 
least one true variable and one false variable?
\end{prob}
%Universality component
%encoding the engine
\subsubsection{Construction of the Logic Engine}
The components of the logic engine are as follows: the rigid frame, the shaft, the armatures, 
the chains, and the flags.  The \textit{rigid frame} is a rectangular enclosure with a horizontal 
shaft place at mid-height.  The \textit{armatures} are concentric rectangular frames contained 
within the rigid frame.  Each armature can rotate about the shaft; other motions on the armature 
are disallowed.  Given an NAE3SAT, for each variable there is a corresponding armature. On each 
armature, there are chains.  A pair of \textit{chains}, $a_j$ and $\bar{a}_j$ correspond to the 
variable $x_j$ and $\bar{x}_j$ respectively.  The pair is placed on each armature, reflected at a 
height of $h$ above and below the shaft, i.e. one place above the shave at a height of $h$, the 
other placed below the shaft at a height of $-h$.
%insert an armature graphic

\subsubsection{Encoding the Logic Engine}
For each clause of an NAE3SAT, there exists a set of corresponding chains, namely the $h^\text{th}$ 
clause is the set of chains on the armatures at the $h^\text{th}$ row above and below the shaft. A 
chain is \textit{flagged} if the corresponding variable resides within the clause.  The flag can 
point in either the left or right directions indicating a truth assignment for that variable within 
the clause.  A flag is attached to the $i^\text{th}$ chain of every $a_j^\text{th}$ and 
$\bar{a}_j^\text{th}$ chain with the following exceptions:
\begin{enumerate}
 \item if the variable $x_j$  is in clause $C_i$, then link $i$ of $a_j$ is unflagged,
 \item if the variable $\bar{x}_j$ is in clause $C_i$, then link $i$ of $a_j$ is unflagged.
\end{enumerate}
\begin{thm}\label{thm:Satisfiability-1}
 An instance of $NAE3SAT$ is a ``yes'' instance if and only if the corresponding logic engine has a 
flat, collision-free configuration.
\end{thm}
\begin{pf}
 If an instance of $NAE3SAT$ is a ``yes'', then every clause in $C$ contains at least one true 
variable and one false variable.  Now suppose the following truth assignment:
\begin{equation}\label{eqn:Satisfiability-1}
 t\left( x_j \right) = \left\lbrace\begin{array}{cr}
  1 & x_j\text{ and }\bar{x}_j \text{are placed at the top and bottom respectively}\\
  0 & x_j\text{ and }\bar{x}_j \text{are placed at the bottom and top respectively}\\
 \end{array}\right.
\end{equation}
For each clause $c_i \in C$, there exists a variables in $c_i$ such that $t\left( y_i \right) = 1$ 
and $t\left( z_i \right) = 0$.  This implies that there exists an unflagged chain in the 
$i^\text{th}$ and $-i^\text{th}$ row of the frame.  To avoid a collision in each row, trigger the 
flags to point towards the unflagged link. Thus, the corresponding logic engine has a 
flat, collision-free configuration.

If the corresponding logic engine has a flat, collision-free configuration, then there must exist 
an unflagged link in each row.  Without loss of generality, we have that the variables $y_j$ and 
$z_i$ is in clause $C_i$ such that $t\left( y_i \right) = 1$ and $t\left( z_i \right) = 0$ for each 
$i$.  Thus, we have an instance of $NAE3SAT$ is a ``yes'' instance. \cite{BET+99}
\end{pf}


%Here goes something
% \begin{figure}[htbp]
% \begin{center}
% \includegraphics{graphics/RightSwitchBetweenTwoPolygons.pdf}
% \caption{blah blah blah}
% \end{center} 
% \end{figure} 
% \begin{figure}[htbp]
% \begin{center}
% \includegraphics{graphics/LeftSwitchBetweenTwoPolygons.pdf}
% \caption{blah blah blah}
% \end{center} 
% \end{figure} 
\begin{figure}[h]
\begin{center}
  ~ %add desired spacing between images, e. g. ~, \quad, \qquad etc.
    %(or a blank line to force the subfigure onto a new line)
  \begin{subfigure}[b]{0.45\textwidth}
	  \includegraphics[width=\textwidth]{graphics/RightSwitchBetweenTwoPolygons.pdf}
	  \caption{blah blah blah}
	  \label{fig:linkage-1-1}
  \end{subfigure}
  \begin{subfigure}[b]{0.45\textwidth}
	  \includegraphics[width=\textwidth]{graphics/LeftSwitchBetweenTwoPolygons.pdf}
	  \caption{blah blah blah}
	  \label{fig:circlePacking1-2}
  \end{subfigure}
\end{center} 
\caption{Due to the strip in the plane that the hexagon is bounded within the configuration space is limited to just two realizations.}\label{fig:circlePacking-1}
\end{figure}
\subsection{Circle Packing}


Any circle embedded in a plane has a given center point and radius.  This information of planar embedded circle packings allows us to establish the relationship to linkages with the following construction:
\begin{itemize}
\item[\rn{1}] let the centerpoints of the circle packing be a set of vertices $V$;
\item[\rn{2}] if two circles in a circle packing are tangent, we define an edge between their centerpoints.  The distance of this edge is the sum the radii of the two tangent circles.
\end{itemize}  
This construction establishes a relationship between linkages and circle packings.  It begs questioning as to whether every connected simple planar graph has a circle packing.  The question is answered in the following theorem.




A proof of Theorem \ref{thm2-1} is found in chapter 7 of \cite{stephenson2005introduction}.  Theorem \ref{thm2-1} also gives us the ability to establish an equivalent definition of configuration spaces on circle packings and allows us to pose the same realizability problems found with simple planar graphs.  To narrow the focus of the types of circle packing realizabilty problems that we are interested in, we add the following restriction: all circles in a circle packing have unit diameter. 
\subsubsection{Realizability Problems in Unit Disk Packings}
In \cite{Breu19983}, it was shown that unit disk graph recognition is NP-Hard. 
\subsection{Area Packing Problem}


% \begin{definition}[Circle Packing]\label{def:circlePacking}
% $P$ of a planar graph $G$ is a set of of circles with disjoint
% interiors $\left\lbrace C_v \right\rbrace_{v \in G} $ such that two
% circles are tangent if and only if the corresponding vertices form an edge.
% \cite{arXiv13113363v1}
% \end{definition} 
% \begin{thm}[Circle Packing Theorem]\label{thm2-1}
% For every connected simple planar graph $G$ there is a circle packing in the
% plane whose intersection graph is (isomorphic to) $G$.
% \end{thm}
% \begin{figure}[!ht]
% \begin{center}
% \includegraphics[scale=.5]{graphics/circlePackingTheoremExample.pdf}
% \end{center} 
% \caption{This figure is an example of a circle packing for the given simple planar graph.}
% \end{figure} 
% A proof of Theorem \ref{thm2-1} is found in \cite{stephenson2005introduction}.
% 
% \subsubsection{Circle Packings and Polygonal Linkages}
% Given a circle of radius $r$ and its center point, $(x,y)$, we establish the isomorphism  to a hexagon by
% circumscribing the vertices of the regular hexagon.
% \begin{figure}[h]
% \begin{center}
% \includegraphics{graphics/circumscribedHexagon.pdf}
% \caption{A circumbscribed hexagon}
% \end{center}
% \end{figure}
\subsubsection{Hinged Polygons}
\begin{definition}[Polygonal Chain]\label{def}
A polygonal chain $P = \left( v_0, v_1, \dots, v_{n-1}\right) $ is a sequence of
consecutively joined segments (or edges) $e_i = v_i v_{i+1}$ of fixed lengths
$l_i = \left\vert e_i\right\vert $, in a plane. \cite{Biedl99lockedand}
\end{definition}
A chain is said to be closed if $v_{n-1} = v_1$, otherwise it is said to be
open. Hinged polygons have been researched for decades and related to linkage problems
\cite{Biedl99lockedand,canny1988complexity}.

Consider the locked configuration of figure \ref{figure:7hexLocked}.  We can
 configure the hexagons to be locked by placing hinged points as follows:
\begin{figure}[!ht]
\begin{center}
\includegraphics[scale=.33]{graphics/7hexLocked.pdf}
\caption{A locked 7 hexagonal configuration.  (needs to modify picture by
placing red points for hing points.)}
\label{figure:7hexLocked}
\end{center} 
\end{figure}

\subsubsection{Hinged Hexagons of Fixed Size}

\paragraph{The Shapes}
Figure \ref{fig:lockingShape} is a locking shape:
% \begin{figure}[h]
% \begin{center}
% \includegraphics{graphics/lockingShape.pdf}
% \caption{This is the shape that resides in boundary of the lattice.}
% \label{fig:lockingShape}
% \end{center}
% \end{figure}
\begin{figure}[h]
\begin{center}
\includegraphics[scale=.33]{graphics/shapeInChannel.pdf}
\end{center} 
\caption{A locking shape in the lattice boundary's channel.}
\label{fig:lockingShape}
\end{figure}
Figure \ref{fig:lockingShape} shall reside in the boundary of a lattice and have
a hinge point at one vertex where the locking shape and boundary meet.

\paragraph{Junctions}
We define junctions to be the point three hexagons meet in a hexagonal lattice,
e.g. Figure \ref{fig:lattice}.
%Radius of regular polygons 
\newdimen\R
\R=4.5cm
\begin{figure}[h] 
\begin{center}
\begin{tikzpicture}
\begin{scope}
\filldraw[pattern=hexagons]  (0:\R) \foreach \x in {60,120,...,359} {
                -- (\x:\R)
            }-- cycle (90:\R);
\end{scope}
\end{tikzpicture}
\caption{A portion of a hexagonal lattice.}
\label{fig:lattice}
\end{center}
\end{figure}
\newpage
\paragraph{Central Scaling}
\paragraph{Junctions in Conjunctive Normal Form}
Explain the configurations we're interested in.

\nocite{demaine2008geometric,frederickson1997dissections}
\bibliographystyle{plain}
\bibliography{bibliography}
\end{document}