\section{Background}
\subsection{NP and Computational Intractibility}
The objective is explore a space of computationally hard problems.  By comparing the relative difficulty of one
problem versus another problem we can establish statements such as ``Problem X is atleat as hard as Problem Y''.
\begin{definition}[Reduction]\label{def:ncpi-1}
To show that a problem, $X$,  is at least as hard as some other problem, $Y$.
\end{definition} 
\begin{prob}[Reduction Problem]\label{prob:ncpi-1}%Problem/Question
Suppose that problem $X$ can be solved in polynomial time.  Can arbitrary instances of problem $Y$ be solved using
a polynomial number of standard computational steps, plus a polynomial number of calls to a black box that solves
problem X?
\end{prob}
We write $Y \leq_p X$ or state ``$Y$ is polynomial time reducible to $X$'', if the answer to (\ref{prob:ncpi-1}) is
true.
\begin{thm}\label{thm:ncpi-1}
Suppose $Y \leq_p X$.  If $X$ can be solved in polynomial time, then $Y$ can be solved in polynomial time.
\end{thm} 
\subsection{The Independent Set Problem}
\begin{definition}[Independent Set]\label{def:ncpi-2}
Given a graph $G = (V,E)$, we say a set of nodes $S \subset V$ is \textit{independent} if no two nodes in $S$ are
joined by an edge.
\end{definition} 
\begin{definition}[Vertex Cover]\label{def:ncpi-3}
Given a graph $G = (V,E)$, we say a set of nodes $S \subset V$ is a \textit{vertex cover}, if every edge $e \in E$
has at least one end in $S$
\end{definition} 
\begin{prob}[Independent Set Problem]\label{prob:ncpi-2}%Problem/Question
Given a graph $G$ and a number $k$, does $G$ contain an independent set of size at least $k$?
\end{prob} 
\begin{prob}[The Vertex Cover Problem]\label{prob:ncpi-5}%Problem/Question
Given a graph $G$ and a number $k$, does $G$ contain cover of size at most $k$?
\end{prob} 
\subsubsection{Version of the Independent Set Problem}
\begin{prob}[Independent Set Optimization Problem]\label{prob:ncpi-3}%Problem/Question
Find the maximum size of an independent set.
\end{prob} 
\begin{prob}[Independent Set Decision Problem]\label{prob:ncpi-4}%Problem/Question
Decide, yes or no, whether $G$ has an independent set of size at least a given $k$.
\end{prob} 
\begin{thm}\label{thm:ncpi-2}
Let $G=(V,E)$ be a graph.  Then $S$ is an independent set if and only if its complement is a vertex cover.
\end{thm} 
\begin{itemize}
\item[\rn{1}]Suppose $S$ is an independent set of $G = (V,E)$. 
\item[\rn{2}]If $e = (u,v) \in E$, then $u$ and $v$ cannot both be in $S$. 
\item[\rn{3}]Either $u$ or $v$ must be in $V-S$.  Suppose $u \in V-S$.
\end{itemize}  
\begin{itemize}
\item[\rn{1}]Suppose $V-S$ is a vertex cover.
\item[\rn{2}]If $(u,v) = e \in E$ such that $u,v \in S$,  then we have a contradiction with (\rn{1}).
\item[\rn{3}]Thus for any $u,v \in S$, there is no $e = (u,v) \in E$ and thus $S$ must be an independent set.
\end{itemize}  
\begin{thm}[Independent Set is Polynomial-Time Reducible to Vertex Cover]\label{thm}
Independent Set $\leq_p$ Vertex Cover
\end{thm} 

\subsubsection{SAT to 3-SAT Gadget}
\paragraph{SAT Problems}
\begin{prob}[Satisfiability Problem]\label{prob:ncpi-6}%Problem/Question
Let $\left\lbrace x_i \right\rbrace_{i=1}^{n} $ be boolean variables, and $t_i \in \left\lbrace x_i\right\rbrace_{i=1}^{n}  \cup \left\lbrace \bar{x}_i\right\rbrace_{i=1}^{n}   $.  A \textit{clause} is is said to be a disjuction of distinct terms:
$$
t_1 \vee \cdots \vee t_{j_k} = C_k
$$
Then the \textit{satisfiability problem} is the decidability of a conjuction of a set of clauses, i.e.:
$$ \wedge_{i=1}^m C_i$$
\end{prob} 
\paragraph{3-SAT Problems}
A 3-SAT problem is a SAT problem with all clauses having only three boolean variables. 
\paragraph{Converting SAT to 3-SAT}
\begin{itemize}
\item[\rn{1}] Given a clause, $t_1 \vee t_2 \vee t_3 \vee \cdots \vee t_n$.
\item[\rn{2}] Transform the clause given in (\rn{1}) to a conjuction of clauses as the following:
$$\left( t_1 \vee t_2 \vee x_2 \right)\wedge\left( \bar{x}_2 \vee t_3 \vee x_3 \right)\wedge\left( \bar{x}_3 \vee t_4 \vee x_4 \right)\wedge \cdots \wedge \left( \bar{x}_{n-2} \vee t_{n-1} \vee t_n \right)$$
\item[\rn{3}] Clauses with less than three boolean variables and be rewritten as $$\left( t_1 \vee x \vee \bar{x} \right) $$ or $$\left( t_1 \vee t_2 \vee x \right)\wedge \left( \bar{x}  \vee y \vee \bar{y} \right)$$
\end{itemize}  
\subsubsection{3-SAT $\leq_p $ Independent Set}


\subsection{Satifiability Problems}
\subsubsection{3-SAT Problems}
\subsubsection{SAT Reducibility to 3-SAT}
\subsection{Hinged Polygons}
\subsubsection{Hinged Hexagons}
\paragraph{Central Scaling}
\paragraph{The Shapes}
\paragraph{Junctions}
\paragraph{Junctions in Conjunctive Normal Form}
Explain the configurations we're interested in.
\subsubsection{Configurations and Locked Configurations}