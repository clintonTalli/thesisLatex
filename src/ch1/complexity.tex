%I) Algorithm Complexity
%	a) Algorithm
%		procedure of calculations
%		has a running time
%		utilizes resources
%		goal: have an algorithm that runs quickly and utilizes a small amount of resources
%	b) Qualitative Analysis of Algorithms
%		worst case running time
%		brute force
%		efficiency
%	c) Spaces of Algorithms
%		polynomial time
%		PSPACE
%		NP
%		P
%		NP-Complete
\section{Algorithm Complexity}
\textit{Algorithms} are a set of procedural calculations.  When an algorithm executes its procedure it can be measured in terms of units of consumed resources (in computers, that is memory) and the time it takes to complete the procedure of calculations.  Ideally, a desirable algorithm would run quickly and utilizes a small amount of resources.

\subsection{Qualitative Analysis of Algorithms  }
Determining the time and space that algorithms use determine their efficiency.  The \textit{worst-case} running time is the largest possible running time that an algorithm could have over all inputs of a given size $N$.  \textit{Brute force} is when an algorithm tries all possibilities to see if any formulates a solution.  An algorithm is said to be \textit{efficient} if it achieves qualitatively better worst-case performance, at an analytical level, than brute force search.% \cite{kleinberg2006algorithm}.

\subsection{Categorization of Algorithms}
For combinatorial problems, as the number of inputs of the problem grows, the solution space tends to grow exponentially.  In general, as problems grow, it is desirable to minimize the \textit{running time}, time take to run an algorithm that solves a problem. Formally, we quantify running time with Big O notation.
\begin{definition}[Big $O$ Notation]
Let $f$ and $g$ be defined on some subset of $\bbR$.  $f(x) = O\left(g(x)\right)$ if and only if there exists a positive real number $M$ and $x_0$ such that $$\left\vert g(x)\right\vert \leq M \left\vert f(x) \right\vert$$
for all $x \geq x_0$
\end{definition}
 There are various types of running times; the running time that we will focus on in this thesis is polynomial running time ($\P$), nondeterministic polynomial running time ($\NP$), and non-deterministic polynomial complete running time ($\NP$ complete).

An algorithm has a \textit{polynomial running time} if there is a polynomial function $p$ such that for every input string $s$, the algorithm terminantes on $s$ in at most $O\left( p \left( \left\vert s \right\vert\right)\right)$ steps.  Before we continue with the definitions for $\NP$ and $\NP$ complete, we will look into a type of problem, a reduction of a problem, and what an efficient certification is.  This facilitates the reader for the definitions and illustrate complexity better.
\subsubsection{Independent Sets and Vertex Covers}
Given a graph $G = (V,E)$, a set of vertices $S \subset V$ is \textit{independent} if no two vertices in $S$ are joined by an edge. A \textit{vertex cover} of a graph $G = (V,E)$  is a set of vertices $S \subset V$ if every edge $e \in E$, has at least one end corresponding in $S$.

\begin{thm}\label{thm:ch1-complexity-1}
Let $G = (V,E)$ be a graph.  Then $S$ is an independent set if and only if its complement $V-S$ is a vertex cover.
\end{thm}
\begin{pf}
If $S$ is an independent set. Then for any pair of vertices in $S$, the pair are not joined by an edge if and only if for any $v_1, v_2 \in S$, $e = \left( v_1, v_2 \right) \not \in E$.  We have two cases.  The first case is if $v \in S$, then any vertex $u \in V$ that forms an edge $e = (v,u) \in E$ must reside in $V-S$. The second case is if there is an edge which no pair of vertices is in $S$, then both vertices are in $V-S$.  Both cases together imply that every edge has at least one end corresponding in $V-S$. 


If $V-S$ is a vertex cover.  Every edge $e \in E$ has at least one vertex in $V-S$.  The two possible cases, the first case is that the second vertex is in $V-S$, and the second case is that the second vertex is in $S$.  The first case would yield $S = \emptyset$.  The second case implies that the edge $e \in E$ has exactly one vertex in $V-S$ and exactly one vertex in $S$.  $V-S$ is a vertex cover would disallow $S$ to have a pair of vertices to form an edge in the graph.
\end{pf}
Theorem \ref{thm:ch1-complexity-1} allows for problem reductions for independent set and vertex cover problems.
\subsubsection{Reduction of the Independent Set and Vertex Cover Problem}
There are two problems for the independent set: an optimization problem and a decision problem.
\begin{prob}[Optimization of an Independent Set in $G$]
Given a graph $G$, what is the largest independent set in $G$?
\end{prob}
\begin{prob}[Decision of an Independent Set of Size $k$]
Given a graph $G$ and a number $k$, does $G$ contain an independent set of size at least $k$?
\end{prob}
%NEEDS TO PROVE REDUCIBILITY IN THIS PARAGRAPH BELOW
An algorithm that solves the optimization problem automatically solves the decision problem of the independent set.  An algorithm that solves the decision problem for all size $k$ solves the optimization problem where the decision is "yes" for the largest value of $k$.  This establishes a reduction of the optimization problem to the decision problem and vice versa. 
%NEEDS TO PROVE REDUCIBILITY IN THIS PARAGRAPH ABOVE
\subsubsection{Efficient Certification}

To categorize problems \cite{kleinberg2006algorithm}, we ask the following:
\begin{prob}
Can arbitrary instances of problem $Y$ by solved using a polynomial number of standard computational steps, plus a polynomial number of calls to an algorithm that solves $X$?
\end{prob}
The class of problems that can be solved in polynomial running time is called the \textit{polynomial time} class, $\P$.  A second property of problems is whether if its solution can be verified efficiently.  This property is independent of whether it can be solved efficiently.  $B$ is said to be an efficient certifier for a problem $X$  if the following properties hold:
\begin{itemize}
\item[(i)] $B$ is a polynomial-time algorithm that takes two inputs $s$ and $t$.
\item[(ii)] There exists a polynomial function $p$ such that for every string $s$, we have $s \in X$ if and only if there exists a string $t$ such that $\vert t \vert \leq p\left( \vert s \vert \right)$ and $B(s,t) = \text{'yes'}$.
\end{itemize}

The class of problems which have an efficient certifier is said to be the \textit{nondeterministic polynomial time} class, $\NP$.  Other classes of problems are $\NP-\text{complete}$ and $\NP-\text{hard}$.  $\NP-\text{complete}$ is a class of decision problems.  A problem $\mathcal{C}$ is said to be $\NP-\text{complete}$ if $\mathcal{C} \in \NP$ and every problem $\mathcal{D} \in \NP$ is reducible to $\mathcal{C}$ in polynomial time.  $\NP-\text{hard}$ is a class of problems that are at least as hard as $\NP-\text{complete}$ problems.
