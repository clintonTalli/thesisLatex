\section{Algorithm Complexity}
\textit{Algorithms} are a set of procedural calculations.  Determining the time and space that algorithms use determine their efficiency.  The \textit{worst-case} running time is the largest possible running time that an algorithm could have over all inputs of a given size $N$.  \textit{Brute force} is when an algorithm tries all possibilities to see if any formulates a solution.  An algorithm is said to be \textit{efficient} if it achieves qualitatively better worst-case performance, at an analytical level, than brute force search. %cite algorithm design

\subsection{Running Time}
For combinatorial problems, as the number of inputs of the problem grows, the solution space tends to grow exponentially.  In general, as problems grow, it is desirable to minimize the running time for an algorithm.   \textit{Polynomial running time} is if the input size increases from $N$ to $2\cdot N$, the bound on the running time increases from $c \cdot (2N)^d = c \cdot 2^d \cdot N^d$.  
