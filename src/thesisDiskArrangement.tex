\section{Disk Arrangements}

\subsection{Oriented Realizations}
\subsection{Disk Packing Confinement Problem}
Consider the iterative problem:
\begin{enumerate}%1,2,3,4....
\item Start with a circle of diameter 1.
\item Add two kissing circles, each of diameter 1, that do not intersect with any other circle (they may kiss other
circles).
\item For each new kissing circle added, add two more non-intersecting kissing circles to it.
\end{enumerate} 
Figure (\ref{fig:circlePacking-1}) illustrates the iterative problem.  The problem with this is that the area in
which is necessary to contain this disk growing disk arrangement will exceed the area needed to contain it.
\begin{figure}[h]
\begin{center}
    %add desired spacing between images, e. g. ~, \quad, \qquad etc.
    %(or a blank line to force the subfigure onto a new line)
  \begin{subfigure}[b]{0.32\textwidth}
	  \includegraphics[width=\textwidth]{graphics/degree2arrangement.pdf}
	  \caption{A disk arrangement with two layers of disks}
	  \label{fig:circlePacking1-1}
  \end{subfigure}
  \begin{subfigure}[b]{0.32\textwidth}
	  \includegraphics[width=\textwidth]{graphics/degree3arrangement.pdf}
	  \caption{A disk arrangement with three layers of disks}
	  \label{fig:circlePacking1-2}
  \end{subfigure}
  \begin{subfigure}[b]{0.32\textwidth}
	  \includegraphics[width=\textwidth]{graphics/degree4arrangement.pdf}
	  \caption{A disk arrangement with four layers of disks}
	  \label{fig:circlePacking1-3}
  \end{subfigure}
\end{center} 
\caption{The gradual growth of disk arrangements by adding two kissing disks to each of the previously generated disks.  By continuing this arrangement growth, the space needed to contain the kissing disks will exceed the area containing the disk arrangements.}\label{fig:circlePacking-1}
\end{figure}
\begin{tabular}{}
n&
\end{tabular} 