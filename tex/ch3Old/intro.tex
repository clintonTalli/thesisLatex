\chapter{Realizability of Polygonal Linkages with Fixed Orientation\label{chapter:polygonalLinkage}}

We begin the chapter with describing several gadgets that translates the associated graph $A(\Phi)$ of a P3SAT boolean formula.  
These gadgets will be used together to form a special hexagonal tiling that behaves in a similar nature to the logic engine that encoded a NAE3SAT instance of Chapter \ref{chapter:logicEngine} but instead encodes a Planar 3-SAT and its associated graph.
Together the gadgets will form what is called the auxilary construction.
The hexagonal tiling would then be used to prove the following theorem:
\begin{thm}\label{thm:hinge2}
It is strongly NP-hard to decide whether a polygonal linkage whose hinge graph is a \textbf{tree} can be realized with counter-clockwise orientation.
\end{thm}
Our proof is a reduction from P3SAT.
Given an instance $\Phi$ of P3SAT with $n$ variables and $m$ clauses and its associated graph $A(\Phi)$, we construct a simply connected polygonal linkage $(\PP,H)$, of polynomial size in $n$ and $m$, such that $\Phi$ is satisfiable if and only if $(\PP,H)$ admits a realization with fixed orientation. 

We construct a polygonal linkage in two main steps: first, we construct an auxiliary structure where some of the polygons have fixed position in the plane (called \emph{obstacles}), while other polygons are flexible, and each flexible polygon is hinged to an obstacle. 
Second, we modify the auxiliary construction into a polygonal linkage by allowing the obstacles to move freely, and by adding new polygons and hinges as well as an exterior \emph{frame} that holds the obstacle polygons in place.
All polygons in our constructions are regular hexagons or long, skinny rhombi because these are the polygons that we can ``simulate'' with disk arrangements in Chapter \ref{chp:disk}.

Here is a glossary of important formulas for this chapter:
$$
\begin{array}{|rcl|}
\hline
z(n,m)&=& 4 s(n,m)																  \\\hline
J_h (z) &=& 6z(n,m)+1												= 24s(n,m)+1  \\\hline
J_d (z) &=& 4z(n,m)+1												= 16s(n,m)+1  \\\hline
N(n,m)&=& ...														= \\\hline
t(n,m)&=& 2 N^3(n,m)-1												= \\\hline
H(n,m) &=& (12s+1) 2 s\sqrt{3} + 12s \lr{\frac{1}{100N} + \sqrt{3}}	= \\\hline
\end{array}
$$
\paragraph{Modifying the Associated Graph of a P3SAT.}

Given an instance of P3SAT boolean formula $\Phi$ of $n$ variables and $m$ clauses with associated graph $A(\Phi)$, we construct a finite \textit{honeycomb} grid $H_{A \lr{\Phi}}$ of regular hexagons over the plane centered at origin.
We modify the associated graph drawing $A\lr{\Phi}$ by overlaying it onto a honeycomb in the following way:

\begin{enumerate}
%variables represent cycles of 2m 
\item \textbf{Variable:} A vertex representing a variable shall encompass a consecutive set of hexagons along a horizontal line in the honeycomb (see Figure \ref{fig:VariablesExample.pdf}).

\begin{minipage}{\linewidth}
\begin{center}
\includegraphics[width=.75\textwidth]{graphics/VariablesExample.pdf}
\captionof{figure}{The four shaded groups of horizontally adjacent hexagons represent four distinct variables from a boolean formula in the honeycomb.}\label{fig:VariablesExample.pdf}
\end{center}
\end{minipage}

Let $D = \max_{v \in V} \deg(v)$ where $V$ is the set of vertices of $A(\Phi)$.
Every variable vertex $v$  must encompass at least $2 \cdot \deg(v)$ consecutive hexagons but can encompass upto $2 \cdot D$ consecutive hexagons.
\item \textbf{Clause:} A vertex representing a clause shall be a vertex of a hexagon in the honeycomb.
\item \textbf{Edge:} Edges of the associated graph $A(\Phi)$ are paths between the variable $x_i$ and clause $C_j$.  An edge $\left\lbrace x_i, C_j \right\rbrace$ of the associated graph is pariwise edge disjoint. 
The edges of the drawing shall traverse the edges of hexagons in a vertically or horizontally zigzagging manner (see Figure \ref{fig:HoneyCombAssociatedGraphSmall}) in the honeycomb from the literal to the corresponding clause. 
Edges traverse a hexagon in two edges vertically, three edges horizontally.  
The vertical zigzagging edge segments traverse the left or right sides of a hexagon(s).
The horizontal zigzagging edge segments traverse the top or bottom halves of a hexagon(s).
When the edge transisitions from a vertical to horizontal traversal, the edge traverses in over 4 edges about the hexagon.
The length of the edges are bounded above by $6 \cdot \lr{\ell_1 \lr{x_i,C_j} + D}$ where $\ell_1$ is the $L_1$ norm. 
\end{enumerate}
It was shown that if $G=(V,E)$ is planar, it can be embedded in an $\vert V \vert \times \vert V \vert$ grid with $2.4\vert V\vert + 4$ bends \cite{storer1984minimal,tamassia1987efficient}.
From this result, we can define the side length polynomial $s(n,m)$ to account for the hexagons in the grid and let $s(n,m)$ to be something greater than $2.4(n+m) + 4$, e.g.:
$$6 (n+m) + 4.$$


Figure \ref{fig:HoneyCombAssociatedGraphSmall} illustrates an associated graph of a P3SAT overlayed on a honeycomb.
This type of construction emulates an \textit{orthoganal drawning} over a hexagonal grid; an orthoganal drawing where edges are drawn with alternating vertical and horizontal line segments.
% Biedl showed that for graphs maximum degree 3 can be embedded in $\ceil{\frac{n}{2}}\times \ceil{\frac{n}{2}}$ grid \cite{biedl}.
% We can extend this finding to say that our construction over the hexagonal grid can be drawn over a finite region of the plane.
% Each hexagon of $H_{A \lr{\Phi}}$ is of unit side length. 
Let the region in which the construction lies in be a regular hexagon region with polynomial side length $s(n,m)$. 
%withpolynomial $h(n,m)$ such that the region is of size $h(n,m) \times h(n,m)$ \cite{BK+98}.

% \begin{minipage}{\linewidth}
% \begin{center}
% \includegraphics[scale=.5]{graphics/HoneyCombAssociatedGraphSmall.pdf}
% \captionof{figure}{
% This is an instance of an associated graph for a P3SAT overlayed onto a honeycomb grid.  
% This honeycomb graph could correspond to Boolean formula $\lr{\lnot x_1 \lor \lnot x_2 \lor x_4} \land \lr{x_2 \lor \lnot x_3 \lor x_4} \land \lr{x_1 \lor \lnot x_3 \lor \lnot x_4}$. 
% }\label{fig:HoneyCombAssociatedGraphSmall}
% \end{center}
% \end{minipage}

\begin{minipage}{\linewidth}
\begin{center}
\includegraphics[width=.9\textwidth]{graphics/HoneyCombAssociatedGraphSideBySide.pdf}
\captionof{figure}{
(a) This is an instance of an associated graph for a P3SAT overlayed onto a honeycomb grid and placed into a regular hexagonal region of sidelength $s(n,m)$.  
This honeycomb graph could correspond to Boolean formula $\lr{\lnot x_1 \lor \lnot x_2 \lor x_4} \land \lr{x_2 \lor \lnot x_3 \lor x_4} \land \lr{x_1 \lor \lnot x_3 \lor \lnot x_4}$. (b) This is the same instance as (a) shown without the hexagonal region.
}\label{fig:HoneyCombAssociatedGraphSmall}
\end{center}
\end{minipage}

The honeycomb construction will act as preliminary concept that will be refined further in the Auxilary Contruction.
The polynomial $s(n,m)$ will be used to construct the gadgets for the construction.
% We denote the associated graph overlayed on the honeycomb as $\tilde{A}\lr{\Phi}$.