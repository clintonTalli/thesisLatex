\paragraph{Vertical Displacement $\delta$}

When an obstacle hexagon is rotated by $\alpha_i$, the height of the hexagon becomes $2 s(n,m) \sqrt{3} \sec \alpha_i$.
Figure \ref{fig:hexagonNonCanonical.pdf} shows the geometry of a rotated obstacle hexagon.%height of the obstacle hexagon is $2 N(n,m) \sqrt{3}$. 

\begin{minipage}{\linewidth}
\begin{center}
\includegraphics[width=.33\columnwidth]{graphics/hexagonNonCanonical2.pdf}
\captionof{figure}{
% The obstacle hexagon here is in non-canonical position, and showing the side lengths adjacent to $\alpha_i$.
This figure shows a right triangle with angle $\alpha_i$ and sides of length $h$ and $\frac{h}{\cos \alpha_i}$.
}\label{fig:hexagonNonCanonical.pdf}
\end{center}
\end{minipage}

%The height of the cross section of an obstacle polygon and  

%%%%%%%%%%% PEETS COFFEE 2016 06 15 %%%%%%%%%%
% Using Equation \ref{eqn:Hnm} of $H(n,m)$, the length from the canonical position can also be represented as a sum of widths of corridors and cross sectional heights obstacle hexagons in arbtrary position.
% We use this to bound the height of an arbitrary modified auxilary construction:
% \begin{eqnarray*}
% 3 z \lr{\frac{1}{100N} + \sqrt{3}} + 2 N \sqrt{3} \sum_{i=1}^{3z+1} \sec \alpha_i &\leq& 3z \lr{\frac{1}{100N} + \sqrt{3}} + (3z+1) 2 N \sqrt{3}\\
%  2 N \sqrt{3} \sum_{i=1}^{3z+1} \sec \alpha_i &\leq& (3z+1) 2 N \sqrt{3}\\
% \sum_{i=1}^{3z+1} \sec \alpha_i &\leq& (3z+1)\\
% &\leq& \sum_{i=1}^{3z+1} \lr{1 + \frac{a_i^2}{2}}\\
% &=& 3z+1 + \sum_{i=1}^{3z+1}  \frac{a_i^2}{2}
% \end{eqnarray*}
% Note that $\left\vert \sec \alpha_i \right\vert \geq 1$.
%%%%%%%%%%% PEETS COFFEE 2016 06 15 %%%%%%%%%%


%%%%%%%%%% COPA VIDA COMMENT OUT ON SATURDAY 2016 06 11 %%%%%%%%%%
% \begin{eqnarray*}
% H(n,m) = (u+1) \frac{5t-1}{2} \sqrt{3} + u \lr{\frac{1}{100N} + \sqrt{3}}&=& \sum_{i=1}^{u} \sqrt{3} + \sum_{i=1}^{u+1} \frac{5t-1}{2}\sqrt{3} \sec \alpha_i \\ 
% &\leq& u \sqrt{3} + \frac{5t-1}{2} \sqrt{3} \sum_{i = 1}^{u+1} \sec \alpha_i\\
% u \lr{\frac{1}{100N} + \sqrt{3} - \sqrt{3} } &\leq&(t+1) \sqrt{3}    \sum_{i = 1}^{u+1}  \lr{\sec \alpha_i - 1}\\
% \frac{u}{100N(t+1)\sqrt{3}}&\leq& \sum_{i = 1}^{u+1} \lr{1 + \frac{\alpha_i^2}{2} - 1}\\
% \frac{2u}{100N(t+1)\sqrt{3}}&\leq& \sum_{i = 1}^{u+1} \alpha_i^2\\
% \sqrt{\frac{2u}{100N(t+1)\sqrt{3}}} &\leq& \sum_{i=1}^{u+1} \alpha_i\\
% \sqrt{\frac{2u}{100N(t+1)\sqrt{3}}} &\leq& \sum_{i=1}^{u+1} \alpha_i + \alpha_i - \alpha_i\\
% \sum_{i=1}^{u+1} \alpha_i + \alpha_i - \alpha_i &\leq& \sum_{i=1}^{u+1} \alpha_i + \frac{3N^2 - 4 - \frac{4}{(100N)^2}}{N^3}
% \end{eqnarray*}
% Which implies:
% $$\sqrt{\frac{2u}{100N(t+1)\sqrt{3}}} \leq \sum_{i=1}^{u+1} \alpha_i  \leq (u+1) \lr{\frac{3N^2 - 4 - \frac{4}{(100N)^2}}{N^3}}$$
% \begin{eqnarray*}
%%%%%%%%%% COPA VIDA COMMENT OUT ON SATURDAY 2016 06 11 %%%%%%%%%%


% h_\text{max} = (u+1) (t+1) \sqrt{3} + u\lr{\sqrt{3}+ \frac{1}{100N}}&\geq& \sum_{i = 1}^{u+1} (t+1) \cdot \sec \alpha_i \cdot \sqrt{3} + m \sqrt{3}\\
% (m+1) (t+1) \sum_{i = 1}^{m+1} \lr{1 - \sec \alpha_i}&\geq& m \lr{\sqrt{3} - \lr{\sqrt{3}+ \frac{1}{100N}}}\\
% \sum_{i = 1}^{m+1} \lr{ \sec \alpha_i - 1} &\leq& \frac{m}{100 N \sqrt{3} (m+1)(t+1) }\\
% \sum_{i = 1}^{m+1} \lr{ \lr{1 + \frac{\alpha_i^2}{2}} - 1} &\leq&\frac{m}{100 N \sqrt{3} (m+1)\lr{\lr{2N^3 - 1}+1} }\\
% \sum_{i = 1}^{m+1} \frac{\alpha_i^2}{2} &\leq&\frac{m}{200 N^4 (m+1) \sqrt{3} } \\
% \sum_{i = 1}^{m+1} \alpha_i^2 &\leq& \frac{m}{100N^4 (m+1) \sqrt{3}}
% \end{eqnarray*}
% Focusing on the right hand side of the inequality above, we have the following result:
% \begin{eqnarray*}
% \frac{m}{100N^4 (m+1) \sqrt{3}} &<& \frac{m}{m+1}\\
% \frac{1}{100N^4 \sqrt{3}}&<& 1
% \end{eqnarray*}


%As $N \rightarrow \infty$, $(u+1) \lr{\frac{3N^2 - 4 - \frac{4}{(100N)^2}}{N^3}}\rightarrow 0$ which implies $\sum_{i = 1}^{m+1} \alpha_i^2$ is bounded.  
%%%%%%%%%% COPA VIDA COMMENT OUT ON SATURDAY 2016 06 11 %%%%%%%%%%
% The number of obstacle hexagons is determined by a polynomial $N(m,n)$ where $a$ is the number of variables in a corresponding Boolean formula and $b$ is the number of clauses in the Boolean formula.
% Since $\alpha_i$ is bounded above, then there is a maximal rotation of each obstacle hexagon from canonical position.
%Thus every realization of $P$, the obstacle polygons are close to canonical position.
%%%%%%%%%% COPA VIDA COMMENT OUT ON SATURDAY 2016 06 11 %%%%%%%%%%


%%%%%%%%%% COPA VIDA COMMENT OUT ON SATURDAY 2016 06 18 %%%%%%%%%%
% \begin{minipage}{\linewidth}
% \begin{center}
% \includegraphics[width=.33\columnwidth]{graphics/smallHexagonalGridWithTwoLines.pdf}
% \captionof{figure}{}\label{fig:smallHexagonalGridWithTwoLines.pdf}
% \end{center}
% \end{minipage}
%%%%%%%%%% COPA VIDA COMMENT OUT ON SATURDAY 2016 06 18 %%%%%%%%%%

To show that the vertical displacement from canonical position is small, we first consider a column of obstacle hexagons in canonical position (see Figure \ref{fig:verticalDisplacementArgument.pdf}).  
For canonical position, the $\jth$ obstacle has $\delta_j = 0$.

\begin{minipage}{\linewidth}
\begin{center}
\includegraphics[width=.33\columnwidth]{graphics/verticalDisplacementArgument.pdf}
\captionof{figure}{This illustration is of a column of obstacle hexagons in canonical position along a vertical line segment $\ell$.}\label{fig:verticalDisplacementArgument.pdf}
\end{center}
\end{minipage}

%%%%%%%%%% COPA VIDA COMMENT OUT ON SUNDAY 2016 06 19 %%%%%%%%%%
% Suppose we are given an arbitrary realization where $\vert \delta_j \vert > 0$ where $j = 1$, $\cdots$, $u+1$.
% From Equation \ref{eqn:Hnm}, we know the exact height of the $\ell$ in terms of the heights of the corridors and obstacle hexagons in canonical position.
% If $\vert \delta_j \vert > 0$, then we have either $\delta_j > 0$ or $\delta_j <0$.  
% When $\delta_j >0$, the vertical position of $O_j$ is higher than canonical, thus giving less room for the remaining corridors and obstacle hexagons.
% \begin{equation}\label{eqn:vertical3}
% \sum_{i=1}^{j} 2 \sqrt{3} s(n,m) + \sum_{i=1}^{j-1} \lr{\sqrt{3}+\frac{1}{100N}} \leq \sum_{i=1}^{j-1} 2 \sqrt{3} s(n,m) + 2 \sqrt{3} s(n,m) \sec \alpha_j + \delta_j +\sum_{i=1}^{j-1} \lr{\sqrt{3}+\frac{1}{100N}}
% \end{equation}
% The left hand side of Inequality \ref{eqn:vertical3} shows the canonical height of $j$ obstacle hexagons and $j-1$ corridors that reside between the hexagons. 
% The right hand side of Inequality \ref{eqn:vertical3} shows the canonical height of $j-1$ obstacle hexagons, the $j^\text{th}$ obstacle hexagon in noncanonical position with respect to rotation and vertical displacement, and $j-1$ corridors that reside between the hexagons.
% Inequality \ref{eqn:vertical3} is continued in Inequality \ref{eqn:vertical4} where the right hand side is the shows the non-canonical height of $j$ obstacle hexagons with respect to rotation and vertical displacement only and $j-1$ corridors:
% \begin{equation}\label{eqn:vertical4}
%  \sum_{i=1}^{j-1} 2 \sqrt{3} s + 2 \sqrt{3} s \sec \alpha_j + \delta_j + \sum_{i=1}^{j-1} \lr{\sqrt{3}+\frac{1}{100N}}\leq\sum_{i=1}^{j} 2 \sqrt{3} s \sec \alpha_i+ \sum_{i=1}^{j-1} \lr{\sqrt{3}+\frac{1}{100N}} 
% \end{equation}
% From Inequality \ref{eqn:vertical4}, we can derive a bound on $\delta_j$:
% \begin{eqnarray*}
%  \sum_{i=1}^{j-1} 2 \sqrt{3} s + 2 \sqrt{3} s \sec \alpha_j + \delta_j + \sum_{i=1}^{j-1} \lr{\sqrt{3}+\frac{1}{100N}}&\leq&\sum_{i=1}^{j} 2 \sqrt{3} s \sec \alpha_i+ \sum_{i=1}^{j-1} \lr{\sqrt{3}+\frac{1}{100N}} \\
%  \delta_j &\leq& \sum_{i=1}^{j-1} 2 \sqrt{3} s \lr{\sec \alpha_i - 1}\\
%  &\leq&  2 \sqrt{3} s  \sum_{i=1}^{j-1} \lr{1 + \frac{\alpha_i^2 }{2} - 1}\\
%  &=&\sqrt{3} s  \sum_{i=1}^{j-1}  \alpha_i^2
% \end{eqnarray*}
% This bound is fairly large.
% We used the first two terms of  Maclaurin series of secant in the inequalities above.  
% % \begin{equation}\label{eqn:vertical1}
% % \sum_{i=j+1}^u 2 N \sqrt{3}  \sec \alpha_i + \delta_j + (u - j) \sqrt{3} \leq (u-j-1) 2 N \sqrt{3}   + (u-j-1) \lr{\frac{1}{100N} + \sqrt{3}}
% % \end{equation}
% The argument is similar when $\delta_j < 0$, the position of $O_j$ is lower than vertical canonical position and reduces the total amount of room for the previous corridors and hexagons:
% \begin{eqnarray*}\label{eqn:vertical2}
% \delta_j &\geq& -\sqrt{3} s  \sum_{i=1}^{j-1}  \alpha_i^2
% \end{eqnarray*}
% Suppose we are given an arbitrary realization 
%%%%%%%%%% COPA VIDA COMMENT OUT ON SUNDAY 2016 06 19 %%%%%%%%%%

%%%%%%%%%% COPA VIDA COMMENT OUT ON SUNDAY 2016 06 26 %%%%%%%%%%
% From Equation \ref{eqn:Hnm}, we know the exact height of $\ell$ in terms of the heights of the corridors and obstacle hexagons in canonical position.
% Consider the first $j$ terms for the height of the column of obstacle hexagons and corridors for an arbitrary construction with angular rotation and vertical displacement given for each obstacle hexagon where $\vert \delta_j \vert > 0$ where $j = 1$, $\cdots$, $u+1$:
%%%%%%%%%% COPA VIDA COMMENT OUT ON SUNDAY 2016 06 26 %%%%%%%%%%

From Equation \ref{eqn:Hnm}, we know the exact height of $\ell$ in terms of the heights of the corridors and obstacle hexagons in canonical position.  
Consider the first $j$ terms for the height of the column of obstacle hexagons and corridors for an arbitrary construction with angular rotation and vertical displacement for \textit{one} obstacle hexagon $\vert \delta_v \vert > 0$, where $j=2$, $\cdots$, $u+1$ and $1 < v \leq j$.
\begin{eqnarray*}
\sum_{i=1}^j \lr{2 \sqrt{3} s \sec \lr{ \alpha_i}} + \delta_v  + (j-1) \lr{\frac{1}{100N}+\sqrt{3}} &\leq& j \cdot 2 s \sqrt{3} + (j-1) \cdot \lr{\frac{1}{100N}+\sqrt{3}}\\
2 \sqrt{3} s \sum_{i=1}^j \sec \lr{\alpha_i} + \delta_ v &\leq& j \cdot 2 s \sqrt{3}  \\
\sum_{i=1}^j \sec \lr{\alpha_i} + \delta_v &\leq& j\\
 \delta_v &\leq& j- \sum_{i=1}^j \sec \lr{\alpha_i}\\
 \delta_v &\leq& j - \lr{j - \sum_{i=1}^j \frac{\alpha_i^2}{2}}\\
\delta_v &\leq&  \sum_{i=1}^j \frac{\alpha_i^2}{2}
\end{eqnarray*}

%%%%%%%%%% COPA VIDA COMMENT OUT ON SUNDAY 2016 06 26 %%%%%%%%%%
% \begin{eqnarray*}
% \sum_{i=1}^j \lr{2 \sqrt{3} s \sec \lr{ \alpha_i} + \delta_i } + (j-1) \lr{\frac{1}{100N}+\sqrt{3}} &\leq& j \cdot 2 s \sqrt{3} + (j-1) \cdot \lr{\frac{1}{100N}+\sqrt{3}}\\
% 2 \sqrt{3} s \sum_{i=1}^j \sec \lr{\alpha_i} + \delta_i &\leq& j \cdot 2 s \sqrt{3}  \\
% \sum_{i=1}^j \sec \lr{\alpha_i} + \delta_i &\leq& j\\
%  &\leq& j- \sum_{i=1}^j \sec \lr{\alpha_i}\\
%  \sum_{i=1}^j \delta_i &\leq& j- \lr{\sum_{i=1}^j 1 + \frac{\alpha_i^2}{2}}\\
%   \sum_{i=1}^j \delta_i &\leq& - \sum_{i=1}^j \frac{\alpha_i^2}{2}
% \end{eqnarray*}
%%%%%%%%%% COPA VIDA COMMENT OUT ON SUNDAY 2016 06 26 %%%%%%%%%%


%%%%%%%%%% BOURBON CAFE COMMENT OUT ON WEDNESDAY 2016 06 29 %%%%%%%%%%
% For any given modified auxiliary construction, the bottom most obstacle hexagon is either hinged to the bottom of the frame or hinged to a half sized hexagon shown (e.g. Figure \ref{fig:HalfSizeHexagon.pdf}).
% In either case $\alpha_i = 0$.  
% From Inequalities \ref{eqn:angularBound} and \ref{eqn:angularSumBound}, we can conclude that:
% $$
% \begin{array}{rcl}
% \sum_{i=2}^{u-1} \vert \alpha_i - \alpha_{i+1} \vert  &\leq& \frac{5(u-1)}{10N^3 -6}\\
% \end{array}
% $$
%%%%%%%%%% BOURBON CAFE COMMENT OUT ON WEDNESDAY 2016 06 29 %%%%%%%%%%

%%%%%%%%%% UAL COMMENT OUT ON SUNDAY 2016 06 27 %%%%%%%%%%
Using Inequalities \ref{eqn:angularSumBound} and \ref{eqn:angularMaxBound}, we derive the following result:
$$
\begin{array}{rcl}
\sum_{i=1}^j \frac{\alpha_i^2}{2} &\leq& \frac{\sum_{i=1}^j \lr{\sum_{\kappa=1}^i \frac{\kappa}{2\lr{N^3-1} }}^2}{2}\\
\sum_{i=1}^j \alpha_i^2 &\leq& \sum_{i=1}^j \lr{\sum_{\kappa=1}^i \frac{\kappa}{2 \lr{N^3 - 1}}}^2\\
&\leq&\sum_{i=1}^j  \lr{ \frac{1}{2\lr{N^3-1}} \sum_{\kappa=1}^i \kappa}^2\\
&\leq&\sum_{i=1}^j  \lr{ \frac{1}{2\lr{N^3-1}}\cdot \frac{i^2+i}{2}}^2\\
&\leq&\sum_{i=1}^j  \lr{ \frac{i^2+i}{4\lr{N^3-1}}}^2\\
&\leq&\sum_{i=1}^j  \lr{ \frac{i^2 + i}{\lr{\frac{5t-1}{2}}^3 - 1 }}^2\\
&\leq&\sum_{i=1}^j  \lr{ \frac{8\lr{i^2 + i}}{\lr{\lr{5t-1}^3-8 } } }\\
&\leq&\sum_{i=1}^j  \lr{ \frac{8\lr{i^2 + i}}{\lr{\lr{5s^k-1}^3-8 } } }\\
% &\leq&\sum_{i=1}^u \lr{\frac{5 \lr{i^2+i}}{20N^2-12}}^2\\
% &\leq&\sum_{i=1}^u \lr{\sum_{j=1}^u \frac{5j}{10N^3-6}}^2\\	
% &\leq&\sum_{i=1}^u \lr{\frac{5}{10N^3-6}\cdot \sum_{j=1}^u j}^2\\
% &\leq&\sum_{i=1}^u \lr{\frac{5}{10N^3-6}\cdot \frac{u^2+u}{2}}^2\\
% &\leq&\sum_{i=1}^u \lr{\frac{5(u^2+u)}{2\lr{10N^3-6}}}^2\\
% &\leq&u\cdot\lr{\frac{5(u^2+u)}{2\lr{10N^3-6}}}^2
% \sum_{i=1}^j \alpha_{i-1} - \alpha_i &\leq& 
% \sum_{i=2}^{u-1} \vert \alpha_i - \alpha_{i+1} \vert \leq \frac{5(u-1)}{10N^3 -6}\\
% \sum_{i=1}^u \alpha_i &\leq& \frac{5(u-1)}{10N^3 -6}\\
% \sum_{i=1}^u \frac{\alpha_i}{\sqrt{2}} &\leq& \frac{5(u-1)}{\sqrt{2} \lr{10N^3 -6} }\\
% \sum_{i=1}^u \frac{\alpha_i^2}{2} + \sum_{i=1}^u \sum_{\substack{j=1\\j\neq i}}^u \frac{\alpha_i}{\sqrt{2}} \cdot \frac{\alpha_j}{\sqrt{2}} &\leq&\frac{25(u-1)^2}{2\lr{10N^3 - 6}^2}\\
% \sum_{i=1}^u \frac{\alpha_i^2}{2}  &\leq&\frac{25(u-1)^2}{2\lr{10N^3 - 6}^2} -  \sum_{i=1}^u \sum_{\substack{j=1\\j\neq i}^u} \frac{\alpha_i}{\sqrt{2}} \cdot \frac{\alpha_j}{\sqrt{2}}\\
% \sum_{i=1}^u \frac{\alpha_i^2}{2}  &\leq&\frac{25(u-1)^2}{2\lr{10N^3 - 6}^2}
\end{array}$$
%%%%%%%%%% UAL COMMENT OUT ON MONDAY 2016 06 27 %%%%%%%%%%
Thus we finally say that the bound for $\delta_v$, where $1<v\leq j$, is small:
\begin{equation}\label{eqn:verticalBound}
\delta_v \leq \frac{\sum_{i=1}^j  \lr{ \frac{i^2+i}{4\lr{N^3-1}}}^2}{2}
\end{equation}



% \sum_{i=1}^j \frac{\alpha_i^2}{2}\leq \lr\frac{5(u-1)}{10N^3 -6}$$ 
% \begin{equation}\label{eqn:vertical2}
% \sum_{i=1}^{j} (t+1) \sqrt{3} \sec \alpha_i + \delta_i + (u - j-1) \sqrt{3}  \leq (u-j) (t+1) \sqrt{3} + (u-j) \lr{\frac{1}{100N} + \sqrt{3}}
% \end{equation}
% From Inequality \ref{eqn:vertical1}:
% \begin{eqnarray*}
% \delta_i &\leq& t+j-u+3 - \sum_{i=j+1}^{u}\frac{\alpha_i^2}{2}
% \end{eqnarray*}
%%%%%%%%%% COPA VIDA COMMENT OUT ON SATURDAY %%%%%%%%%%
% \begin{eqnarray*}
% \sum_{i=j+1}^u (t+1) \sqrt{3} \sec \alpha_i + \delta_i + (u - j-1) \sqrt{3} &\leq& (u-j-1) (t+1) \sqrt{3} + (u-j-1) \lr{\frac{1}{100N} + \sqrt{3}}\\
% (u-j-1) \lr{\sqrt{3} \lr{1 + \sum_{i=j+1}^{u} \sec \alpha_i + \delta_i}} &\leq& (u-j-1) \lr{ (t+1) \sqrt{3} + \lr{\frac{1}{100N} + \sqrt{3}} }\\
% \lr{\sqrt{3} \lr{1 + \sum_{j+1}^{u} 1 + \frac{\alpha_i}{2} + \delta_i}} &\leq& \lr{ (t+1) \sqrt{3} + \lr{\frac{1}{100N} + \sqrt{3}} }\\
% \sqrt{3}\sum_{i=j+1}^{u}   \delta_i & \leq&  t\sqrt{3} + \sqrt{3}+ \sqrt{3} - \lr{\sqrt{3}\sum_{i=j+1}^{u} 1 + \frac{\alpha_i^2}{2}} \\
% \sum_{i=j+1}^{u}   \delta_i &\leq& \frac{\sqrt{3} \lr{t+2 - \lr{u-j-1} - \sum_{i=j+1}^{u}\frac{\alpha_i^2}{2}}}{\sqrt{3}}\\
% \sum_{i=j+1}^{u}   \delta_i &\leq& t+j-u+3 - \sum_{i=j+1}^{u}\frac{\alpha_i^2}{2}
% \end{eqnarray*}
%%%%%%%%%% COPA VIDA COMMENT OUT ON SATURDAY %%%%%%%%%%