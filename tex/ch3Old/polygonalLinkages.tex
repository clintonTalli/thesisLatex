\chapter{Realizability of Polygonal Linkages with Fixed Orientation\label{chapter:polygonalLinkage}}

We begin the chapter with describing several gadgets that translates the associated graph $A(\Phi)$ of a planar 3SAT instance.  
These gadgets will be used together to form a special hexagonal tiling that behaves in a similar nature to the logic engine of Chapter \ref{chapter:logicEngine} but simulate a Planar 3-SAT and its associated graph.
Together the gadgets will form what is called the auxilary construction.
The hexagonal tiling would then be used to prove the following theorem:
\begin{thm}\label{thm:hinge2}
It is strongly NP-hard to decide whether a polygonal linkage whose hinge graph is a \textbf{tree} can be realized with fixed orientation.
\end{thm}
Our proof is a reduction from P3SAT.
Given an instance $\Phi$ of P3SAT with $n$ variables and $m$ clauses and its associated graph $A(\Phi)$, we construct a simply connected polygonal linkage $(\PP,H)$, of polynomial size in $n$ and $m$, such that $\Phi$ is satisfiable if and only if $(\PP,H)$ admits a realization with fixed orientation. 

We construct a polygonal linkage  in two main steps: first, we construct an auxiliary structure where some of the polygons have fixed position in the plane (called \emph{obstacles}), while other polygons are flexible, and each flexible polygon is hinged to an obstacle. 
Second, we modify the auxiliary construction into a polygonal linkage by allowing the obstacles to move freely, and by adding new polygons and hinges as well as an exterior \emph{frame} that holds the obstacle polygons in place.
All polygons in our constructions are regular hexagons or long and skinny rhombi because these are the polygons that we can ``simulate'' with disk arrangements in Section~\ref{sec:disk}.
\paragraph{Modifying the Associated Graph of a P3SAT.}

Given an instance of P3SAT boolean formula $\Phi$ of $n$ variables and $m$ clauses with associated graph $A(\Phi)$, we construct a finite \textit{honeycomb} grid $H_{A \lr{\Phi}}$ of regular hexagons over the plane centered at origin.
We modify the associated graph drawing $A\lr{\Phi}$ by overlaying it onto a honeycomb in the following way:

\begin{enumerate}
%variables represent cycles of 2m 
\item \textbf{Variable:} A vertex representing a variable shall encompass a consecutive set of hexagons along a horizontal line in the honeycomb. 
Every variable vertex $v$  must encompass at least $2 \cdot \deg(v)$ consecutive hexagons but can encompass upto $2 \cdot D$ consecutive hexagons.
\item \textbf{Clause:} A vertex representing a clause shall be a vertex of a hexagon in the honeycomb.
\item \textbf{Edge:} Edges of the associated graph $A(\Phi)$ are paths from the variable $x_i$ and clause $C_j$.  An edge $\left\lbrace x_i, C_j \right\rbrace$ of the associated graph is pariwise edge disjoint. 
An edge of the drawing shall traverse the edges of hexagons in a vertically or horizontally zigzagging manner in the honeycomb from the literal to the corresponding clause. 
The edges are drawn in a manner that best represents an orthogonal graph drawing over the honeycomb.  
The length of the edges are bounded above by $6 \cdot \lr{\ell_1 \lr{x_i,C_j} + D}$. Edges traverse a hexagon in two edges vertically, three edges horizontally.
  When the edge transisitions from a vertical to horizontal traversal, the edge traverses in over 4 edges about the hexagon.
\end{enumerate}
Figure \ref{fig:HoneyCombAssociatedGraphSmall} illustrates an associated graph of a P3SAT overlayed on a honeycomb.
Each hexagon of $H_{A \lr{\Phi}}$ is of unit side length and has a polynomial number of hexagons, $s(n,m)$.
It is known that the size of the honeycomb is finite and can be determined by polynomials $h(n,m) \times w(n,m)$ \cite{BK+98}.
Let $D = \max_{v \in V} degree(v)$ where $V$ is the set of vertices of $A(\Phi)$.
\begin{minipage}{\linewidth}
\begin{center}
\includegraphics[scale=.5]{graphics/HoneyCombAssociatedGraphSmall.pdf}
\captionof{figure}{
This is an instance of an associated graph for a P3SAT overlayed onto a honeycomb grid.  
This honeycomb graph could correspond to Boolean formula $\lr{\lnot x_1 \lor \lnot x_2 \lor x_4} \land \lr{x_2 \lor \lnot x_3 \lor x_4} \land \lr{x_1 \lor \lnot x_3 \lor \lnot x_4}$.
%This boolean formula for this P3SAT is $\lr{\lnot x_1 \lor \lnot x_2 \lor x_4} \land \lr{x_2 \lor \lnot x_3 \lor x_4} \land \lr{x_1 \lor \lnot x_3 \lor \lnot x_4}$.  The associated bipartite graph is overlaid onto a honeycomb. 
}\label{fig:HoneyCombAssociatedGraphSmall}
\end{center}
\end{minipage}

The honeycomb construction will act as preliminary concept that will be refined further in the Auxilary Contruction.
We denote the associated graph overlayed on the honeycomb as $\tilde{A}\lr{\Phi}$.