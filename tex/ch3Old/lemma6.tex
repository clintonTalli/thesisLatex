The position of each hexagon can be defined by the isometry from its canonical position; an isometry is given by the triple $\lr{\alpha, \beta, \delta}$ where $\alpha$ is a counter clockwise rotation about the center of the hexagon and $\lr{\beta,\delta}$ is a translation vector.
% where $\beta$ is the translation of the obstacle hexagon in the $x$ axis and $\delta$ is the translation of the obstacle hexagon in the $y$ axis.
Canonical position would have each obstacle hexagon's position as $(0,0,0)$.

\begin{lem}\label{lem:aux-C}
Let P be a polygonal linkage obtained from the modified auxilary construction.  
In every realization of $P$, the obstacle polygons are close to canonical position where
$$\vert \lr{\alpha,\beta,\delta}\vert < ???$$
\end{lem}
We need to show that the modified auxilary construction could not deform in such a way that any information the construction encodes is lost or modified and the functionality of the gadgets within the construction behave as stated in the description.
In Figure \ref{fig:tiltedObstaclesInFrame.pdf}, we have a column of obstacle hexagons veering off $\ell$.
% The bottom most obstacle hexagon is glued to the side of the bottom frame hexagon.  
% The subsequent obstacle hexagons are rotated 
%Our goal is show that this quality cannot happen in the modified auxilary construction.  
Lemma \ref{lem:aux-C} serves as assurance that once a boolean formula of P3SAT is encoded into an arbitrary realization of the modified auxilary construction, the information of the boolean formula is preserved regardless of the positioning of the gadgets and components in the construction.
This quality shows that the information is stable and preserved in an arbitrary realization of the modified auxilary construction.

\begin{minipage}{\linewidth}
\begin{center}
\includegraphics[width=.3\columnwidth]{graphics/tiltedObstaclesInFrame.pdf}
\captionof{figure}{This figure depicts a column of obstacle hexagons rotated such that the obstacle hexagons veer of the vertical line $\ell$.}\label{fig:tiltedObstaclesInFrame.pdf}
\end{center}
\end{minipage}

\begin{proof}

\paragraph{Bound for $\alpha$.}
We compare the relative angular rotation of two adjacent obstacle hexagons.
Rotations of two adjacent obstacle hexagons do not differ by more than
$$\left\vert \alpha_i - \alpha_{i+1} \right\vert < \frac{\lr{6N^2 - 8 - \frac{8}{(100N)^2}}\cdot \sqrt{1 + \frac{1}{(100N)^2}} }{3N^3}.$$

\begin{minipage}{\linewidth}
\begin{center}
\includegraphics[width=.9\columnwidth]{graphics/corridorNonCanonical.pdf}
\captionof{figure}{The obstacle hexagon here is in noncanonical position, and showing the side lengths adjacent to $\alpha_i$.}\label{fig:corridorNonCanonical.pdf}
\end{center}
\end{minipage}

In Figure \ref{fig:corridorNonCanonical.pdf}, we show the corridor of shared by two adjacent obstacle hexagons $O_i$ and $O_{i+1}$; $s_i$ represents the extrema position of the side of $O_{i+1}$ where the angle formed between the corridor and $O_{i+1}$ is $\gamma_i$.  
This extrema position is formed when the displacement of the skinny rhombus is vertical.
$\gamma_i$ represents the maximal angular difference between two adjacent obstacle hexagons.  
$\gamma_j$ is the angle between $s_j$ and the horizontal axis at the height of the flag ($j = 1,2,\ldots, m$).
The bound of $\gamma_j$ is:
\begin{equation}\label{eqn:alphaBound}
\begin{array}{rcl}
\gamma_j &\leq & \tan^{-1} \lr{
								\frac{
										\sqrt{1+ \lr{	\frac{1}{100N}	}^2}
								}{
										\frac{N}{2}
								}	
							}\\
&=& \tan^{-1} \lr{\frac{2\sqrt{1 + \frac{1}{(100N)^2}}}{N}}\\
&\leq& \frac{2\sqrt{1 + \frac{1}{(100N)^2}}}{N} - \frac{1}{3}\lr{\frac{2\sqrt{1 + \frac{1}{(100N)^2}}}{N}}^3\\
&=&\frac{\lr{6N^2 - 8 - \frac{8}{(100N)^2}}\cdot \sqrt{1 + \frac{1}{(100N)^2}} }{3N^3}
\end{array} 
\end{equation}
Inequality \ref{eqn:alphaBound} uses the first two terms Maclaurin series of $\tan^{-1}$.
From this inequality, it is clear that as $N \rightarrow \infty$, $\gamma_j \rightarrow 0$.
The sum of maximal relative differences between adjacent obstacle hexagons is
$$\sum_{i=1}^{h(n,m)-1} \left\vert \alpha_i - \alpha_{i+1} \right\vert < \sum_{i=1}^{h(n,m)-1} \gamma_i$$
\paragraph{Vertical Displacement Bound for $\beta$.}
The cross section of the corridor must have a minimum height of $\sqrt{3}$ everywhere.
The height of an obstacle polygon in noncanonical position is $(t+1) \cdot \sec \alpha_i \cdot \sqrt{3}$.
%The height of the cross section of an obstacle polygon and   
\begin{eqnarray*}
h_\text{max} = (m+1) (t+1) \sqrt{3} + m \lr{\sqrt{3}+ \frac{1}{100N}}&\geq& \sum_{i = 1}^{m+1} (t+1) \cdot \sec \alpha_i \cdot \sqrt{3} + m \sqrt{3}\\
(m+1) (t+1) \sum_{i = 1}^{m+1} \lr{1 - \sec \alpha_i}&\geq& m \lr{\sqrt{3} - \lr{\sqrt{3}+ \frac{1}{100N}}}\\
\sum_{i = 1}^{m+1} \lr{ \sec \alpha_i - 1} &\leq& \frac{m}{100 N \sqrt{3} (m+1)(t+1) }\\
\sum_{i = 1}^{m+1} \lr{ \lr{1 + \frac{\alpha_i^2}{2}} - 1} &\leq&\frac{m}{100 N \sqrt{3} (m+1)\lr{\lr{2N^3 - 1}+1} }\\
\sum_{i = 1}^{m+1} \frac{\alpha_i^2}{2} &\leq&\frac{m}{200 N^4 (m+1) \sqrt{3} } \\
\sum_{i = 1}^{m+1} \alpha_i^2 &\leq& \frac{m}{100N^4 (m+1) \sqrt{3}}
\end{eqnarray*}
Focusing on the right hand side of the inequality above, we have the following result:
\begin{eqnarray*}
\frac{m}{100N^4 (m+1) \sqrt{3}} &<& \frac{m}{m+1}\\
\frac{1}{100N^4 \sqrt{3}}&<& 1
\end{eqnarray*}

As $N \rightarrow \infty$, $\frac{m}{100N^4 (m+1) \sqrt{3}} \rightarrow 0$ which implies $\sum_{i = 1}^{m+1} \alpha_i^2$ is bounded.  
The number of obstacle hexagons is determined by a polynomial $N(a,b)$ where $a$ is the number of variables in a corresponding Boolean formula and $b$ is the number of clauses in the Boolean formula.
Since $\alpha_i$ is bounded above, then there is a maximal rotation of each obstacle hexagon from canonical position.
Thus every realization of $P$, the obstacle polygons are close to canonical position.
\paragraph{Horizontal Displacement Bound for $\delta$.}
We show from an induction argument that the the bound for $\delta$ is small.  

\begin{minipage}{\linewidth}
\begin{center}
\includegraphics[width=.5\columnwidth]{graphics/FrameObstacleHexagonCloseUp.pdf}
\captionof{figure}{This shows the close up view of the obstacle hexagons near a frame.}\label{fig:corridorNonCanonical.pdf}
\end{center}
\end{minipage}

\end{proof}