\documentclass{beamer}
\usetheme{Madrid}
\usecolortheme{crane}
\usepackage{animate}
\usepackage{graphicx}
\usepackage{amsmath}
\usepackage{amsfonts}
\usepackage{amssymb}
\usepackage{amsthm}
\usepackage{complexity}
\usepackage{caption}
% \usepackage{subcaption}
\usepackage{enumerate}
\usepackage{enumitem}
\usepackage{array}   % for \newcolumntype macro
% \usepackage{refcheck}
\newcolumntype{L}{>{$}l<{$}} % math-mode version of "l" column type
\newcolumntype{R}{>{$}r<{$}} % math-mode version of "r" column type
\newcolumntype{C}{>{$}c<{$}} % math-mode version of "c" column type
% %%%theoerms, etc%%%%
\newtheorem{thm}{Theorem}
\newtheorem{prob}{Problem}

%%%%%%%%custom commands
\newcommand{\ith}{i^\text{th}}
\newcommand{\jth}{j^\text{th}}
\newcommand{\kth}{k^\text{th}}
\newcommand{\NN}{\mathbb{N}} %  set of natural numbers
\newcommand{\ZZ}{\mathbb{Z}} %  set of integer number
\newcommand{\RR}{\mathbb{R}} %  set of real numbers
\newcommand{\SH}{\mathbb{S}} %  set of unit vectors
\newcommand{\HH}{{\mathcal{H}}} %  Calligraphic H
\renewcommand{\PP}{{\mathcal{P}}} %  Calligraphic P
\newcommand{\DD}{{\mathcal{D}}} %  Calligraphic D
\newcommand{\QQ}{{\mathcal{Q}}} %  Calligraphic D
\newcommand{\FF}{{\mathcal{F}}} %  Calligraphic D
\newcommand{\bbH}{{\mathbb{H}}}
\newcommand{\bbR}{{\mathbb{R}}}
\newcommand{\bbP}{{\mathbb{P}}}
\newcommand{\bbZ}{{\mathbb{Z}}}
\newcommand{\bbC}{{\mathbb{C}}}
\newcommand{\bbQ}{{\mathbb{Q}}}
\newcommand{\bbA}{{\mathbb{A}}}
\newcommand{\bbF}{{\mathbb{F}}}
\newcommand{\bbh}{{\mathbb{H}}}
\newcommand{\bbr}{{\mathbb{R}}}
\newcommand{\bbp}{{\mathbb{P}}}
\newcommand{\bbz}{{\mathbb{Z}}}
\newcommand{\bbc}{{\mathbb{C}}}
\newcommand{\bbq}{{\mathbb{Q}}}
\newcommand{\bba}{{\mathbb{A}}}
\newcommand{\bbf}{{\mathbb{F}}}
\newcommand{\bbn}{{\mathbb{N}}}
\newcommand{\bbN}{{\mathbb{N}}}

\title[Planar Configuration Spaces]{Planar Configuration Spaces }
\subtitle{of Disk Arrangements and Hinged Polygons}
\author{Clinton Bowen}
\institute
{
  Cal State Northridge
}
\date
{December 6, 2016}
\subject{Mathematics}
\begin{document}
\frame{\titlepage}

\begin{frame}\frametitle{Motivation: Hinged Dissection}
    \begin{columns}[c]
    \column{.5\textwidth}
        \begin{itemize}
            \item[*] \textit{Dudeney Problem}: Can a square and an equilateral triangle of the same area have a common dissection into four pieces?
            \item[*]  Can finite collection of polygons of equal area has a common hinged dissection?
        \end{itemize}
    \column{.5\textwidth}
        \begin{minipage}{\linewidth}
            \begin{center}
            \includegraphics[width=.9\textwidth]{graphics/hingeOnThreeDistinctPolygons.pdf}
            \captionof{figure}{A collection of polygons with hinges.}\label{gfx:hingeOnThreeDistinctPolygons.pdf}
            \end{center}
        \end{minipage}
    \end{columns}
\end{frame}

\begin{frame}\frametitle{Motivation: Hinged Dissection}
    \begin{columns}[c]
    \column{.5\textwidth}
        \begin{itemize}
            \item[*] \textit{Dudeney Problem}: Can a square and an equilateral triangle of the same area have a common dissection into four pieces?
            \item[*]  Can finite collection of polygons of equal area has a common hinged dissection?
        \end{itemize}
  \column{.5\textwidth}
  \begin{minipage}{\linewidth}
    \begin{center}
    \animategraphics[loop,controls,width=\linewidth]{12}{graphics/hingedanimation-}{0}{107}
    \captionof{figure}{blah}
    \end{center}
  \end{minipage}
  \end{columns}
\end{frame}

\begin{frame}
% Protein folding is the physical process by which a protein chain acquires its native 3-dimensional structure, a conformation that is usually biologically functional, in an expeditious and reproducible manner.
  \frametitle{Protein Folding}
  Protein folding is the process in which a protein chain acquires its 3-dimensional structure.
  \begin{columns}[c] % the "c" option specifies center vertical alignment
  \column{.5\textwidth} % column designated by a command
   \begin{itemize}
    \item[*] Proteins in an organism fold into a specific geometric pattern (sometimes referred as its \textit{native state}).
    \item[*]  Geometric patterns can determine a protein's function and behavior.
   \end{itemize}
  \column{.5\textwidth}
     \begin{minipage}{\linewidth}
    \begin{center}
    \includegraphics[width=.66\columnwidth]{graphics/5FH3.png}
    \captionof{figure}{
The structure of rat cytosolic PEPCK variant E89A in complex with oxalic acid and GTP \cite{Rats}.}
    \end{center}
  \end{minipage}
  \end{columns}
\end{frame}

\begin{frame} \frametitle{Logic Engine Realized as Hinged Polygons.pdf}
    \begin{columns}[c]
    \column{.5\textwidth}
        \begin{itemize}
            \item[*] Suppose we are given an Boolean formula with $m$ clauses and $n$ variables in 3-CNF form, $\Phi$, we construct the polygonal linkage similarly to the logic engine.
        \end{itemize}
    \column{.5\textwidth}
        \begin{minipage}{\linewidth}
            \begin{center}
            \includegraphics[width=.9\textwidth]{graphics/HingedLogicEngineSmallEnumerated.pdf}
            \captionof{figure}{Caption Text}\label{gfx:HingedLogicEngineSmallEnumerated.pdf}
            \end{center}
        \end{minipage}
    \end{columns}
\end{frame}

\begin{frame} \frametitle{Logic Engine Realized as Hinged Polygons.pdf}
    \begin{columns}[c]
    \column{.5\textwidth}
        \begin{itemize}
            \item[*] Breu and Kirkpatrick~\cite{BK98} proved that it is NP-hard to decide whether a graph $G$ is the contact graph of unit disks in the plane, i.e., recognizing \emph{coin graphs} is NP-hard; see also~\cite{BET+99}.
        \end{itemize}
    \column{.5\textwidth}
    % \begin{figure}[!htpb]
    % \begin{center}
    %   \begin{subfigure}[b]{0.24\textwidth}
    %       \includegraphics[width=\textwidth]{graphics/degree2arrangement.pdf}
    %       \caption{A disk arrangement with two layers of disks}
    %       \label{fig:circlePacking2-1}
    %   \end{subfigure}
    %   \begin{subfigure}[b]{0.24\textwidth}
    %       \includegraphics[width=\textwidth]{graphics/degree3arrangement.pdf}
    %       \caption{A disk arrangement with three layers of disks}
    %       \label{fig:circlePacking2-2}
    %   \end{subfigure}
    %   \begin{subfigure}[b]{0.24\textwidth}
    %       \includegraphics[width=\textwidth]{graphics/degree4arrangement.pdf}
    %       \caption{A disk arrangement with four layers of disks}
    %       \label{fig:circlePacking2-3}
    %   \end{subfigure}
    %   \begin{subfigure}[b]{0.24\textwidth}
    %       \includegraphics[width=\textwidth]{graphics/degree5arrangement.pdf}
    %       \caption{A disk arrangement with five layers of disks}
    %       \label{fig:circlePacking2-4}
    %   \end{subfigure}
    % \end{center}
    %     \caption{For $i=2,3,4,5$ the tree $T_i$ is a contact graph of unit disks.}\label{fig:circlePacking-2}
    % \end{figure}
    \end{columns}
\end{frame}

\begin{frame} \frametitle{Planar 3SAT}
    \begin{columns}[c]
    \column{.5\textwidth}
        \begin{itemize}
            \item[*] Define the \textit{associated graph} $A(\Phi)$ as follows: the vertices correspond to the variables and clauses in $\Phi$.   
We place an edge in the graph if variable $x_i$ appears in clause $C_j$.
            \item[*] Given a Boolean formula $\Phi$ in 3-CNF such that its associated graph is planar, decide whether it 
is satisfiable is a \textit{3-SAT problem}.
        \end{itemize}
    \column{.5\textwidth}
        \begin{minipage}{\linewidth}
            \begin{center}
            \includegraphics[width=.9\textwidth]{graphics/fig-assoc-hex.pdf}
            \captionof{figure}{Caption Text}\label{gfx:fig-assoc-hex.pdf}
            \end{center}
        \end{minipage}
    \end{columns}
\end{frame}

\begin{frame} \frametitle{honeycomb.pdf}
    \begin{columns}[c]
    \column{.5\textwidth}
        \begin{itemize}
            \item[*] Define the \textit{associated graph} $A(\Phi)$ as follows: the vertices correspond to the variables and clauses in $\Phi$.   
We place an edge in the graph if variable $x_i$ appears in clause $C_j$.
            \item[*] Given a Boolean formula $\Phi$ in 3-CNF such that its associated graph is planar, decide whether it 
is satisfiable is a \textit{3-SAT problem}.
        \end{itemize}
    \column{.5\textwidth}
        \begin{minipage}{\linewidth}
            \begin{center}
            \includegraphics[width=.9\textwidth]{graphics/honeycomb.pdf}
            \captionof{figure}{Caption Text}\label{gfx:honeycomb.pdf}
            \end{center}
        \end{minipage}
    \end{columns}
\end{frame}

\begin{frame} \frametitle{Transmitter Gadget}
    \begin{columns}[c]
    \column{.5\textwidth}
        \begin{itemize}
            \item[*] A {\it transmitter gadget} is constructed for each edge $\left\lbrace x_i,C_j\right\rbrace$ of the graph $A(\Phi)$; it consists of a sequence of junctions and corridors from a variable gadget's junction to a clause junction.  
        \end{itemize}
    \column{.5\textwidth}
        \begin{minipage}{\linewidth}
            \begin{center}
            \includegraphics[width=.9\textwidth]{graphics/FalseVariableNonNegatedLiteralTransmitter.pdf}
            \captionof{figure}{Caption Text}\label{gfx:FalseVariableNonNegatedLiteralTransmitter.pdf}
            \includegraphics[width=.9\textwidth]{graphics/VariableGadgetTruthness.pdf}
            \captionof{figure}{Caption Text}\label{gfx:VariableGadgetTruthness.pdf}
            \end{center}
        \end{minipage}
    \end{columns}
\end{frame}

% \begin{frame} \frametitle{VariableGadgetTruthness.pdf}
%     \begin{columns}[c]
%     \column{.5\textwidth}
%         \begin{itemize}
%             \item[*] item 1
%             \item[*] item 2
%         \end{itemize}
%     \column{.5\textwidth}
%         \begin{minipage}{\linewidth}
%             \begin{center}
%             \includegraphics[width=.9\textwidth]{graphics/VariableGadgetTruthness.pdf}
%             \captionof{figure}{Caption Text}\label{gfx:VariableGadgetTruthness.pdf}
%             \end{center}
%         \end{minipage}
%     \end{columns}
% \end{frame}

\begin{frame} \frametitle{Variable Gadget}
    \begin{columns}[c]
    \column{.5\textwidth}
        \begin{itemize}
            \item[*] Variable $x_i$ corresponds to a cycle in the associated graph $\tilde{A}(\Phi)$, which has been embedded as a cycle in the hexagonal tiling, with corridors and junctions. 
In each junction along this cycle, attach a small hexagon in the common boundary of the two corridors in the cycle. 
        \end{itemize}
    \column{.5\textwidth}
                    \begin{minipage}{\linewidth}
            \begin{center}
            \includegraphics[width=.9\textwidth]{graphics/VariableGadgetSmall.pdf}
            \captionof{figure}{Caption Text}\label{gfx:VariableGadgetSmall.pdf}
            \includegraphics[width=.9\textwidth]{graphics/fig-variable-hex+.pdf}
            \captionof{figure}{Caption Text}\label{gfx:fig-variable-hex+.pdf}
            \end{center}
        \end{minipage}
    \end{columns}
\end{frame}

\begin{frame} \frametitle{Clause Junction Gadget}
    \begin{columns}[c]
    \column{.5\textwidth}
        \begin{itemize}
            \item[*] The {\it clause gadget} lies at a junction adjacent to three transmitter gadgets (see Fig.~\ref{fig:clause} and Section \ref{transmitterGadget}).  At such a junction, we attach a unit line segment to an arbitrary vertex of the junction, and a small hexagon of side length $\frac{1}{3}$ to the other end of the segment. 
        \end{itemize}
    \column{.5\textwidth}
        \begin{minipage}{\linewidth}
            \begin{center}
            \includegraphics[width=.9\textwidth]{graphics/fig-clause-hex.pdf}
            \captionof{figure}{Caption Text}\label{gfx:fig-clause-hex.pdf}
            \end{center}
        \end{minipage}
    \end{columns}
\end{frame}

\begin{frame}\frametitle{Motivation: Weighted Trees and Disk Arrangements}
    \begin{columns}[c]
    \column{.5\textwidth}
        \begin{itemize}
            \item[*] Is it strongly NP-hard to decide whether a polygonal linkage whose hinge graph is a \textit{tree} can be realized? 
            \item[*] Is it NP-Hard to decide whether a given ordered tree with positive vertex weights is the contact graph of a disk arrangements with specified radii?
        \end{itemize}
    \column{.5\textwidth}
        \begin{minipage}{\linewidth}
            \begin{center}
            \includegraphics[width=.9\textwidth]{graphics/fig2+.pdf}
            \captionof{figure}{Caption Text}\label{gfx:fig1+.pdf}
            \end{center}
        \end{minipage}
    \end{columns}
\end{frame}

\begin{frame} \frametitle{Modified Auxiliary Gadget}
    \begin{columns}[c]
    \column{.5\textwidth}
        \begin{itemize}
            \item[*] The modified auxiliary gadget channels and junctions in a hexagonal grid enclosed by six frame hexagons.
            \item[*] A comment about NAE3SAT....
        \end{itemize}
    \column{.5\textwidth}
        \begin{minipage}{\linewidth}
            \begin{center}
            \includegraphics[width=.9\textwidth]{graphics/fig-frame-hex.pdf}
            \captionof{figure}{Caption Text}\label{gfx:fig-frame-hex.pdf}
            \end{center}
        \end{minipage}
    \end{columns}
\end{frame}

\begin{frame} \frametitle{hexagonPetiolesLeafs9Layers.pdf}
    \begin{columns}[c]
    \column{.5\textwidth}
        \begin{itemize}
            \item[*] item 1
            \item[*] item 2
        \end{itemize}
    \column{.5\textwidth}
        \begin{minipage}{\linewidth}
            \begin{center}
            \includegraphics[width=.9\textwidth]{graphics/hexagonPetiolesLeafs9Layers.pdf}
            \captionof{figure}{Caption Text}\label{gfx:hexagonPetiolesLeafs9Layers.pdf}
            \end{center}
        \end{minipage}
    \end{columns}
\end{frame}

\begin{frame} \frametitle{honeycomb.pdf}
    \begin{columns}[c]
    \column{.5\textwidth}
        \begin{itemize}
            \item[*] item 1
            \item[*] item 2
        \end{itemize}
    \column{.5\textwidth}
        \begin{minipage}{\linewidth}
            \begin{center}
            \includegraphics[width=.9\textwidth]{graphics/honeycomb.pdf}
            \captionof{figure}{Caption Text}\label{gfx:honeycomb.pdf}
            \end{center}
        \end{minipage}
    \end{columns}
\end{frame}

\begin{frame} \frametitle{HoneyCombAssociatedGraphSmall.pdf}
    \begin{columns}[c]
    \column{.5\textwidth}
        \begin{itemize}
            \item[*] item 1
            \item[*] item 2
        \end{itemize}
    \column{.5\textwidth}
        \begin{minipage}{\linewidth}
            \begin{center}
            \includegraphics[width=.9\textwidth]{graphics/HoneyCombAssociatedGraphSmall.pdf}
            \captionof{figure}{Caption Text}\label{gfx:HoneyCombAssociatedGraphSmall.pdf}
            \end{center}
        \end{minipage}
    \end{columns}
\end{frame}
 \begin{frame}\frametitle{Problem}
    It is strongly NP-hard to decide whether a polygonal linkage whose inge graph is a \textit{tree} can be realized.

It is strongly NP-hard to decide whether a polygonal linkage whose inge graph is a \textit{tree} can be realized with fixed orientation.

It is NP-Hard to decide whether a given ordered tree with positive vertex weights is the contact graph of a disk arrangements with pecified radii.
  \end{frame}
\end{document}
