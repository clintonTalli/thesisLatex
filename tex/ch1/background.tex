\chapter{Background}
In this section, we cover the background subjects needed to formally pose the 
problems and present solutions in this thesis.  We start with two types of combinatorial structures, linkages 
and polygonal linkages.  We then discuss the configuration spaces of linkages and polygonal 
linkages.   We then look into an alternate representation of linkages, disk arrangements and state 
the disk arrangement theorem.  Next we look at satisfiability problems and then review a framework, the 
logic engine, which can encode a type of satisfiability problem.  Finally, we cover the basic 
definitions of algorithm complexity for $\textbf{P}$ and $\textbf{NP}$.

Decidability problems study whether there exists a way to determine whether an element is a member of a set.  In this paper we focus on four such decidability problems surrounding graph theory and geometry. The first set of problems involve a special type of graph called a tree and the second set of problems involve something called a polygonal linkage.  In each problem, set membership is determined if the tree or polygonal linkage has a particular property when visualized in the plane. 

This thesis first presents the preliminary information needed to pose our four problems, then we formally pose each problem and then provide solutions on decidability for each problem. 