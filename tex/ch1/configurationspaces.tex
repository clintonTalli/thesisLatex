\section{Configuration Spaces}
\begin{quote}
Just as one can compose colors or forms, so one can compose motions.
\end{quote}
{\raggedright{}Alexander Calder, 1933}

Recall Figure \ref{fig:HingedHaberdasher} illustrating the hinged dissection that formed a square and triangle and several drawings of the hinged dissections that simulate the motion of moving the polygons around the hinge points to form each shape.  The set of all drawings in that motion represents the \textit{configuration space} for that polygonal linkage.  In this section we will formally describe the configuration space for each object we've drawn thus far.


% We'd like to describe motions and range of motions of embedded graphs, linkages, polygonal 
% linkages, and disk arrangements.  Table \ref{table:configurationSpace-1} provides the definition of 
% \textit{reconfiguration} for each type of object covered so far:
% \begin{center}
% \begin{table}[htbp!]
% \begin{tabular}{|p{.2\textwidth}|p{.79\textwidth}|}
% \hline
% Object Type&Definition of Reconfiguration\\\hline
% Graph Embeddings&a continuous motion of the vertices that never causes the edges to 
% intersect.\\\hline
% Linkage&a continuous motion of the vertices that preserves the lengths of the edges and never 
% causes 
% the edges to intersect.\\\hline
% Polygonal Linkage&a continuous motion of polygons that preserves shapes of polygons, hinge point 
% pairings, and never causes the polygonal sides to intersect.\\\hline
% Disk Arrangement&a continuous motion of disks that preserves disk radii, pairs of contact points, 
% and never causes disks to intersect.\\\hline
% \end{tabular}\label{table:configurationSpace-1}
% \end{table}
% \end{center}




% For graphs, a \textit{reconfiguration} is a continuous motion of 
% the vertices that preserve the length edges and never cause edges to collide \cite{AKR+04}.  For 
% polygonal linkages, a reconfiguration is 

\subsection{Configuration Spaces of Graph Drawings}
Recall that for a graph drawing we have an injective mapping $\Pi : V \mapsto \bbR^{2}$ which maps vertices to distinct points in the plane and for each edge $\curlybraces{u,v} \in E$, a straight line segment, $c_{u,v}:[0,1]\mapsto \bbR^2$ such that $c_{u,v}(0) = \Pi(u)$ and $c_{u,v}(1) = \Pi(v)$, and does not pass through other vertices.
For each vertex of $G$, the embedding of the vertex lies in the plane, i.e. $\Pi(v) \in \bbR^2$.  
By enumerating each vertex of $G$, e.g. $v_1, v_2, \dots, v_k, \dots, v_{n}$, we can create a projection mapping from $\mu: \Pi \mapsto \bbR^{2\vert V \vert}$ where the corresponding coordinates of $\Pi(v_k)$ are in the $(2k)^\text{th}$ and $(2k+1)^{th}$ coordinates in $\bbR^{2\vert V \vert}$.  
$\mu(\Pi)$ is a configuration.
The configuration space is the set of $\mu(\Pi)$ for all drawings $\Pi$.  

\subsection{Configuration Spaces of Linkages}

Consider drawings of a graph that respects the length assignment.  
A \textit{realization} of a linkage, $(G,\ell)$, is a drawing of a graph, $\Pi$, such that for every edge $\{u,v\} \in E$, $\ell\left( \{u,v\} \right) = \left\vert \Pi(u) - \Pi(v) \right\vert = \left\vert \Pi(v) - \Pi(u) \right\vert$. 
A \textit{plane realization} is a plane drawing with the property, $\ell\left( \{u,v\} \right) = \left\vert \Pi(u) - \Pi(v) \right\vert$.
First let's define the space of realizations for a corresponding linkage, i.e.:
$$P_{(G,\ell)} = \set{\Pi_{(G,\ell)}}{\forall \{u,v\} \in E\text{, }\ell\left( \{u,v\} \right) = \left\vert \Pi(u) - \Pi(v) \right\vert}$$
With respect to $P$, we can establish a \textit{configuration space} that allows one to study problems of motion.  For each vertex of $G$, the drawing of the vertex lies in the plane, i.e. $\Pi(v) \in \bbR^2$.  
By enumerating each vertex of $G$, e.g. $v_1, v_2, \dots, v_k, \dots, v_{n}$, we can create a projection mapping from $\mu: P \mapsto \bbR^{2\vert V \vert}$ where the corresponding coordinates of $\Pi(v_k)$ are in the $(2k)^\text{th}$ and $(2k+1)^{th}$ coordinates in $\bbR^{2\vert V \vert}$.  
The configuration space is $\mu(P)$.  

Using standard definitions from real analysis, we can begin to pose problems about linkages with respect to a corresponding configuration space.  
We define a path $\gamma: [0,1]\mapsto \mu(P)$ where $\gamma(0)$ corresponds the the projection of a realization of a linkage $\Pi_0$ and $\gamma(1)$ corresponds to another realization of a linkage $\Pi_1$.  
If for any two elements $a,b \in \mu(P)$ that there exists a continuous path $\gamma$ such that $\gamma(0)=a$ and $\gamma(1)=b$, $\mu(P)$ is said to be path connected.   
For $\gamma$ to be continuous we would have that for every $\epsilon > 0$, there exists a $\delta >0$ such that if $x,y \in [0,1]$ and $\vert x-y \vert <\delta$ then $\vlr{\vlr{\gamma(x)-\gamma(y)}}<\epsilon$.
$\gamma$ can be thought of as an animation of drawings that starts at $\gamma(0)$ and ends at $\gamma(1)$.
To ask if $\mu(P)$ is a connected space, is to ask if $\mu(P)$ is connected in $\bbR^{2\vert V \vert}$.
The Carpenter's Rule states that every realization of a path linkage can be continuously moved (without self-intersection) to any other realization \cite{CDR03,Str05}.
In other words, the realization space of such a linkage is always connected.

% \subsection{Configuration Spaces of Linkages}

% \textbf{NOTE THAT THIS SUBSECTION MAY HAVE REPEATED CONTENT}

% Let's focus on the space of embeddings of a linkage. If there are $n$ vertices of a linkage, 
% the \textit{configuration space} of a linkage is said to be a vector space of dimension $2 \cdot n$ 
% where edge length is preserved.  

% A \textit{configuration space} for a linkage $G$ and corresponding proper embedding, $L_1$ is said 
% to be for any other proper embedding of a linkage $G$, $L_2$, such that the lengths 
% of every edge of $G$ is preserved between the two embeddings, i.e.: 
% $$l\left( \left(u,v\right) 
% \right) = \left\vert 
% L_1(u) - L_1(v) \right\vert = \left\vert L_2(u) - L_2(v) \right\vert$$
% Equivalent embeddings include translations and rotations about the center of mass on $L(V)$.  We 
% further our embeddings by requiring that one vertice is pinned to the point of origin on the plane 
% as well as a neighboring vertex.


% Can a simple planar polygon be moved continuously to a position where all its vertices are in convex position, so that the edge lengths and simplicity are preserved along the way?

\begin{figure}[!htbp]%blah
\begin{center}
\includegraphics{graphics/twoEmbeddingsOfSameLinkage.pdf}
\end{center} 
\caption{(a) and (b) show a linkage in two embeddings.  Any realization of a path can be continuously moved without self-intersection to any other realizations.}
\label{fig:configuration-3}
\end{figure}

\subsection{Configuration Spaces of Polygonal Linkages}
Recall a realization of a polygonal linkage is an interior-disjoint placement of congruent copies of the polygons in $\PP$ such that the copies of a hinge are mapped to the same point (e.g., Figure \ref{fig:linkage-1}).
First consider the set of all realizations for the polygonal linkage $\left(\PP,\HH\right)$ and call it $P$.  
For any realization $R \in P$, the parameterization $\mu:R \mapsto \bbR^{2m}$ where $m$ is the number of distinct vertices in $\PP$.  The configuration space is the set $\mu(P)$.

 \subsection{Configuration Spaces of Disk Arrangements}
%show an equivalent disk arrangment to the geometric dissection in thier corresponding/equivalent
% configurations.

\begin{figure}[!htbp]
\begin{center}
\includegraphics[scale=.75]{graphics/DiskPackingReconfiguration.pdf}
\end{center} 
\caption{An example of a disk arrangement where $A$ and $B$ have a large range of freedom to move around.  $C$, $D$, $E$, and $F$ are limited in their range of motion to due to their hinge points.}
\label{fig:configuration-5}
\end{figure}

Consider the set of realizations $P$ for a given disk arrangement $\DD = \left\lbrace D_i \right\rbrace_{i=1}^n$.  For any realization $R \in P$, there exists a corresponding contact graph, $C$.  The configuration spaces of $\DD$ are sets of $R \in P$ that are classified by the equivalent contact graphs, i.e. if $R_1$, $R_2 \in P$ and their corresponding contact graphs $C_1$ and $C_2$ have a graph isomorphism, $\phi$, then $R_1$ and $R_2$ belong to the same configuration space.





% \subsection{Reconfiguration}



% We'd like to describe motions and range of motions of embedded graphs, linkages, polygonal 
% linkages, and disk arrangements.  Table \ref{table:configurationSpace-1} provides the definition of 
% \textit{reconfiguration} for each type of object covered so far:
% \begin{center}
% \begin{table}[htbp!]\label{table:configurationSpace-1}
% \begin{tabular}{|p{.2\textwidth}|p{.79\textwidth}|}
% \hline
% Object Type&Definition of Reconfiguration\\\hline
% Graph Embeddings&a continuous motion of the realized vertices that never causes the edges to 
% intersect.\\\hline
% Linkage&a continuous motion of the realized vertices that preserves the lengths of the edges and never 
% causes 
% the edges to intersect.\\\hline
% Polygonal Linkage&a continuous motion of polygons that preserves shapes of polygons, hinge point 
% pairings, and never causes the polygonal sides to intersect.\\\hline
% Disk Arrangement&a continuous motion of disks that preserves disk radii, pairs of contact points, 
% and never causes disks to intersect.\\\hline
% \end{tabular}
% \end{table}
% \end{center}








% 
% \paragraph{Confining Linkages to a Restricted Space Within a Configuration Space}
% So we've covered the idea of linkages within a plane; now let's constrain the plane to a strip and 
% have a linkage that is a \textit{polygon}, i.e. a linkage that forms a closed chain (e.g. Table 
% \ref{table:linkage-1}), hugging the boundaries of the strip:
% \begin{figure}[h]
% \begin{center}
%   ~ %add desired spacing between images, e. g. ~, \quad, \qquad etc.
%     %(or a blank line to force the subfigure onto a new line)
%   \begin{subfigure}[b]{0.49\textwidth}
% 	  \includegraphics[width=\textwidth]{graphics/hexagonInChannelWithPinnedJointRight.pdf}
% 	  \caption{A bounded hexagon that resides in a channel with a pinned vertex}
% 	  \label{fig:linkage-1-1}
%   \end{subfigure}
%   \begin{subfigure}[b]{0.49\textwidth}
% 	  \includegraphics[width=\textwidth]{graphics/hexagonInChannelWithPinnedJointLeft.pdf}
% 	  \caption{The second realization of the hexagon residing in a channel with a pinned 
% vertex.}
% 	  \label{fig:linkage-1-2}
%   \end{subfigure}
% \end{center} 
% \caption{Due to the strip in the plane that the hexagon is bounded within the configuration space is 
% limited to just two realizations.}\label{fig:linkage-1}
% \end{figure}
% So here we have a linkage whose conifguration space is limited to just two realizations.  With just 
% two realizations, we can assign a binary value to them and have the linkage act as a boolean 
% variable.  We will revisit this concept when we cover satisfiability problems later on in the paper.
% \begin{figure}[h]
% \begin{center}
%   ~ %add desired spacing between images, e. g. ~, \quad, \qquad etc.
%     %(or a blank line to force the subfigure onto a new line)
%   \begin{subfigure}[b]{0.49\textwidth}
% 	  \includegraphics[width=\textwidth]{graphics/switchTerminalFinalized2.pdf}
% 	  \caption{A pentagon that is pinned in a channel junction that is formed by the sides of 3 
% large regular hexagons. It has two possible configurations, much like that of \ref{fig:linkage-1}}
% 	  \label{fig:linkage-2-1}
%   \end{subfigure}
%   \begin{subfigure}[b]{0.49\textwidth}
% 	  \includegraphics[width=\textwidth]{graphics/switchTerminalFinalized3.pdf}
% 	  \caption{A pinned pentagon residing in a channel junction that is formed by the sides of 
% 3 large regular hexagons with 2 dashed pentagons intersecting it.}
% 	  \label{fig:linkage-2-2}
%   \end{subfigure}
% \caption{Suppose the channel formed is a junction of three regular hexagons.  The polygon partially 
% residing in the junction is a regular hexagon with an equalateral triangle appended at an edge.  
% This polygon would prevent other polygons (i.e. the dashed polygons) of the same shape residing in 
% the center of the channel without intersection. This demonstrates that a the configuration space 
% within a multichannel environment can have concurrency issues, i.e. some configurations cannot be 
% realizable.}
% \end{center} \label{fig:linkage-2}
% \end{figure}\newpage
% Expanding upon the idea of \ref{fig:linkage-1}, forming channels with junctions as shown in Figure 
% \ref{fig:linkage-2} can be formed as such by evenly spacing the edges of a hexagonal lattice.  
% Visually, it is shown that only one of three possible pentagons can reside in the channel at one 
% time.  By asserting certain conditions on the lattice, and extending the problem to a greater 
% region 
% of a hexagonal lattice, we will be able to pose a realizability problem of whether a configuration 
% $\mathcal{A}$ can be reconfigured to $\mathcal{B}$ by switching pentagons without violating 
% overlapped polygon conditions.
% %Radius of regular polygons 
% %\newdimen\R
% %\R=3cm
% % \begin{figure}[h] 
% % \begin{center}
% % \begin{tikzpicture}
% % \begin{scope}
% % \filldraw[pattern=hexagons]  (0:\R) \foreach \x in {60,120,...,359} {
% %                 -- (\x:\R)
% %             }-- cycle (90:\R);
% % \end{scope}
% % \end{tikzpicture}
% % \caption{A hexagonal lattice contained in a hexagon.}
% % \label{fig:lattice}
% % \end{center}
% % \end{figure}
% %\newpage
% 
% % \begin{definition}[Graph]\label{def:linkages-2}
% % An ordered pair $G = (V, E)$ comprising a set $V$ of vertices or nodes together with a set $E$ of 
% edges or lines
% % \end{definition} 
% % \begin{definition}[Linkage]\label{def:linkages-1}
% % A collection of fixed-length 1D segments joined at their endpoints to form a graph.
% % \end{definition} 
% % A linkage can be thought of as a type of path-connected graph, i.e. the segments of a linkage are 
% the edges of a graph, and the endpoints of the segments are the vertices. For this paper, we 
% restrict our self to linkages that are simple planar graphs, i.e. a linkage that:
% % \begin{itemize}
% % \item[\rn{1}] does not have multiple edges between any pair of vertices,
% % \item[\rn{2}] does not have edges that cross, or
% % \item[\rn{3}] have loops (i.e. $(v,v) \in E$).
% % \end{itemize}  
% % \begin{definition}[Cycle]\label{def:linkages-3}
% %  A closed walk with no repetitions of vertices or edges allowed, other than the repetition of the 
% starting and ending vertex
% % \end{definition} 
% % \begin{definition}[Configuration]\label{def:linkages-6}
% % A specification of the location of all the link endpoints, link orientations and
% % joint angles.\cite{demaine2008geometric}
% % \end{definition}
% % \begin{definition}[Configuration Space]\label{def:linkages-7}
% % The space of all configurations of a linkage.
% % \end{definition} 
% % A configurations space is said to be continuous if for any two configurations, $\mathcal{A}$ and 
% $\mathcal{B}$ of a linkage $L$, $\mathcal{A}$ can be continuously reconfigured to $\mathcal{B}$ such 
% that, the reconfigurations reside in the configuration domain, $L$ remains rigid throughout 
% reconfiguration (i.e. all links' lengths are preserved), and no violations of linkage intersection 
% conditions. 
% % \begin{definition}[Pinned Joint]\label{def:linkages-8}
% % A vertex of a graph (or linkage) that is fixed to a position in a plane.
% % \end{definition} 
% % \begin{definition}[Free Joint]\label{def:linkages-8}
% % A vertex of a graph (or linkage) that is not fixed to a position in a plane.
% % \end{definition} x
% % \begin{figure}[h]
% % \begin{center}
% % \includegraphics[scale=.5]{graphics/randomLinkage.pdf}
% % \end{center} 
% % \caption{A linkage with joints.}
% % \end{figure} 
% % \begin{figure}[h]
% % \begin{center}
% % \includegraphics{graphics/freeJointPinnedJoint.pdf}
% % \end{center} 
% % \caption{The cross represents a free joint; the pinned joints are denoted as disks.  The range of 
% motion shown by the arc describes the continous configuration space of the linkage.}
% % \end{figure} 
% % 
% % For illustrations in the remainder of this paper, free joints will be represented as crosses and 
% pinned joints will be represented as disks.
