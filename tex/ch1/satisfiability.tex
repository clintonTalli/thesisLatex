\section{Satisfiability}
Let $x_1, \dots, x_n$ be boolean variables.  A boolean formula is a combination of conjunction, 
disjunctions, and negations of the boolean variables $x_1,  \dots, x_n$.   A boolean formula is 
\textit{satisfiable} if one can assign true or false value to each variable so that the formula is 
true. 
\begin{prob}[Satisfiability Problem (SAT)]\label{prob:Satisfiability-1}%Problem/Question
Given a boolean formula, decide whether it is satisfiable.
\end{prob} \cite{skiena2009algorithm}
It is known that every boolean formula can be rewritten in \textit{conjunctive normal form} (CNF), 
a conjunction of clauses.  A \textit{clause} is a disjunction of distinct literals.  
A \textit{literal} is a variable or a negated variable, $x_i$ or $\bar{x}_i$, for $i = 1,\dots,n$.  
Furthermore, it is also known that every boolean formula can be written in CNF such that each 
clause has exactly three literals.  This form is called 3-CNF.  Given a boolean formula in 3-CNF, 
decide whether it is satisfiable is a \textit{3-SAT problem}.

The problems we focus on in this thesis have a geometry.  A special geometric 3-SAT problem is that 
Planar 3-SAT Problem.   Given a 3-CNF boolean formula, $B$, we define the associated graph as 
follows: the vertices correspond to the variables and clauses in $B$, when a variable or its 
negation appears in a clause there is an edge between the corresponding vertices.

%draw a picture for this one.
\begin{prob}[Planar 3-SAT]
 Given a boolean formula $B$ in 3-CNF such that its associated graph is planar, decide whether it 
is satisfiable is a \textit{3-SAT problem}.
\end{prob}

% 
% Given a boolean 3-SAT formula $B$, define the associated graph of $B$ as follows:  
% \begin{equation}\label{eqn:sat-1}
% G(B) = \left(\set{v_x}{v_x\text{ represents a variable in }B} \cup \set{v_C}{v_C\text{ represent a 
% clause in }B}  , \set{\left( v_x, v_C\right) }{x \in C \text{ or } \bar{x} \in C}  \right) 
% \end{equation} 
% If $G(B)$ in equation (\ref{eqn:sat-1}) is planar, then $B$ is said to be a \textit{Planar 3-SAT 


\begin{prob}[Not All Equal 3 SAT Problem]\label{prob:Satisfiability-2}%Problem/Question
Given a boolean formula in 3-CNF, decide whether it is satisfiable so that each clause contains a 
true and false literal.
\end{prob}


3-SAT, PLANAR 3-SAT, NAE3SAT are NP hard [ADD REFERENCE].  THE PROOF FOR THESE REDUCTIONS ARE NOT 
OBVIOUS BUT ARE SHOWN [HERE, THERE, AND OVER THERE] respectively.
% Definition 1. (PLANAR 3-SAT) Let Φ be a
% Boolean formula in 3-CNF. The formula graph of
% Φ, G(Φ), has one variable-vertex vx for each variable
% x and one clause-vertex vC for each clause C. The
% variable-vertices vx are connected by edges to form a
% variable cycle, and for each clause-vertex vC an edge
% (vC, vx) is added if C contains either literal x or x.
% We say Φ is planar iff G(Φ) is planar. The PLANAR
% 3-SAT problem is equivalent to the 3-SAT problem
% restricted to planar formulae.
% Theorem 2.1. (Lichtenstein [14] Theorem 2)
% PLANAR 3-SAT is NP-complete
% \begin{prob}[Planar 3 SAT Problem]
%  A planar 3SAT instance is a 3SAT instance for which the graph built using the following rules is 
% planar:
% \begin{enumerate}
%  \item add a vertex for every $x_i$ and $\bar{x}_i$
%  \item add a vertex for every clause $C_j$
%  \item add an edge for every $\left(x_i,\bar{x}_i \right)$ pair
%  \item add an edge from vertex $x_i$ (or $\bar{x}_i$) to each vertex that represent a clause that 
% contains it
%  \item add edges between two consecutive variables $(x_1,x_2)$, $(x_2,x_3)$, $\dots$,$(x_n,x_1)$
% \end{enumerate}
% In particular, rule 5 builds a "backbone" that splits the clauses in two distinct regions.
% \end{prob}

%%Universality component
%\subsection{Logic Engine}
%The logic engine simulates the well known Not All Equal 3 SAT Problem (NAE3SAT).  
%%add figure of logic engine.
%\subsection{Construction of the Logic Engine}
%The components of the logic engine are as follows: the rigid frame, the shaft, the armatures, 
%the chains, and the flags.  The \textit{rigid frame} is a rectangular enclosure with a horizontal 
%shaft place at mid-height.  The \textit{armatures} are concentric rectangular frames contained 
%within the rigid frame.  Each armature can rotate about the shaft; other motions on the armature 
%are disallowed.  Given an NAE3SAT, for each variable there is a corresponding armature. On each 
%armature, there are chains.  A pair of \textit{chains}, $a_j$ and $\bar{a}_j$ correspond to the 
%variable $x_j$ and $\bar{x}_j$ respectively.  The pair is placed on each armature, reflected at a 
%height of $h$ above and below the shaft, i.e. one place above the shave at a height of $h$, the 
%other placed below the shaft at a height of $-h$.
%%insert an armature graphic
%
%\subsection{Encoding the Logic Engine}
%For each clause of an NAE3SAT, there exists a set of corresponding chains, namely the $h^\text{th}$ 
%clause is the set of chains on the armatures at the $h^\text{th}$ row above and below the shaft. A 
%chain is \textit{flagged} if the corresponding variable resides within the clause.  The flag can 
%point in either the left or right directions indicating a truth assignment for that variable within 
%the clause.  A flag is attached to the $i^\text{th}$ chain of every $a_j^\text{th}$ and 
%$\bar{a}_j^\text{th}$ chain with the following exceptions:
%\begin{enumerate}
% \item if the variable $x_j$  is in clause $C_i$, then link $i$ of $a_j$ is unflagged,
% \item if the variable $\bar{x}_j$ is in clause $C_i$, then link $i$ of $a_j$ is unflagged.
%\end{enumerate}
%\begin{thm}\label{thm:Satisfiability-1}
% An instance of $NAE3SAT$ is a ``yes'' instance if and only if the corresponding logic engine has a 
%flat, collision-free configuration.
%\end{thm}
%\begin{pf}
% If an instance of $NAE3SAT$ is a ``yes'', then every clause in $C$ contains at least one true 
%variable and one false variable.  Now suppose the following truth assignment:
%\begin{equation}\label{eqn:Satisfiability-1}
% t\left( x_j \right) = \left\lbrace\begin{array}{cr}
%  1 & x_j\text{ and }\bar{x}_j \text{are placed at the top and bottom respectively}\\
%  0 & x_j\text{ and }\bar{x}_j \text{are placed at the bottom and top respectively}\\
% \end{array}\right.
%\end{equation}
%For each clause $c_i \in C$, there exists a variables in $c_i$ such that $t\left( y_i \right) = 1$ 
%and $t\left( z_i \right) = 0$.  This implies that there exists an unflagged chain in the 
%$i^\text{th}$ and $-i^\text{th}$ row of the frame.  To avoid a collision in each row, trigger the 
%flags to point towards the unflagged link. Thus, the corresponding logic engine has a 
%flat, collision-free configuration.
%
%If the corresponding logic engine has a flat, collision-free configuration, then there must exist 
%an unflagged link in each row.  Without loss of generality, we have that the variables $y_j$ and 
%$z_i$ is in clause $C_i$ such that $t\left( y_i \right) = 1$ and $t\left( z_i \right) = 0$ for each 
%$i$.  Thus, we have an instance of $NAE3SAT$ is a ``yes'' instance. \cite{BET+99}
%\end{pf}

