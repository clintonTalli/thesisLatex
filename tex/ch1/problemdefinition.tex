\section{Problem}
The \emph{realizability} problem for a polygonal linkage asks whether a given polygonal linkage has 
a realization (resp., orientated realization). For a weighted planar (resp., plane) graph,, it asks 
whether the graph is
the contact graph (resp., ordered contact graph) of some disk arrangement with specified radii. 
These problems, in general, are known to be NP-hard. Specifically, it is NP-hard to decide whether a 
given planar (or plane) graph can be embedded in $\RR^2$ with given edge lengths~\cite{CDD+10,EW90}. 
Since an edge of given length can be modeled by a suitably long and skinny rhombus, the 
realizability of polygonal linkages is also NP-hard. The recognition of the contact graphs of unit 
disks in the plane (a.k.a. coin graphs) is NP-hard~\cite{BK98}, and so the realizability of weighted 
graphs as contact graphs of disks is also NP-hard. However, previous reductions crucially rely on 
configurations with high genus: the planar graphs in~\cite{CDD+10,EW90} and the coin graphs 
in~\cite{BK98} have many cycles.

In this thesis, we consider the above four realizability problems when the union of the polygons 
(resp., disks) in the desired configuration is simply connected (i.e., contractible). That is, the 
contact graph of the disks is a tree, or the ``hinge graph'' of the polygonal linkage is a tree (the 
vertices in the \emph{hinge graph} are the polygons in $\PP$, and edges represent a hinge between 
two polygons). Our main result is that realizability remains NP-hard when restricted to simply 
connected structures.
 
 % \subsection{Problem Statement} 
%write up thm 1, 2 (unoriented versions of the problem)
%write up thm 3, 4 (oriented versions of the problem)
\begin{prob}[Unordered Realizibility Problem for the Tree]\label{problem:UnorderedTree}
For a tree with positive weights for the verticies, it asks whether it is a contact graph of some 
disk arrangement where the radii are equal to the vertex weights.
\end{prob}
\begin{prob}[Unordered Realizibility Problem for Polygonal Linkages]\label{problem:UnorderedPolygonal}
The \emph{realizability} problem for a polygonal linkage asks whether a given polygonal linkage has 
a realization.
\end{prob}
\begin{prob}[Ordered Realizibility Problem for the Tree]\label{problem:OrderedTree}
For a tree with positive weights for the vertices, it asks whether its corresponding graph is the 
ordered contact graph of some disk arrangement where the radii equal the vertex weights.
\end{prob}
\begin{prob}[Ordered Realizibility Problem for Polygonal Linkages]label{problem:OrderedPolygonal}
The \emph{realizability} problem for a ordered polygonal linkage asks whether a given polygonal 
linkage has a realization with respect to order.
\end{prob}

\begin{thm}\label{thm:hinge}
It is NP-Hard to decide whether a polygonal linkage whose hinge graph is a tree can be realized 
(both with and without orientation).
\end{thm}

\begin{thm}\label{thm:disk}
It is NP-Hard to decide whether a given tree (resp., plane tree) with positive vertex weights
is the contact graph (resp., ordered contact graph) of a disk arrangements with specified radii.
\end{thm}

The unoriented versions, where the underlying graph (hinge graph or contact graph) is a tree can 
easily be handled with the logic engine method (Section~\ref{sec:logic}). We prove 
Theorem~\ref{thm:hinge} for \emph{oriented} realizations with a reduction from {\sc Planar-3SAT} 
(Section~\ref{sec:hinge}), and then reduce the realizability of ordered contact trees to the 
oriented realization of polygonal linkages by simulating polygons with arrangements of disks 
(Section~\ref{sec:disk}).


% \subsection{Decidability of Problem} test
% \subsection{Hexagonal Locked Configuration}