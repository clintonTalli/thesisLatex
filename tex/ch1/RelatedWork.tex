\section{Related Work and Results}
Previous research has established NP-hardness in several easy cases, but realizability for simply connected structures remained open. Polygonal linkages (or body-and-joint frameworks) are a generalization of classical linkages (bar-and-joint frameworks) in rigidity theory. A linkage is a graph $G=(V,E)$ with given edge lengths~\cite{CD-ch9}. A realization of a linkage is a (crossing-free) straight-line embedding of $G$ in the plane. Based on ideas developed by Bhatt and Cosmadakis~\cite{BC87}, who proved that the realizability of linkages is NP-complete on the integer grid, the \emph{logic engine} method~\cite{BET+99,EW96,FHW97,HK01} has become a standard tool for proving NP-hardness in graph drawing. The logic engine is a graph composed of rigid 2-connected components, where two possible realizations of a 2-connected component encode a binary variable.

However, the logic engine method is \textbf{not} applicable to problems with fixed embedding or orientation, where the circular order of the neighbors of each vertex is part of the input. Cabello et al.~\cite{CDR07,EW90} used a significantly more elaborate reduction to show that the realizability of 3-connected linkages (where the orientation is unique by Whitney's theorem~\cite{W33}) is NP-hard. This problem is efficiently decidable, though, for near-triangulations~\cite{CDR07,BV96}.

Note that every \emph{tree} linkage can be realized in $\RR^2$ with almost collinear edges. According to the celebrated \emph{Carpenter's Rule Theorem}~\cite{CDR03,Str05}, every realization of a path (or a cycle) linkage can be continuously moved (without self-intersection) to any other realization. In other words, the realization space of such a linkage is always connected. However, there are trees of maximum degree 3 with as few as 8 edges whose realization space is disconnected~\cite{BCD+09}; and deciding whether the realization space of a tree linkage is connected is PSPACE-complete~\cite{AKR+04}.
(Earlier, Reif~\cite{Rei79} showed that it is PSPACE-complete to decide whether a polygonal linkage can be moved from one realization to another among polygonal obstacles in $\RR^3$.)
Cheong et al.~\cite{CdG+07} consider the ``inverse'' problems of introducing the minimum number of point obstacles to reduce the configuration space of a polygonal linkage to a unique realization.

Connelly et al.~\cite{CDD+10} showed that the Carpenter's Rule Theorem generalizes to certain polygonal linkages obtained by replacing the edges of a path linkage with special polygons (called \emph{slender adornments}). Our Theorem~\ref{thm:hinge2} indicates that if we are allowed to replace the edges of a linkage with arbitrary convex polygons, then deciding whether the realization space is empty or not is already NP-hard.

Recognition problems for intersection graphs of various geometric object have a rich history~\cite{HK01}.
Breu and Kirkpatrick~\cite{BK98} proved that it is NP-hard to decide whether a graph $G$ is the contact graph of unit disks in the plane, i.e., recognizing \emph{coin graphs} is NP-hard; see also~\cite{BET+99}. Recognizing outerplanar coin graphs is already NP-hard, but decidable in linear time for caterpillars~\cite{KNR15}. It is also NP-hard to recognize the contact graphs of pseudo-disks~\cite{HK01} and disks of bounded radii~\cite{BK95} in the plane, and unit disks in higher dimensions~\cite{Hli97,HK01}. Eades and Wormald~\cite{EW90} showed that it is NP-hard to decide whether a given tree is a \emph{subgraph} of a coin graph. All these hardness reductions produce graphs with a large number of cycles, and do not apply to trees. Note that the contact graphs of disks \emph{of arbitrary radii} are exactly the planar graphs (by Koebe's circle packing theorem), and planarity testing is polynomial. Consequently, every tree is the contact graph of disks of \emph{some} radii in the plane.
However, deciding whether a given star is realizable as a contact graph of disks of given radii but arbitrary embedding is already NP-hard~\cite{KNR15}.

Schaefer~\cite{Sch13} proved that deciding whether a graph with given edge lengths can be realized by a straight-line drawing (possibly with crossing edges) has the same complexity as the existential theory of the reals. Both reductions crucially rely on a large number of cycles. Our work is the first to simulate rigid polygons with truly flexible combinatorial structures that have simply connected topology.

% Boris Klemz's Master's thesis ``Weighted Disk Contact Graph'' shows the Unit Disk Touching Graph Recognition problem is NP-Hard.  
% % In Chapter 2 we investigate recognition problems with uniform (and ρ-bounded) radii. In Section 2.1 we consider the Unit Disk Touching Graph Recognition (UDT) problem and strengthen the result of Breu and Kirkpatrick [BK98] by showing that the UDT problem is NP-hard even for outerplanar graphs. On the positive side, we provide a linear-time algorithm for deciding the UDT problem in caterpillars. In this section we also briefly consider ρ-bounded Disk Touching Graph Recognition and show that for spiders this problem can be solved in linear time in the Real RAM model. In Section 2.2 we extend our result from the previous section by showing that the Unit Disk Touching Graph Recognition with fixed Embedding (UDTE) problem is also NP-hard, even for outerplanar graphs.
% % In Chapter 3 we explore the more general scenario with fixed but not necessarily uniform radii. In Section 3.1 we consider the Disk Touching Graph Recognition with fixed Radii (DTR) problem and strengthen the result by Breu and Kirkpatrick [BK98] by showing that the DTR problem is NP-hard even for stars. We also show that for any cycle and a corresponding radius assignment there exists a realizing disk touching graph. In contrast, in Section 3.2 we devise a linear-time algorithm for deciding the Disk Touching Graph Recognition with fixed Radii and Embedding (DTRE) problem for stars in the Real RAM model.
% % In Chapter 4 we concern ourself with the scenario in which a seed assignment is required to be respected. In Section 4.1 we strengthen the result of Atienza et al. [AdCC+12] by showing that the Disk Touching Graph Recognition with fixed Seeds (DTS) prob- lem is NP-hard even for trees. In Section 4.2 we combine this scenario with assigning fixed radii, more specifically uniform radii. We show that the Unit Disk Touching Graph Recognition with fixed Seeds (UDTS) problem is NP-hard, even for paths, implying that the Unit Disk Touching Graph Recognition with fixed Seeds and Embedding (UDTSE) problem is also NP-hard even for paths.
% It also improves the Breu and Kirkpatrick result on Disk Touching Graph Recognition problem \cite{klemzthesis,BK98}.