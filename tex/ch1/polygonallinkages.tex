\section{Polygonal Linkages}
\begin{figure}[h]
\begin{center}
\includegraphics[scale=1]{graphics/hingeOnThreeDistinctPolygons.pdf}
\end{center} 
\caption{(a) A polygonal linkage with a non-convex polygon and two hinge points corresponding to 
three polygons.  Note that hinge points correspond to two distinct polygons.(b) Illustrating that 
two hinge points can correspond to the same boundary point of a polygon.}
\label{fig:linkage-1}
\end{figure}
Formally, a \textit{polygonal linkage} is an ordered pair $\left(\PP,\HH
\right)$ where $\PP$ is a finite set of polygons and $\HH$ is a finite set of hinges; a 
\textit{hinge} $h\in \HH$ 
corresponds to two points on the boundary of two distinct polygons in $\PP$.  A \emph{realization 
of a polygonal linkage} is an interior-disjoint placement of 
congruent copies of the polygons in $\PP$ such that the points corresponding to each hinge are 
identified (Fig. \ref{fig:linkage-1}). 

\subsection{Geometric Dissections}
The Wallace-Bolyai-Gerwien Theorem simply states that two polygons are congruent by dissection iff they have the same area.  A \textit{dissection} being a collection of smaller polygons that when hinged together form a polygon.

\begin{figure}[h]
\begin{center}
\includegraphics{graphics/GeometricDissectionBusschop.pdf}
\end{center}
\caption{Two configurations of polygonal linkage where the polygons touch on boundary segments 
instead of hinges.  These two realizations of the polygonal linkage are invalid to our definitions. 
 }
\label{fig:polygonallinkage-4}
\end{figure}

% With three cuts, dissect an equilateral triangle into a square. The problem was first proposed by Dudeney in 1902, and subsequently discussed in Dudeney (1958), and Gardner (1961, p. 34), Stewart (1987, p. 169), and Wells (1991, pp. 61-62). The solution can be hinged so that the four pieces collapse into either the triangle or the square. Two of the hinges bisect sides of the triangle, while the third hinge and the corner of the large piece on the base cut the base in the approximate ratio 0.982:2:1.018.
The Haberdasher problem was proposed in 1902 by Henry Dudeny which dissects an equilateral triangle into a square.
\begin{figure}[!htbp]
\begin{center}
\includegraphics{graphics/HaberdasherProblem.pdf}
\end{center}
\caption{The Haberdasher problem was proposed in 1902 and solved in 1903 by Henry Dudeny.  The dissection is for polygons that forms a square and equilateral traingle
 }
\label{fig:polygonallinkage-5}
\end{figure}



% %%%%%%%%%%%%%%%%%%%%%%%%%%%%%%%%%%%%%%%%%%%%%%%%%%%%%%%%%%%%%%%%%%%%
% %%%%%%%%%%%%%%%%%%%%%%%%%%%%%%%%%%%%%%%%%%%%%%%%%%%%%%%%%%%%%%%%%%%%
% %%%%%%%%%%%%%%%%%%%%%%%%%%%%%%%%%%%%%%%%%%%%%%%%%%%%%%%%%%%%%%%%%%%%
% %%%%%%%%%%%%%%%%%%%%%%%%%%%%%%%%%%%%%%%%%%%%%%%%%%%%%%%%%%%%%%%%%%%%
% %%%%%%%%%%%%%%%%%%%%%%%%%%%%%%%%%%%%%%%%%%%%%%%%%%%%%%%%%%%%%%%%%%%%
% %%%%%%%%%%%%%%%%%%%%%%%%%%%%%%%%%%%%%%%%%%%%%%%%%%%%%%%%%%%%%%%%%%%%
% %%%%%%%%%%%%%%%%%%%%%%%%%%%%%%%%%%%%%%%%%%%%%%%%%%%%%%%%%%%%%%%%%%%%
% \begin{figure}[h]
% \begin{center}
% \includegraphics[scale=1]{graphics/hingeOnThreeDistinctPolygons.pdf}
% \end{center} 
% \caption{(a) A polygonal linkage with a non-convex polygon and two hinge points corresponding to 
% three polygons.  Note that hinge points correspond to two distinct polygons.(b) Illustrating that 
% two hinge points can correspond to the same boundary point of a polygon.}
% \label{fig:linkage-1}
% \end{figure}
% %describe how it is a generalization of Linkages.
% A generalization of linkages are polygonal linkages where the edges of given lengths are replaced 
% by rigid polygons.  Formally, a \textit{polygonal linkage} is an ordered pair $\left(\PP,\HH 
% \right)$ where $\PP$ is a finite set of polygons and $\HH$ is a finite set of hinges; a 
% \textit{hinge} $h\in \HH$ 
% corresponds to two points on the boundary of two distinct polygons in $\PP$.  A \emph{realization} 
% of a polygonal linkage is an interior-disjoint placement of 
% congruent copies of the polygons in $\PP$ such that the points corresponding to each hinge are 
% identified (Fig. \ref{fig:1}). 
% A \textbf{realization with orientation} uses only translated or rotated copies of the polygons in $\PP$ (no reflections) and for each hinge, the cyclic order of incident polygons is given. 
% The topology of a polygonal linkage can be represented by the \textbf{hinge graph}, a bipartite graph where the vertices correspond to polygons in $\PP$ and the hinges in $H$, and edges represent the polygon-hinge incidences.
% This definition of realization rules well known geometric 
% dissections (e.g. Fig. \ref{fig:polygonallinkage-4}).
% %this is where the geometric dissection figure belongs
% \begin{figure}[h]
% \begin{center}
% \includegraphics{graphics/GeometricDissectionBusschop.pdf}
% \end{center}
% \caption{Two configurations of polygonal linkage where the polygons touch on boundary segments 
% instead of hinges.  These two realizations of the polygonal linkage are invalid to our definitions. 
%  }
% \label{fig:polygonallinkage-4}
% \end{figure}

% \begin{figure}[h]
% \begin{center}
% \includegraphics[scale=1]{graphics/linkageillustration.pdf}
% \end{center} 
% \caption{(a) A polygonal linkage with a non-convex polygon and a hinge point corresponding to three 
% polygons.  (b) A polygonal linkage with 8 regular polygons.}
% \label{fig:linkage-2}
% \end{figure}
% %%%%%%%%%%%%%%%%%%%%%%%%%%%%%%%%%%%%%%%%%%%%%%%%%%%%%%%%%%%%%%%%%%%%
% %%%%%%%%%%%%%%%%%%%%%%%%%%%%%%%%%%%%%%%%%%%%%%%%%%%%%%%%%%%%%%%%%%%%
% %%%%%%%%%%%%%%%%%%%%%%%%%%%%%%%%%%%%%%%%%%%%%%%%%%%%%%%%%%%%%%%%%%%%
% %%%%%%%%%%%%%%%%%%%%%%%%%%%%%%%%%%%%%%%%%%%%%%%%%%%%%%%%%%%%%%%%%%%%
% %%%%%%%%%%%%%%%%%%%%%%%%%%%%%%%%%%%%%%%%%%%%%%%%%%%%%%%%%%%%%%%%%%%%
% %%%%%%%%%%%%%%%%%%%%%%%%%%%%%%%%%%%%%%%%%%%%%%%%%%%%%%%%%%%%%%%%%%%%
% %%%%%%%%%%%%%%%%%%%%%%%%%%%%%%%%%%%%%%%%%%%%%%%%%%%%%%%%%%%%%%%%%%%%

% For the remainder of this thesis, we'll focus on the polygonal linkages with the following 
% restrictions:
% \begin{enumerate}
% \item Embedded polygons must be convex, i.e. for any two embedded points $u,v \in P'$, the set:
% $$\left\lbrace u  \cdot t + (1-t) \cdot v : t \in [0,1] \right\rbrace \in P'$$
%  \item  Polygons can only intersect at hinge points.  No two polygons can intersect at 
% their boundary or interior with the exception of possible a hinge point.  
% \end{enumerate}