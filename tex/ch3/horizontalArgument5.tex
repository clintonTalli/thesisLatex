\paragraph{Horizontal Displacement $\beta$}

We show that the relative horizontal displacement between two vertically adjacent obstacle hexagons is small and polynomial size.
We first identify where horizontal displacement can occur in the modified auxiliary construction.  

\begin{minipage}{\linewidth}
\begin{center}
\includegraphics[width=.99\columnwidth]{graphics/someRangeSkinny.pdf}
\captionof{figure}{(a) A pair of vertically adjacent obstacle hexagons and their corresponding corridor in canonical position.  (b) shows the same obstacle hexagons and corridors with the exception that the skinny rhombus is not in cannonical position. (c) is the same as (b) with the exception that the obstacle hexagon $O_{i+1}$ has rotation displacement.  (d) shows the rotation that is between the central flag and $O_i$.}\label{fig:someRangeSkinny.pdf}
\end{center}
\end{minipage}

Refer to the illustrations in Figure \ref{fig:someRangeSkinny.pdf}.  
Without loss of generality, the horizontal argument does not consider prior displacements with respect to the $\ith$ obstacle hexagon and focuses on the relative displacement between $O_{i}$ and $O_{i+1}$.  
If $O_i$ is fixed, then $O_{i+1}$ has three degrees of freedom.  
Together, the central flag and the skinny rhombus below $O_{i+1}$ have three hinges upon which these elements can move and ultimately move $O_{i+1}$:
\begin{enumerate}
\item The hinge that connects the central flag and $O_i$.
\item The hinge that connects the central flag and the skinny rhombus.
\item The hing that connects the skinny rhombus and $O_{i+1}$
\end{enumerate}
Recall that flags have two states, $R$ and $L$.  
The central flag would be realized in some sense as ``right'' or ``left''.   
Similarly, the skinny rhombus has in some sense a ``right'' or ``left'' realization.  
For the skinny rhombus, these realizations can either be above the central flag  or aside the flag (see Figure \ref{fig:betaOmegaFigure.pdf}).  
\begin{itemize}
\item Figure \ref{fig:someRangeSkinny.pdf}(a) shows the cannonical position of all objects.  
\item Figure \ref{fig:someRangeSkinny.pdf}(b) illustrates some range of angular motion $\omega_i$ that a skinny rhombus can rotate on its hinge with a flag.  
\item Figure \ref{fig:someRangeSkinny.pdf}(c) shows a rotational displacement of $O_{i+1}$ with the skinny rhombus' angular displacement.
\item Figure \ref{fig:someRangeSkinny.pdf}(d) is the same as (c) with the exception of an addtional rotation $\phi_i$ that is between the central flag and $O_i$.   
\end{itemize}
Each of these rotations can create horizontal displacement.  
We now show that each of these horizontal displacements are small.
% Let $\beta_{\alpha_i}$ be the horizontal displacement generated by the rotational displacement of $O_i$.  
% Let $\beta_{\omega_i}$ be the horizontal displacement generated by the rotational displacement between the skinny rhombus and the central flag.  
% Lastly, let $\beta_{\phi_i}$ be the horizontal displacement generated by the rotational displacement between the central flag and $O_i$.
\paragraph{\textit{Horizontal Displacement Generated by Rotational Displacement of the Obstacle Hexagon} $\beta_{\alpha_{i+1}}$}
We first show that the horizontal displacement generation by the rotational displacement of obstacle hexagon $O_{i+1}$.  
The obstacle hexagon above a corridor can rotate with respect the center of the hexagon.  
The half diameter of the obstacle hexagon is $ \frac{(5t-1)\sqrt{3}}{2}$.  
The rotation of the hexagon can generate a horizontal distance of travel by:
\begin{equation}
\begin{array}{rcl}
\sqrt{3} \frac{5t-1}{2} \sin \alpha_{i+1} &\leq&\sqrt{3} \frac{5t-1}{2}  \alpha_{i+1} \\
&\leq& \sqrt{3} \frac{5t-1}{2}  \frac{48}{s^{3\kappa-1}}\\
&\leq& \frac{48 \sqrt{3} \lr{5s^\kappa - 1}}{2s^{3\kappa-1}}\\
&\leq& \frac{100 \lr{6s^\kappa}}{2s^{3\kappa-1}}\\
&\leq& \frac{300}{s^{2\kappa-1}}\\
\end{array}
\end{equation}
Next we show the horizontal displacement generated by the rotation of the central flag.
\paragraph{\textit{Horizontal Displacement Generated by Rotational Displacement of the Central Flag} $\beta_{\phi_{i}}$}
Let $\phi$ be the anglular displacement of the central flag.  
This argument assumes the position of $O_i$ is in a relatively fixed, canonical position.  
Thus for any realization, we assume that placement of all other objects are relative to $O_i$.  

The maximum vertical displacement between $O_i$ and $O_{i+1}$ is $2 \delta_i$.  
Including the vertical displacement generated by the rotation of $O_{i+1}$, an upper bound on the vertical height of a corridor is:
%$$\frac{5t-1}{2} \lr{sin \lr{\frac{\pi}{3} + alpha_i}  - \frac{pi}{3} }$$
%$$v_\text{max}=\sqrt{3} + \frac{1}{100N} + 2 \delta_i + \sqrt{3}- 2 \sin \lr{\frac{\pi}{3} + \alpha_{i+1}}.$
$$v_\text{max}=\sqrt{3} + \frac{1}{100N} + 2 \delta_i + \frac{5t-1}{4}  \lr{ \sin \alpha_{i+1} +  \sin \alpha_{i} }$$
The central flag can rotate upto the height of the polygonal diameter of $v_\text{max}$.  
Using the Taylor series of sine and the following substitution $x = \frac{\pi}{3} + \phi_i$:
%sin(x) = \sin a + \cos a (x-a) - \frac{1}{2} \sin a (x-a)^2 - \frac{1}{6} \cos a (x-a)^3
\begin{eqnarray*}
2 \sin x &\geq&  2 \left( \sin \frac{\pi}{3} + \cos \frac{\pi}{3} \lr{x - \frac{\pi}{3}}\right.\\
&& \left.- \frac{1}{2} \sin \frac{\pi}{3} \lr{x - \frac{\pi}{3}}^2 - \frac{1}{6} \cos \frac{\pi}{3} \lr{x - \frac{\pi}{3}}^3\right. \\
&\geq&2 \lr{ \sin \frac{\pi}{3} + \cos \frac{\pi}{3} \lr{x - \frac{\pi}{3}}}\\
2 \lr{ \sin \frac{\pi}{3} + \cos \frac{\pi}{3} \lr{x - \frac{\pi}{3}}} &=& \sqrt{3} + \phi_i 
\end{eqnarray*}
In the inequality above, we bound $2 \sin  \lr{\frac{\pi}{3} + \phi_i}$ from below. 
We bound $\phi_i$ as follows: 
\begin{eqnarray*}
\sqrt{3} + \phi_i  &\leq & 2\sin \lr{ \frac{\pi}{3} + \phi_i }\\
2\sin \lr{ \frac{\pi}{3} + \phi_i } &\leq&\sqrt{3} + \frac{1}{100N} + 2 \delta_i + \frac{5t-1}{4}  \lr{ \sin \alpha_{i+1} +  \sin \alpha_{i} }\\
&\leq& \sqrt{3} + \frac{1}{100 \frac{5s^\kappa - 1}{2}} + 2 \frac{24}{s^{2\kappa-1}} + \frac{5s^\kappa-1}{2}  \sin \lr{\frac{48}{s^{3\kappa-1}}}\\
&\iff&\\
\phi_i &\leq&  \frac{1}{250s^\kappa - 50} +  \frac{48}{s^{2\kappa-1}} + \frac{5s^\kappa-1}{2} \frac{48}{s^{3\kappa-1}} \\
\phi_i &\leq& \frac{1}{200 s^\kappa} + \frac{48}{s^{2\kappa-1}} + \frac{24\lr{5s^\kappa - 1}}{s^{3\kappa-1}}\\
\phi_i &\leq& \frac{s^{2\kappa - 1} + 48s^{\kappa - 1}+ 120 s^\kappa}{s^{3\kappa-1}}\\
\phi_i &\leq& \frac{169}{s^{3\kappa - 1}}
\end{eqnarray*}
Lastly, we show that the horizontal displacement generated by the angle formed, $\omega_i$, by the hinge between the skinny rhombus and the central flag is small.
\paragraph{\textit{Horizontal Displacement Generated by Rotational Displacement of $\omega_i$}, $\beta_{\alpha_{i+1}}$}
The maximum vertical displacement between $O_i$ and $O_{i+1}$ is $2 \delta_i$.  
Including the vertical displacement generated by the rotation of $O_{i+1}$, an upper bound on the vertical height of the corridor between $O_{i+1}$ rotated by $\alpha_{i+1}$ and the central flag is:
$$\frac{1}{100N} + 2 \delta_i + \sqrt{3}- 2 \sin \lr{\frac{\pi}{3} - \alpha_{i-1}}.$$
The diameter of the rhombus, $\sqrt{1 + \lr{\frac{1}{100N}}^2}$, is bounded below by 1.
% State clearly that this is a lower bound (an approximation is not a
% lower or upper bound)
% In line 2 of the formula, you need a lower bound for sin omega_i. However,
%    sin omega_I \leq omega_i.
% Instead, for small values of omega, you can use the lower bound
%    omega_i/2 \leq sin omega_i.
% Also, k should be \kappa in these formulas.
% Page 20: "Conclusion of the Horizontal Argument"
% This is incomplete. So far, you have shown that the horizontal
% displacement between O_i and O_{i+1}
% is very close to -2, 0, or 2. You stil need to argue why it cannot be
% close to -2 and 2,

\begin{minipage}{\linewidth}
\begin{center}
\includegraphics[width=.46\columnwidth]{graphics/betaOmegaFigure.pdf}
\captionof{figure}{This figure shows the binary positions of the central flag and skinny rhombus.}\label{fig:betaOmegaFigure.pdf}
\end{center}
\end{minipage}

Figure \ref{fig:betaOmegaFigure.pdf} shows a central flag in the left and right position, each flag hinged with two skinny rhombi; the skinny rhombi depict its left and right position with respect to the central flag.
The skinny rhombus can either be in the position where the central flag is not below it, or where the central flag is below it.  
In the case that the central flag is not beneath the rhombus, the hoziontal displacement of $O_{i+1}$ becomes $\approx \pm 2$.  
This cannot be, because the adjacent corridors that is left (or right) of $O_{i+1}$ completely collapse and result in obstacle hexagons overlapping.
For small $\omega_i$, i.e. the case where the central flag \textit{is} beneath the skinny rhombus, we bound $\omega_i$ as follows:
\begin{eqnarray*}
\frac{\omega_i}{2}&\leq& \omega_i - \frac{\omega_i^3}{6} \\
\omega_i - \frac{\omega_i^3}{6} &\leq& \sin \omega_i\\
\sin \omega_i &\leq& \omega_i \\
\omega_i &\leq& \frac{1}{100N} + 2 \delta_i + \sqrt{3}- 2 \sin \lr{\frac{\pi}{3} + \alpha_{i+1}}\\
&\leq&+ \frac{1}{5s^\kappa - 1} + \frac{48}{s^{2\kappa - 1}} +\sqrt{3} - 2 \lr{\sin \frac{\pi}{3} + \cos \frac{\pi}{3} \lr{\alpha_{i+1} - \frac{\pi}{3} } }\\
&\leq&  \frac{s^{\kappa-1} + 192}{4 s^{2\kappa-1}} - \alpha_{i+1} - \frac{\pi}{3} + \frac{\pi}{3}\\
&\leq& \frac{48}{s^\kappa}
\end{eqnarray*}

\paragraph{\text{Conclusion of the Horizontal Argument}}  We've indirectly shown that the horizontal displacements are bounded by finding the bounds on the three degrees of freedom that can generate horizontal displacement for $O_{i+1}$: $\alpha_{i+1}$, $\phi_i$, and $\omega_i$.  
For each degree of freedom and corresponding angle, there exists a corresponding radius component.  
The total absolute displacement caused by $\alpha_{i+1}$, $\phi_i$, and $\omega_i$ is:
\begin{eqnarray*}
2\sqrt{3} \frac{5t-1}{2} \sin \frac{\alpha_{i+1}}{2} + 4 \sin \frac{\phi_i}{2} + 2 \sqrt{1 + \frac{1}{100N}} \sin \frac{\omega_i}{2}  \\
&\leq& \sqrt{3} \frac{5t-1}{2} \frac{\alpha_i}{2} + 4 \frac{\phi_i}{2} + 2 \sqrt{1 + \frac{1}{100N}} \frac{\omega_i}{2}  \\  
&\leq&  5s^\kappa \frac{24}{s^{3\kappa-1}} +  \frac{344}{s^{3\kappa - 1}} +  \frac{96}{s^\kappa}\\
&\leq& \frac{120s^\kappa + 344 + 96 s^{2\kappa - 1} }{s^{3\kappa - 1}} \\
&\leq& \frac{560 s^{2\kappa - 1} }{s^{3\kappa - 1}}\\
&\leq& \frac{560  }{s^\kappa }
\end{eqnarray*}