\paragraph{Horizontal Displacement $\beta$}

We show the horizontal displacement of an obstacle hexagon is small.
First we analyze the change of corridor's cross-section with respect to horizontal displacement.

\begin{minipage}{\linewidth}
\begin{center}
\includegraphics[width=.8\textwidth]{graphics/HorizontalCorridorArgument.pdf}
\captionof{figure}{(a) A corridor and corresponding junctions in canoncial position with cross section $N(n,m) \times (\sqrt{3} + (100N)^-1)$.  (b) A corridor whose total horizontal displacement is $\beta_{i+1} + \beta_{i-1}$ on the top and total vertical displacement of $\delta_{i+1} + \delta_{i-1}$ on the bottom; its cross section is $(N(n,m) + \beta_{i+1} + \beta_{i-1}) \times (\sqrt{3} + (100N)^{-1}) + \delta_{i+1} + \delta_{i-1}$. (c) shows the cross sectional area of a corridor with full rotational, horizontal, and vertical displacement.}\label{fig:HorizontalCorridorArgument.pdf}
\end{center}
\end{minipage}

Figure \ref{fig:HorizontalCorridorArgument.pdf} show the same corridor and corresponding junctions in canonical position and non-canoncical positions.  
The cross sectional areas for Figure \ref{fig:HorizontalCorridorArgument.pdf}(a) and Figure \ref{fig:HorizontalCorridorArgument.pdf}(b) are $N(n,m) \times (\sqrt{3} + (100N)^-1)$ and $(N(n,m) + \beta_{i+1} + \beta_{i-1}) \times (\sqrt{3} + (100N)^{-1}) + \delta_{i+1} + \delta_{i-1}$ respectively.  
Figure \ref{fig:HorizontalCorridorArgument.pdf}(c) shows the same corridor with rotational, horizontal, and vertical displacement.

Figure \ref{fig:HorizontalArgument.pdf} helps illustrate that in order to compute the diameter the hexagonal frame of a canonical construction, we can describe it in terms of obstacle hexagons, corridors and junctions.

\begin{minipage}{\linewidth}
\begin{center}
\includegraphics[width=.37\textwidth]{graphics/HorizontalArgument.pdf}
\captionof{figure}{The horizontal line $\ell_h$ intersects a row of shaded obstacle hexagons, corridors, and junctions.}\label{fig:HorizontalArgument.pdf}
\end{center}
\end{minipage}

By modifying Equation \ref{eqn:Jd} of $J_d(z)$, we can compute the number of obstacle hexagons intersecting $\ell_h$:
$$2z+1$$
The number of corridors intersecting $\ell_h$ is $2z$ and the number of junctions intersecting $\ell_h$ is $4z$.
All together, for a given construction the diameter (and length of $\ell_h$) is:
\begin{equation}\label{eqn:diameterLh}
(2z+1)2N(n,m) + 2z N(n,m) + 4z \frac{\sqrt{3}}{2} \lr{\sqrt{3}+\frac{1}{100N}}
\end{equation}
To generalize this for any row that $\ell_h$ may intersect where the row has obstacle hexagons on both ends, we use Sequence \ref{eqn:hexagonalGridSequence}:
$$a(i)2N(n,m) + \lr{a(i)-1} N(n,m) + 2(a(i)-1) \frac{\sqrt{3}}{2} \lr{\sqrt{3}+\frac{1}{100N}}$$
For the other rows where this is not the case, it generalizes to:
$$a(i) 2N(n,m) + \lr{a(i) + 1} N(n,m) + \lr{2 a(i) + 2} \lr{\sqrt{3}+\frac{1}{100N}}$$
These upper bounds will be used to show the upper bound of horizontal displacement, $\beta$.
Now suppose we're given an instance of a modified auxilary construction with the $i^\text{th}$ row having a horizontal displacement at the $j^\text{th}$ column.
The $\jth$ corridor will have a cross sectional length of:
$$\left\lbrace
\begin{array}{cl}
N(n,m)+\beta_{i+1}+\beta_{i-1}&1<i<a(i)\\
N(n,m)+\beta_{i+1}&i=1\\
N(n,m)+\beta_{i-1}&i=a(i)
\end{array}\right.$$ 
and the $\ith$ row will have a length of either:
$$a(i)2N(n,m) + \lr{a(i)-1} N(n,m) + 2(a(i)-1) \frac{\sqrt{3}}{2} \lr{\sqrt{3}+\frac{1}{100N}}+\beta_{i+1}+\beta_{i-1}$$
or:
$$a(i) 2N(n,m) + \lr{a(i) + 1} N(n,m) + \lr{2 a(i) + 2} \lr{\sqrt{3}+\frac{1}{100N}}+\beta_{i+1}+\beta_{i-1}$$
% For bounding the horizontal displacement, draw a horizontal line across a "row" of obstacle hexagons. We can do this because the anle and the vertical displacement are already bounded. Then use the same argument as in (1.6) and (1.7), with a slightly adjusted argument. Specifically, we need to tell what the cross section of a corridor would be in the canonical position. It is (sqrt{3}+1/(100N)/sin(pi/6)  with our current parameters. If the two obstacles on the two sides of the corridor have some angle error alpha and vertical error delta, then the horizontal cross-section of the corridor may be smaller, but only a little bit smaller (use trig to approximate the cross-section).
\end{proof}