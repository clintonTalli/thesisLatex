\paragraph{Horizontal Displacement $\beta$}

We show the horizontal displacement of an obstacle hexagon is small.
First we analyze the change of corridor's cross-section and the change in junction with respect to horizontal displacement.

\begin{minipage}{\linewidth}
\begin{center}
\includegraphics[width=.37\textwidth]{graphics/HorizontalJunctionArgument.pdf}
\captionof{figure}{This figure shows a junction formed by three adjacent obstacle hexagons with horizontal and vertical displacement. }\label{fig:HorizontalJunctionArgument.pdf}
\end{center}
\end{minipage}

A canonical junction has an area of $$\frac{\sqrt{3}}{2} \lr{\sqrt{3}+ \frac{1}{100N}}^2.$$
When a junction is formed by three adjacent obstacle hexagons that are displaced only vertically and horizontally, the area of the junction becomes:
$$\alpha$$

\begin{minipage}{\linewidth}
\begin{center}
\includegraphics[width=.37\textwidth]{graphics/HorizontalCorridorArgument.pdf}
\captionof{figure}{The horizontal line $\ell_h$ intersects a row of shaded obstacle hexagons, corridors, and junctions.}\label{fig:HorizontalCorridorArgument.pdf}
\end{center}
\end{minipage}
% For bounding the horizontal displacement, draw a horizontal line across a "row" of obstacle hexagons. We can do this because the anle and the vertical displacement are already bounded. Then use the same argument as in (1.6) and (1.7), with a slightly adjusted argument. Specifically, we need to tell what the cross section of a corridor would be in the canonical position. It is (sqrt{3}+1/(100N)/sin(pi/6)  with our current parameters. If the two obstacles on the two sides of the corridor have some angle error alpha and vertical error delta, then the horizontal cross-section of the corridor may be smaller, but only a little bit smaller (use trig to approximate the cross-section).
\begin{minipage}{\linewidth}
\begin{center}
\includegraphics[width=.9\textwidth]{graphics/HexagonalGridRho.pdf}
\captionof{figure}{(a) shows the displacement of an obstacle hexagon $(\alpha, \beta, \delta)$ where $(\beta_{i,j},\delta_{i,j})$ is the displacement and $\left(\beta_{i,j_0},\delta_{i,j_0}\right)$ is the canonical position of the obstacle hexagon. (b) illustrates three rows of obstacle hexagons in canonical position.  If an obstacle hexagon in the middle row is moved, then it may affect the adjacent obstacle hexagons.
}\label{fig:Hexagonal.pdf}
\end{center}
\end{minipage}


\end{proof}