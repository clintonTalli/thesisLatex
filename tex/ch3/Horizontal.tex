\paragraph{Horizontal Displacement $\beta$}

We show the horizontal displacement of an obstacle hexagon is small.

\begin{minipage}{\linewidth}
\begin{center}
\includegraphics[width=.9\textwidth]{graphics/HexagonalGridRho.pdf}
\captionof{figure}{(a) shows the displacement of an obstacle hexagon $(\alpha, \beta, \delta)$ where $(\beta_{i,j},\delta_{i,j})$ is the displacement and $\left(\beta_{i,j_0},\delta_{i,j_0}\right)$ is the canonical position of the obstacle hexagon. (b) illustrates three rows of obstacle hexagons in canonical position.  If an obstacle hexagon in the middle row is moved, then it may affect the adjacent obstacle hexagons.
}\label{fig:Hexagonal.pdf}
\end{center}
\end{minipage}

Given the diameter of an obstacle hexagon is $2N(n,m)$ and the canonical length of the $i\text{th}$ row is $$a(i) \cdot 2N(n,m) + \lr{a(i) - 1}\frac{\lr{\sqrt{3}+1/(100N)}\sqrt{3}}{2}$$.

\begin{equation*}
a(i)  2N(n,m) + \lr{a(i) - 1} \frac{
										\lr{\sqrt{3}+1/(100N)}\sqrt{3}
									}
									{
										2
									} \leq (k+i)2N + \lr{k+i - 1}\frac{\lr{\sqrt{3}+1/(100N)}\sqrt{3}}{2}
\end{equation*}

where $k$ is determined by $J_z$ corresponding to Equation \ref{eqn:hexagonalGridSequence}

$$4z \cdot \frac{3}{2} + a\lr{i_\text{max}} N(m,n)$$

Given an arbitrary modified auxilary construction where the $j^\text{th}$ obstacle of the $i^\text{th}$ row is in non-canonical position, if the center of the obstacle hexagon is displaced by $\lr{\alpha_{i,j}, \beta_{i,j}, \delta_{i,j}}$, where $\beta_{i,j} > 0 $, then 
\begin{eqnarray*}
\lr{2N - \beta_{i,j}} + \lr{a \lr{i_{max}} - j } 2 N + \lr{a \lr{i_{max}}-j } \frac{\lr{\sqrt{3}+1/(100N)}\sqrt{3}}{2}&\leq& \lr{a \lr{i_{max}} - (j + 1)} 2 N\\
&& + \lr{a \lr{i_{max}}-j } \frac{\lr{\sqrt{3}+1/(100N)}\sqrt{3}}{2}
\end{eqnarray*}
\end{proof}
