\paragraph{Horizontal Displacement $\beta$}

We show that the relative horizontal displacement between two vertically adjacent obstacle hexagons is small and polynomial size.
We first identify where horizontal displacement can occur in the modified auxiliary construction.  

\begin{minipage}{\linewidth}
\begin{center}
\includegraphics[width=.99\columnwidth]{graphics/someRangeSkinny.pdf}
\captionof{figure}{(a) A pair of vertically adjacent obstacle hexagons and their corresponding corridor in canonical position.  (b) shows the same obstacle hexagons and corridors with the exception that the skinny rhombus is not in cannonical position. (c) is the same as (b) with the exception that the obstacle hexagon $O_{i+1}$ has rotation displacement.  (d) shows the rotation that is between the central flag and $O_i$.}\label{fig:someRangeSkinny.pdf}
\end{center}
\end{minipage}

Refer to the illustrations in Figure \ref{fig:someRangeSkinny.pdf}.  
Without loss of generality, the horizontal argument does not consider prior displacements with respect to the $\ith$ obstacle hexagon and focuses on the relative displacement between $O_{i}$ and $O_{i+1}$.  
If $O_i$ is fixed, then $O_{i+1}$ has three degrees of freedom.  
Together, the central flag and the skinny rhombus below $O_{i+1}$ have three hinges upon which these elements can move and ultimately, move $O_{i+1}$.  
Flags have two states, $R$ and $L$.  
The central flag would be realized in some sense as ``right'' or ``left''.   
Similarly, the skinny rhombus has in some sense a ``right'' or ``left'' realization.  
For the skinny rhombus, these realizations can either be above the central flag  or aside the flag (see Figure \ref{fig:betaOmegaFigure.pdf}).  
\begin{itemize}
\item Figure \ref{fig:someRangeSkinny.pdf}(a) shows the cannonical position of all objects.  
\item Figure \ref{fig:someRangeSkinny.pdf}(b) illustrates some range of angular motion $\omega_i$ that a skinny rhombus can rotate on its hinge with a flag.  
\item Figure \ref{fig:someRangeSkinny.pdf}(c) shows a rotational displacement of $O_{i+1}$ with the skinny rhombus' angular displacement.
\item Figure \ref{fig:someRangeSkinny.pdf}(d) is the same as (c) with the exception of an addtional rotation $\phi_i$ that is between the central flag and $O_i$.  
Each of these rotations can create horizontal displacement.  
\end{itemize}
Let $\beta_{\alpha_i}$ be the horizontal displacement generated by the rotational displacement of $O_i$.  
Let $\beta_{\omega_i}$ be the horizontal displacement generated by the rotational displacement between the skinny rhombus and the central flag.  
Lastly, let $\beta_{\phi_i}$ be the horizontal displacement generated by the rotational displacement between the central flag and $O_i$.
\paragraph{\textit{Horizontal Displacement Generated by Rotational Displacement} $\beta_{\alpha_{i+1}}$}
The obstacle hexagon above a corridor can rotate with respect the center of the hexagon.  
The half diameter of the obstacle hexagon is $ \frac{(5t-1)\sqrt{3}}{2}$.  
The rotation of the hexagon can generate a horizontal distance of travel by:
\begin{equation}
\begin{array}{rcl}
\sqrt{3} \frac{5t-1}{2} \sin \alpha_{i+1} &\leq&\sqrt{3} \frac{5t-1}{2}  \alpha_{i+1} \\
&\leq& \sqrt{3} \frac{5t-1}{2}  \frac{48}{s^{3\kappa-1}}\\
&\leq& \frac{48 \sqrt{3} \lr{5s^\kappa - 1}}{2s^{3\kappa-1}}\\
&\leq& \frac{100 \lr{6s^\kappa}}{2s^{3\kappa-1}}\\
&\leq& \frac{300}{s^{2\kappa-1}}\\
\end{array}
\end{equation}

\paragraph{\textit{Horizontal Displacement Generated by Central Flag Rotation} $\phi_i$}
Recall that the greatest vertical displacement in any corridor is $2\delta_i$ from Inequality \ref{eqn:alphaBoundRefined}.  
This limits the possible range of motion of any flag and the skinny rhombus in a given corridor.  
Figure \ref{fig:betaOmegaFigure.pdf} shows a central flag in the left and right position, each flag hinged with two skinny rhombi.
These illustrations depict feasable realizations that meet the constraints of total corridor displacement $2 \delta_i$.  
In the realizations where the skinny rhombus is \textit{not} above the central flag, the horizontal displacement is $\approx \pm 2$ units. 

\begin{minipage}{\linewidth}
\begin{center}
\includegraphics[width=.46\columnwidth]{graphics/betaOmegaFigure.pdf}
\captionof{figure}{}\label{fig:betaOmegaFigure.pdf}
\end{center}
\end{minipage}

\paragraph{\textit{Horizontal Displacement Generated by} $\omega_i$}
Relative to the position of obstacle hexagons $O_i$ and $O_{i+1}$, the maximum vertical displacement that can be created in the corridor between $O_i$ and $O_{i+1}$ of any realization is $2 \delta$.  
This limits vertical distance $v_h$ for with the flags and skinny rhombus can realized in a corridor to:
$$\sqrt{3}\leq v_h \leq \sqrt{3} + \frac{1}{100N} + 2 \delta$$
This limitation on $v_h$ allows for several configurations of the central flag and skinny rhombus.  
Recall that the central flag and the skinny rhombus below $O_{i+1}$ have three hinges upon which these elements can move and ultimately, move $O_{i+1}$.  
For any realization the central flag will have some angle $\phi$ formed from the bottom side of the central flag and the adjacent side of the obstacle hexagon $O_i$; the range of $\phi$ is $-\alpha \leq \phi \leq \alpha$.  
Should $\phi$ displace the central flag up to upper bound height of $v_h$, by Taylor series of $\sin x$:
\begin{eqnarray*}
2 \sin \lr{\frac{\pi}{3}+\phi} - 2 \sin \lr{\frac{\pi}{3}} &=& 2\lr{\frac{\sqrt{3}}{2} + \frac{1}{2} \lr{\phi - \frac{\pi}{3} } - \frac{\sqrt{3}}{4}  \lr{\phi - \frac{\pi}{3}}^2 - \frac{\sqrt{3}}{12}  \lr{\phi - \frac{\pi}{3}}^3   - \frac{\sqrt{3}}{2}} \\
&\leq& 2 \lr{\frac{1}{2} \lr{\phi}}\\
&\leq& \phi\\
&\leq& 2 \delta  
\end{eqnarray*}




\begin{minipage}{\linewidth}
\begin{center}
\includegraphics[width=.99\columnwidth]{graphics/phiFigure.pdf}
\captionof{figure}{}\label{fig:phiFigure.pdf}
\end{center}
\end{minipage}


