\paragraph{Horizontal Displacement $\beta$}

We show that the relative horizontal displacement between two vertically adjacent obstacle hexagons is small and polynomial size.
We first identify where horizontal displacement can occur in the modified auxiliary construction.  

\begin{minipage}{\linewidth}
\begin{center}
\includegraphics[width=.99\columnwidth]{graphics/someRangeSkinny.pdf}
\captionof{figure}{(a) A pair of vertically adjacent obstacle hexagons and their corresponding corridor in canonical position.  (b) shows the same obstacle hexagons and corridors with the exception that the skinny rhombus is not in cannonical position. (c) is the same as (b) with the exception that the obstacle hexagon $O_{i+1}$ has rotation displacement.  (d) shows the rotation that is between the central flag and $O_i$.}\label{fig:someRangeSkinny.pdf}
\end{center}
\end{minipage}

Refer to the illustrations in Figure \ref{fig:someRangeSkinny.pdf}.  
Without loss of generality, the horizontal argument does not consider prior displacements with respect to the $\ith$ obstacle hexagon and focuses on the relative displacement between $O_{i}$ and $O_{i+1}$.  
If $O_i$ is fixed, then $O_{i+1}$ has three degrees of freedom.  
Together, the central flag and the skinny rhombus below $O_{i+1}$ have three hinges upon which these elements can move and ultimately, move $O_{i+1}$.  
Figure \ref{fig:someRangeSkinny.pdf}(a) shows the cannonical position of all objects.  
Figure \ref{fig:someRangeSkinny.pdf}(b) illustrates some range of angular motion $\omega_i$ that a skinny rhombus can rotate on its hinge with a flag.  
Figure \ref{fig:someRangeSkinny.pdf}(c) shows a rotational displacement of $O_{i+1}$ with the skinny rhombus' angular displacement.
Figure \ref{fig:someRangeSkinny.pdf}(d) is the same as (c) with the exception of an addtional rotation $\phi_i$ that is between the central flag and $O_i$.  
Each of these rotations can create horizontal displacement.  
Let $\beta_{\alpha_i}$ be the horizontal displacement generated by the rotational displacement of $O_i$.  
Let $\beta_{\omega_i}$ be the horizontal displacement generated by the rotational displacement between the skinny rhombus and the central flag.  
Lastly, let $\beta_{\phi_i}$ be the horizontal displacement generated by the rotational displacement between the central flag and $O_i$.
\paragraph{\textit{Horizontal Displacement Generated by Rotational Displacement} $\beta_{\alpha_{i+1}}$}
The obstacle hexagon above a corridor can rotate with respect the center of the hexagon.  
The half diameter of the hexagon is $\sqrt{3} \frac{5t-1}{2}$.  
The rotation of the hexagon can generate a horizontal distance of travel by:
\begin{equation}
\begin{array}{rcl}
\sqrt{3} \frac{5t-1}{2} \sin \alpha_{i+1} &\leq&\sqrt{3} \frac{5t-1}{2}  \alpha_{i+1} \\
&\leq& \sqrt{3} \frac{5t-1}{2} \alpha_{i+1}\\
&\leq& \sqrt{3} \frac{5s^\kappa-1}{2} \frac{2}{s^\kappa}\\
&\leq& 2 \frac{5s^\kappa}{2} \frac{2}{s^\kappa}\\
&\leq& 10\\
\end{array}
\end{equation}
\paragraph{\textit{Horizontal Displacement Generated by} $\omega_i$}

\begin{minipage}{\linewidth}
\begin{center}
\includegraphics[width=.99\columnwidth]{graphics/phiFigure.pdf}
\captionof{figure}{}\label{fig:phiFigure.pdf}
\end{center}
\end{minipage}

\paragraph{\textit{Horizontal Displacement Generated by Central Flag Rotation} $\phi_i$}


\begin{minipage}{\linewidth}
\begin{center}
\includegraphics[width=.46\columnwidth]{graphics/betaOmegaFigure.pdf}
\captionof{figure}{}\label{fig:betaOmegaFigure.pdf}
\end{center}
\end{minipage}