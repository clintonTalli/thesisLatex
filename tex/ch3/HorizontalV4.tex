\paragraph{Horizontal Displacement $\beta$}

We show the horizontal displacement of an obstacle hexagon is small.
First we analyze the change of corridor's cross-section with respect to horizontal displacement.

\begin{minipage}{\linewidth}
\begin{center}
\includegraphics[width=.8\textwidth]{graphics/HorizontalCorridorArgument.pdf}
\captionof{figure}{(a) A corridor and corresponding junctions in canoncial position with cross section $N(n,m) \times (\sqrt{3} + (100N)^{-1})$.  (b) A corridor whose total horizontal displacement is $\beta_{i+1} + \beta_{i-1}$ on the top and total vertical displacement of $\delta_{i+1} + \delta_{i-1}$ on the bottom; its cross section is $(N(n,m) + \beta_{i+1} + \beta_{i-1}) \times (\sqrt{3} + (100N)^{-1}) + \delta_{i+1} + \delta_{i-1}$. (c) shows the cross sectional area of a corridor with full rotational, horizontal, and vertical displacement.}\label{fig:HorizontalCorridorArgument.pdf}
\end{center}
\end{minipage}
Figure \ref{fig:HorizontalCorridorArgument.pdf} shows the same corridor and corresponding junctions in canonical position and non-canoncical positions.
This figure is of a corridor and adjacent junctions formed by obstacle hexagons in counter-clockwise order: $O_{i-1,j}$, $O_{i,j-1}$, $O_{i+1,j}$, and $O_{i,j+1}$.
The cross sectional areas for Figure \ref{fig:HorizontalCorridorArgument.pdf}(a) and Figure \ref{fig:HorizontalCorridorArgument.pdf}(b) are $N(n,m) \times (\sqrt{3} + (100N)^-1)$ and $(N(n,m) + \beta_{i+1} + \beta_{i-1}) \times (\sqrt{3} + (100N)^{-1}) + \delta_{i+1} + \delta_{i-1}$ respectively.  
Figure \ref{fig:HorizontalCorridorArgument.pdf}(c) shows the same corridor with rotational, horizontal, and vertical displacement.
To find the crosss sectional area of the corridor in Figure \ref{fig:HorizontalCorridorArgument.pdf}(c), we decompose it into three parts, the upper tringle, the reectangle, and the lower triangle.  

We're given $N(n,m)$ and $(\alpha, \beta, \delta)_{i,j}$ for all $i,j = 1, \dots, u$. 
The area of the upper triangle is 
\begin{equation}\label{eqn:upperTriangle}
\frac{1}{2}\lr{N+\beta_{i+1} + \beta_{i-1}}^2 \tan \lr{\alpha_{i+1,j}}.
\end{equation}
The area of the rectangle is 
\begin{equation}\label{eqn:rectangle}
\lr{N + \beta_{i+1} + \beta_{i-1}} \cdot \lr{\lr{\sqrt{3} + (100N)^{-1}} + \delta_{i+1} + \delta_{i-1}}.
\end{equation}
The area of the lower triangle is 
\begin{equation}\label{eqn:lowerTriangle}
\frac{1}{2} \lr{N \cdot \cos \lr{ \alpha_{i-1,j} } } \cdot \lr{N \cdot \sin \lr{\alpha_{i-1,j}} }= N^2 \sin \lr{2 \cdot \alpha_{i-1,j}}.
\end{equation}



% I think the simplest argument would go by induction on the "height" of
% the obstacle hexagon.
% Assume that angular and vertical displacement of every obstacle hexagon
% is small.
Note that the bottom most obstacle hexagon $O_{j,1}$ is hinged to the frame or hinged to some half sized hexagon such that $\beta_{j,1} = 0$.  
The range of motion horizontal motion that $O_{j,2}$ has is dictated by the range of possible motion from the skinny rhombus between $O_{j,1}$ and $O_{j,2}$.
The diameter of the skinny rhombus is $\sqrt{1+(100N)^{-1}}$.  
%An upperbound of $\beta_{j,2}$ is $\sqrt{1+(100N)^{-1}}$, however it is not a small or tight bound.  
If $\beta_{j,2} = \pm \sqrt{1+(100N)^{-1}}$, there would be a collision a  corridor(s) adjacent to $O_{j,2}$.  

\begin{minipage}{\linewidth}
\begin{center}
\includegraphics[width=.9\columnwidth]{graphics/corridorNonCanonical2.pdf}
\captionof{figure}{The full range of motion is shown dashed half circle about the diameter of the skinny rhombus.}\label{fig:corridorNonCanonical2.pdf}
\end{center}
\end{minipage}

Figure \ref{fig:corridorNonCanonical2.pdf} shows the range of motion of the skinny rhombus.
The angle formed between the diameter of the rhombus and the half length of the corridor is $\omega_i$.
The relative difference between 
% Consider one hexagon (alpha,beta,delta)_{(i,j)}, and further assume that
% the hexagons bellow, lower-left, and lower-right have small horizontal
% displacement
% (here "small" can be quantified using the bounds (1.7) and (1.8)).
% Then we need to show that beta_{(i,j)} is also "small"
% Because of the connector between the hexagon (i,j) and the one directly
% below (i,j-1),
% the horizontal displacement is at most about beta_{(i,j-1)} (1-cos
% (gamma_{(i,j)})).
\end{proof}