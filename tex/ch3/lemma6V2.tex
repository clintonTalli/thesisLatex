% The position of each hexagon can be defined by the isometry from its canonical position; an isometry is given by the triple $\lr{\alpha, \beta, \delta}$ where $\alpha$ is a counter clockwise rotation about the center of the hexagon and $\lr{\beta,\delta}$ is a translation vector.% where $\beta$ is the translation of the obstacle hexagon in the $x$ axis and $\delta$ is the translation of the obstacle hexagon in the $y$ axis.
% Canonical position would have each obstacle hexagon's position as $(0,0,0)$.
\begin{lem}\label{lem:aux-C}
Let P be a polygonal linkage obtained from the modified auxilary construction.  
In every realization of $P$, the obstacle polygons are close to canonical position such that 
\end{lem}
% The bottom most obstacle hexagon is glued to the side of the bottom frame hexagon.  
% The subsequent obstacle hexagons are rotated 
%Our goal is show that this quality cannot happen in the modified auxilary construction.  
Lemma \ref{lem:aux-C} serves as assurance that once a boolean formula of P3SAT is encoded into an arbitrary realization of the modified auxilary construction, the information of the boolean formula is preserved regardless of the positioning of the gadgets and components in the construction.
This quality shows that the information is stable and preserved in an arbitrary realization of the modified auxilary construction.
In Figure \ref{fig:tiltedObstaclesInFrame.pdf}, we have a column of obstacle hexagons veering off $\ell$.
This is an example of extreme angular rotation that should not occur over a vertical stack of hexagons.

% \begin{minipage}{\linewidth}
% \begin{center}
% \includegraphics[width=.3\columnwidth]{graphics/tiltedObstaclesInFrame.pdf}
% \captionof{figure}{This figure depicts a column of obstacle hexagons rotated such that the obstacle hexagons veer of the vertical line $\ell$.}\label{fig:tiltedObstaclesInFrame.pdf}
% \end{center}
% \end{minipage}
\begin{proof}
We need to show that the modified auxilary construction could not deform in such a way that any information the construction encodes is lost or modified and the functionality of the gadgets within the construction behave as stated in the description.

To help identify components of the construction for this proof, let's identify components in the canonical position:

\begin{minipage}{\linewidth}
\begin{center}
\includegraphics[width=.5\columnwidth]{graphics/dualSmallHexagonalGrid.pdf}
\captionof{figure}{(a) depicts a column of obstacle hexagons $O_1$, $\ldots$, $O_{10}$ along the vertical line $\ell$; (b) identifies obstacle hexgons $O_1$, $\ldots$, $O_{10}$ in (a).}\label{fig:dualSmallHexagonalGrid.pdf}
\end{center}
\end{minipage}

Without loss of generality, we can identify a column of obstacle hexagons $O_i$ along a vertical line $\ell$ (See Figure \ref{fig:dualSmallHexagonalGrid.pdf}).
In this proof, unless otherwise specified, we assume that the argument refers to a column that starts and ends with an obstacle hexagon.  
In total there will be $u+1$ number of obstacle hexagons and $u$ corridors in a column. 

The length of $H(n,m)$ (and $\ell$ in Figure \ref{fig:corridorNonCanonical.pdf}(a)) can be expressed as a sum of the heights of the corridors and obstacle polygons.
The width of a skinny rhombus in canonical position is $\frac{1}{100N}$.
The obstacle hexagon has height of $ \frac{5t-1}{2} \cdot \sqrt{3}$, and the flag is of height $\sqrt{3}$.  
\begin{equation}\label{eqn:Hnm}
H(n,m) = (u+1) \frac{5t-1}{2} \sqrt{3} + u \lr{\frac{1}{100N} + \sqrt{3}}
\end{equation}

The cross section of the corridor must have a minimum height of $\sqrt{3}$ everywhere.
The height of an obstacle polygon in noncanonical position is $(t+1) \cdot \sec \alpha_i \cdot \sqrt{3}$.


