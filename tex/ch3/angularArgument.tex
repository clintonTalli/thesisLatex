\paragraph{Angular Rotation $\alpha$.}
First we show that the angular rotation of the obstacle hexagons with respect to canonical position is small.  
We first look at the relative angular difference between two adjacent obstacle polygons
$$\left\vert \alpha_i - \alpha_{i+1} \right\vert.$$
Given an arbitrary instance of a modified auxilary construction, consider $O_i$, $O_{i+1}$, and the corridor between $O_i$ and $O_{i+1}$ (see Figure \ref{fig:corridorNonCanonical2.pdf} for illustration).  
For this argument, we assume that $O_i$ is horizontal and consider the relative position of other objects with respect to $O_{i}$.

\begin{minipage}{\linewidth}
\begin{center}
\includegraphics[width=.9\columnwidth]{graphics/corridorNonCanonical2.pdf}
\captionof{figure}{The obstacle hexagon here is in noncanonical position, and showing the side lengths adjacent to $\alpha_i$.}\label{fig:corridorNonCanonical2.pdf}
\end{center}
\end{minipage}

Figure \ref{fig:corridorNonCanonical2.pdf} $O_{i+1}$ is in a non-canonical position.  We describe the paramaters of the figure:
\begin{itemize}[leftmargin=.75in, rightmargin=.75in]
	\item[$\xi$:] the half length of the corridor,
	\item[$\omega_i$:] the angle between the rhombus and central flag, 
	\item[$\gamma_i$:] The angle formed between the horizontal plane that is $\sqrt{3}$ above $O_i$ and $O_{i+1}$ when rotated to the maximal possible extent before causing possible collision with other flags. 
	\item[$\zeta_i$:]  is the length from the hinge point of $O_{i+1}$ and the rhombus to the point that vertical to corner of $O_i$ that is to the right over the central flag on the horizontal plane that is $\sqrt{3}$ above $O_i$.
	\item[$\phi_i$:] is the angle formed between the central flag and $O_i$.
\end{itemize}

In Figure \ref{fig:dualSmallHexagonalGrid.pdf}(a) we see $\ell$ in the center of the column of obstacle hexagons.  
Our goal is to show that the column of hexagons cannot tilt in the manner shown in Figure \ref{fig:tiltedObstaclesInFrame.pdf} where the column veers greatly into the space occupied by other corridors and obstacle hexagons.
The cross section of an arbitrary corridor must have a height of at least $\sqrt{3}$ everywhere. 
%noncanonical corridor must be at least 
Otherwise, a flag would overlap with an obstacle hexagon; it would no longer remain a realization since the height of a flag is $\sqrt{3}$.
In Figure \ref{fig:corridorNonCanonical2.pdf}, we illustrate an obstacle hexagon, its upper corridor with the central flag rotated counterclockwise $\frac{\pi}{6}$ radians that has a hinge to the skinny rhombus in a vertical position.  
The skinny rhombus has length $\sqrt{1 + \lr{100N}^{-2}}$.
The rhombus is hinged at the midpoint of the upper side of the corridor.
The length from a corridor's midpoint to one end of the corridor is $\frac{5t-1}{4}$.
$\gamma_i$ is the angle between $\zeta_i$ and the horizontal axis at the height of the flag ($i = 1,2,\ldots, u$).
The bound of $\gamma_i$ is:
\begin{equation}\label{eqn:alphaBound}
\begin{array}{rcl}
\gamma_i & \leq & \tan^{-1} \lr{
								\frac{\lr{2 - \sqrt{3}} + \sqrt{1 + \lr{\frac{1}{100N}}^2}	}
									 {	\frac{5t -1}{4}	}
								}\\
& \leq & \frac{
				4 \lr{2 - \sqrt{3}} + 4\sqrt{1 + \lr{\frac{1}{100N}}^2}	}
			  {	
			  	5t -1	
			  }\\
& \leq & \frac{
				\frac{4}{3} + \sqrt{16 + 9}	}
			  {	
			  	5t -1	
			  } \\

& \leq & \frac{\frac{19}{3}}{5t -1} \\
& \leq & \frac{\frac{24}{3}}{4s^\kappa} \\
&\leq& \frac{2}{s^\kappa}
\end{array} 
\end{equation}
Inequality \ref{eqn:alphaBound} uses the first term Maclaurin series of $\tan^{-1}$ (for reference, see table of Maclaurin series equations at the beginning of this chapter) and holds for sufficiently large $s$ and $\kappa$.  
Thus the relative rotational difference between adjacent obstacle hexagons is:
\begin{equation}\label{eqn:angularBound}
\left\vert \alpha_i - \alpha_{i+1} \right\vert \leq\frac{2}{s^\kappa}
\end{equation}
The relative difference between $\alpha_i$ and $\alpha_{i+1}$ is small.
The bottom most obstacle hexagon is hinged to the frame (see Figure \ref{fig:HalfSizeHexagon.pdf} for illustration). 
This implies that $\alpha_1 = 0$ and
$$\vlr{\alpha_1 - \alpha_2}=\vlr{\alpha_2}\leq \frac{2}{s^\kappa}.$$
There are a total of $u$ obstacle hexagons in a column with possibly up to $u-1$ nonzero obstcle hexagons rotations.
We can derive 1) a bounded sum of rotational displacement over a column of obstacle hexagons:
\begin{equation}\label{eqn:angularSumBound}
\sum_{i=1}^{u-1} \vert \alpha_i - \alpha_{i+1} \vert \leq \frac{2(12s-1)}{s^\kappa} \leq \frac{24}{s^{\kappa-1}}
\end{equation}
and 2) derive the maximum rotational displacement at the $\ith$ obstacle hexagon:
\begin{eqnarray*}
\alpha_i &\leq& \frac{2}{s^\kappa} \sum_{j=1}^i j\\
		 &\leq& \frac{2}{s^\kappa}  \frac{i^2+i}{2}\\
		 &\leq& \frac{2}{s^\kappa}  i^2\\
		 &\leq& \frac{2u^2}{s^\kappa}\\
		 &\leq& \frac{288s^2}{s^{\kappa}}\\
		 &\leq& \frac{288}{s^{\kappa-2}} \\
\end{eqnarray*}
For any $i$, the bound for $\alpha_i$:
\begin{equation}\label{eqn:angularMaxBound}
\alpha_i \leq \frac{288}{s^{\kappa-2}}
\end{equation}

% The sum of total displacement in a given column is bounded by:
% $$ \vert \alpha_u \vert \leq \sum_{i=2}^{u-1} \vert \alpha_i - \alpha_{i+1} \vert \leq \frac{u-1}{2\lr{N^3 - 1}}= \frac{12s - 1}{2 \lr{ (c_N s)^3 - 1}}$$
