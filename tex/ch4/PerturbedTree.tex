\subsection{Perturbed Weighted Trees $T$}
In a perfect snowflake there are contacts in the disk arrangement that do not reflect the corresponding edge set of the tree.  
The contact graph is not a tree.
By increasing the weight of the center disk corresponding to $v_0$, the disks around the central disk will have additional degrees of freedom and removes unintended contacts.    
This claim is shown as a result of Lemma \ref{lem:hingedPolygon27} later.
We refer to the positions of the disks of $S_i$ as the \textit{canonical arrangement} of $S_i$.

Given $\epsilon > 0$ and $x >0$, we define $T$ as follows: the tree $T_i$ with weight $\frac{\epsilon}{4}$ for every vertex except $v_0$; $v_0$ has a weight $\frac{\epsilon}{4} + \zeta$ for some $\zeta>0$ that is specified later.  For $T_i$, let 
$$\begin{array}{cc}
i = \ceil{\frac{4x}{\epsilon}}&r = \frac{\epsilon}{4}
\end{array} .$$ 
A perturbed weighted tree $T$ can be realized as an ordered disk contact graph (a disk arrangement).  
A perturbed snowflake realization has some distinct qualities from the perfect snowflake ordered contact graph of $S_i$.    
The angular relationships between adjacent vertices may vary.

\textbf{Modification of $\sigma_1$.}
We show that for any realization of the ordered tree $T$
 the placement of vertices is close to canonical position.  
In order to show this, we show the components of a perturbed snowflake in arbitrary position  are close to canonical position.  
The argument comprises of three parts: (1) Showing that the pertubation of the central disk and the six neighboring disks is small, (2) show that the displacement along the stems for all $S_i$ is small, and (3) show that the displacement along the petioles is small.  

Given a instance of a perturbed snowflake with $v_0$ having weight $\epsilon + \zeta$ where $\epsilon > 0$, vertices neighboring $v_0$ each have a range of placement on the plane when realizaed as a disk arrangement. 
Figure \ref{fig:modifiedContactGraph.pdf} shows a realization of $S_1$ and illustrates one such example of possible gaps, $\zeta$, that could be created between adjacent disks of $S_1$ in a perfect snowflake.



  
 