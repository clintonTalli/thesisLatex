\subsection{Perturbed Weighted Trees $T_\epsilon$}
In a perfect snowflake there are contacts in the disk arrangement that do not reflect the the corresponding edge set of the tree.  
The contact graph is not a tree.
By modifying the weight of the center disk corresponding to $v_0$ to differ from all others weights in a snowflake, the disks around the central disk will have additional area of realization and removes the issue of having unintended contacts that do not reflect a give tree's edge relations.  
This claim is shown as a result of Lemma \ref{lem:hingedPolygon27} later.
We refer to the positions of the disks of $S_i$ as the \textit{canonical arrangement} of $S_i$.

Given $\epsilon > 0$ and $x >0$, we define $T_\epsilon$ as follows: the tree $T_i$ with weight $\epsilon$ for every vertex except $v_0$; $v_0$ has a weight $\epsilon + \zeta(\epsilon)$ for some $\zeta(\epsilon)>0$ that is specified later.  For $T_i$, let $$i = \ceil{\frac{x}{\epsilon}} .$$ 
A perturbed weighted tree $T_\epsilon$ can be realized as a disk touching graph (a disk arrangement).  
A perturbed snowflake realization has some distinct qualities from perfect snowflake realizations.  
The angular relationships between adjacent vertices may vary.

\textbf{Modification of $S_1$.}
We show that for any realization of the ordered tree $T_\epsilon$
 the placement of vertices is close to canonical position.  
In order to show this, we show the components of a perturbed snowflake in arbitrary position  are close to canonical position.  
The argument comprises of three parts: (1) Showing that the pertubation of the central disk and the six neighboring disks is small, (2) show that the displacement along the arms for all $S_i$ for $i \geq 1$ is small, and (3) show that the displacement along the petioles is small.  

Given a instance of a perturbed snowflake with $v_0$ having weight $\epsilon + \zeta(\epsilon)$ where $\epsilon > 0$, vertices neighboring $v_0$ each have a range of placement on the plane when realizaed as a disk arrangement. 
Figure \ref{fig:modifiedContactGraph} shows a realization of $S_1$ and illustrates one such example of possible gaps, $\zeta(\epsilon)$, that could be created between adjacent disks of $S_1$ in a perfect snowflake.  
 