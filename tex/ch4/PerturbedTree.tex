
\subsection{Perturbed Weighted Trees $T$}
In a perfect snowflake there are contacts in the disk arrangement that do not reflect the corresponding edge set of the tree.  
The contact graph is not a tree.
By increasing the weight of the center disk corresponding to $v_0$, the disks around the central disk will have additional degrees of freedom and removes unintended contacts.    
This claim is shown as a result of Lemma \ref{lem:hingedPolygon27} later.
We refer to the positions of the disks of $S_i$ as the \textit{canonical arrangement} of $S_i$.

Given $\epsilon > 0$ and $x >0$, we define $T$ as follows: the tree $T_i$ with weight $\frac{\epsilon}{10}$ for every vertex except $v_0$; $v_0$ has a weight $\frac{\epsilon}{10} + \zeta$ for some $\zeta>0$ that is specified later.  For $T_i$, let 
$$\begin{array}{cc}
r = \frac{\epsilon}{10}&i = \ceil{\frac{10x}{\epsilon}}
\end{array} .$$ 
A perturbed weighted tree $T$ can be realized as an ordered disk contact graph (a disk arrangement).  
A perturbed snowflake realization has some distinct qualities from the perfect snowflake ordered contact graph of $S_i$.    
The angular relationships between adjacent vertices may vary.
