
\subsection{Perturbed Weighted Trees $T_\epsilon$}

A perturbed weighted tree $T_\epsilon$ is a weighted unit tree with unit weight on every vertex with the exception of the root vertex having weight $r+\epsilon$ where $\epsilon>0$ can be realized as a disk touching graph (a disk arrangement) and $r$ is the unit length.  

The perturbed snowflake follows the construction of the perfect snowflake with the exception of $v_0$ having weight $r+ \epsilon$ where $\epsilon > 0$.
A perturbed snowflake realization has some distinct qualities from perfect snowflake realizations.  
The angular relationships between adjacent vertices may vary; the distance between adjacent and neighboring vertices may vary as well.


In general, the perturbation $\epsilon$ can modify the realization of a perfect snowflake $S_i$ in the following ways:

\textbf{Modification of $S_1$.} 

Given a instance of a perturbed snowflake with $v_0$ having weight $\frac{1}{2} + \epsilon$ where $\epsilon > 0$, vertices neighboring $v_0$ each have a range of placement on the plane when realizaed as a disk arrangement. 
Figure \ref{fig:modifiedContactGraph} shows a realization of $S_1$ and illustrates one such example of possible gaps, $\epsilon$, that could be created between adjacent disks of $S_1$ in a perfect snowflake.  

Note that (1) the adjacent disks in a perfect snowflake may or may not be adjacent in a given perturbed snowflake of $S_1$ and (2) $S_1 \subseteq S_i$ for any $i \in \bbN$.  