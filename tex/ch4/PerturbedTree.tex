\subsection{Perturbed Weighted Trees $T_\epsilon$}
Given $\epsilon > 0$, we define $T_\epsilon$ as follows: the tree $T_i$ with weight $\epsilon$ for every vertex except $v_0$; $v_0$ has a weight $\epsilon + \zeta(\epsilon)$ for some $\zeta(\epsilon)>0$ that is specified later.
A perturbed weighted tree $T_\epsilon$ can be realized as a disk touching graph (a disk arrangement).  
% The perturbed snowflake follows the construction of the perfect snowflake with the exception of $v_0$ having weight $\epsilon+ \zeta(\epsilon)$ where $\epsilon > 0$.
A perturbed snowflake realization has some distinct qualities from perfect snowflake realizations.  
The angular relationships between adjacent vertices may vary, the distance between adjacent and neighboring vertices may vary as well.

\textbf{Modification of $S_1$.}
We will show for any $\epsilon >0$ and arbitrary position of vertices, the placement of vertices is close to canonical position.  
In order to show this, we show the components of a perturbed snowflake in arbitrary position  are close to canonical position.  
The argument comprises of three parts: (1) Showing that the pertubation of $S_1$ is small, (2) show that the displacement along the arms for all $S_i$ for $i \geq 1$ is small, and (3) show that the displacement along the petioles is small.  

Given a instance of a perturbed snowflake with $v_0$ having weight $\epsilon + \zeta(\epsilon)$ where $\epsilon > 0$, vertices neighboring $v_0$ each have a range of placement on the plane when realizaed as a disk arrangement. 
Figure \ref{fig:modifiedContactGraph} shows a realization of $S_1$ and illustrates one such example of possible gaps, $\zeta(\epsilon)$, that could be created between adjacent disks of $S_1$ in a perfect snowflake.  
