\chapter{Realizability Problems for Weighted Trees}\label{chp:disk}

We begin this chapter with the preliminary concepts to solving Problems \ref{problem:UnorderedContactGraph} and \ref{problem:OrderedContactGraph}, and then prove related Theorem \ref{thm:disk}.  
Recall the Unordered and Ordered Realizability Problem for a contact graph  and corresponding theorem which state:

\begin{itemize}
\item[\textbf{Problem \ref{problem:UnorderedContactGraph}}] Given a planar graph with positive weighted vertices, is it a contact graph of some disk arrangement where the radii equal the vertex weights?
\item[\textbf{Problem \ref{problem:OrderedContactGraph}}] Given a planar graph with positive weighted vertices and a combinatorial embedding, is it a contact graph of some disk arrangement where the radii equal the vertex weights and the counter-clockwise order of neighbors of each disk is specified by the combinatorial embedding?
\item[\textbf{Theorem \ref{thm:disk}}] It is NP-Hard to decide whether a given tree with positive vertex weights
is the contact graph of a disk arrangements with specified radii.
\end{itemize}

The preliminary concepts for Disk Arrangements are the Unit Disk Touching Graph Recongnition Problem, the Perturbed Root with Unit Disk Leaves Touching Graph Recongnition Problem, and Hausdorff distance.  
All together the proliminary concepts will allow us to approximate the geometry of polygons and allow us to use the results of the polygonal linkages in Chapter \ref{chapter:polygonalLinkage}.
