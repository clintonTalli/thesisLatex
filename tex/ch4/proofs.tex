\textbf{Modification of $\sigma_1$.}
We show that for any realization of the ordered tree $T$
 the placement of vertices is close to canonical position.  
In order to show this, we show the components of a perturbed snowflake in arbitrary position  are close to canonical position.  
The argument comprises of three parts: (1) Showing that the pertubation of the central disk and the six neighboring disks is small, (2) show that the displacement along the stems for all $S_i$ is small, and (3) show that the displacement along the petioles is small.  

Given a instance of a perturbed snowflake with $v_0$ having weight $\epsilon + \zeta$ where $\epsilon > 0$, vertices neighboring $v_0$ each have a range of placement on the plane when realizaed as a disk arrangement. 
Figure \ref{fig:modifiedContactGraph.pdf} shows a realization of $S_1$ and illustrates one such example of possible gaps, $\zeta$, that could be created between adjacent disks of $S_1$ in a perfect snowflake.

\paragraph{Displacement on $\sigma_1$ is small.}
Note that (1) the adjacent disks in a perfect snowflake cannot be adjacent in a given perturbed snowflake of $S_1$ and (2) $S_1 \subseteq S_i$ for any $i \in \bbN$ because every contact is encoded in the tree.  
Given a snowflake in arbirary position with $i$ nodes per stem; the edge length of each segment is $2r$ except the edges having node $v_0$, these edges are length $2r+ \zeta$.  
The stem has a length of at most $2ri+\zeta$.
Figure \ref{fig:LineSegmentDelta.pdf} shows an stem of a tree in arbitrary position corresponds to a compression and shift of vertices.  
The stem realized in arbitrary position in Figure \ref{fig:LineSegmentDelta.pdf} is analgous to a tree realized in arbitrary position where vertices are in a different position than canonical.  
Our goal is to show that for any $\epsilon >0$ and $x >0$, the position of the center $v$ of any disk in the disk arrangement in any arbitrary realization corresponding to $T$ has a small displacement, i.e. $v$ is found in an open ball referenced at the canonical position of $v$, $v_c$: $$v \in b_{\psi_1}(v_c).$$ 
The definition of $\psi_1$ is shown later on.

\begin{minipage}{\linewidth}
\begin{center}
\includegraphics[width=.95\columnwidth]{graphics/LineSegmentDelta.pdf}
\captionof{figure}{The polyline at the bottom represents a snowflake stem in canonical position.  The polyline above represents a snowflake stem in non-canonical position.}\label{fig:LineSegmentDelta.pdf}
\end{center}
\end{minipage}
 
In $S_1$ the six disks around the central disk kiss.  
The angle formed from the center of the central disk to the centers of any two adjacent disks is $\frac{\pi}{3}$.  
The side lengths of the equalateral triangle formed by the centers of three adjacent disks, one of which is the central disk, is $\frac{\epsilon}{5}$.  
For a perturbed $S_1$, the the central disk is weighted $\frac{\epsilon}{10} + \zeta$.  
This can yield a change of angular displacement $\frac{\pi}{3}$ to $\frac{\pi}{3} \pm 2\chi$.  
To find the bounds of how large or small $\chi$ can be, we show a trigonometric relation of the half angle of the triangle corresponding to three adjacent disks (See Figure \ref{fig:part1ch4.pdf}).

\begin{minipage}{\linewidth}
\begin{center}
\includegraphics[width=.20\columnwidth]{graphics/part1ch4.pdf}
\captionof{figure}{This figure depicts a triangle corresponding to the center of the central disk and two adjacent disks.}\label{fig:part1ch4.pdf}
\end{center}
\end{minipage}

\begin{eqnarray*}
\sin \lr{\frac{\pi}{6} - \chi} &=& \frac{\frac{\epsilon}{10}}{\frac{\epsilon}{5}+\zeta}\\
\frac{1}{2}\cos \chi = \sin \frac{\pi}{6} \cos \chi &=& \frac{\frac{\epsilon}{10}}{\frac{\epsilon}{5}+\zeta} + \cos \frac{\pi}{6} \sin \chi = \frac{\frac{\epsilon}{10}}{\frac{\epsilon}{5}+\zeta} +\frac{\sqrt{3}}{2} \sin \chi \\
&\Rightarrow&\\
\frac{\frac{\epsilon}{10}}{\frac{\epsilon}{5}+\zeta}+ \frac{\sqrt{3}}{2} \lr{ \chi - \frac{\chi^3}{6}} &\leq& \frac{\frac{\epsilon}{10}}{\frac{\epsilon}{5}+\zeta} + \frac{\sqrt{3}}{2} \sin \chi =\frac{1}{2} \cos \chi\\
&\Rightarrow&\\
\frac{\sqrt{3}}{2} \lr{ \chi - \frac{\chi^3}{6}}&\leq& \frac{1}{2}-\frac{\frac{\epsilon}{10}}{\frac{\epsilon}{5}+\zeta}  \qquad \text{if }\chi < 1 \\
\frac{5\sqrt{3}}{12} \chi &\leq& \frac{  \zeta }{ \frac{2\epsilon}{5} + 2\zeta } \\
\chi&\leq& \frac{12}{5\sqrt{3}}\frac{  \zeta }{ \frac{2\epsilon}{5} + 2\zeta } = \frac{6 \zeta}{\lr{\epsilon + 5 \zeta}\sqrt{3}}
\end{eqnarray*}
For any $\epsilon > 0$, the bounds for angular displacement formed at the center of the central disk and two adjacent disks is:
$$ \frac{\pi}{3} - \frac{12 \zeta}{\lr{\epsilon + 5 \zeta}\sqrt{3}}\leq \frac{\pi}{3} - 2 \chi  \leq 2\chi \leq \frac{\pi}{3} + 2 \chi \leq \frac{\pi}{3} + \frac{12 \zeta}{\lr{\epsilon + 5 \zeta}\sqrt{3}}$$
This shows the bounds of the angular displacement around the central disk.  
To translate this to a coordinate displacement of the centers of the disks around the central disk, we compute a the radius of the ball of radius $\psi$ in which the centers may lie. 
$$\psi = \lr{2 r + \zeta} \sin \chi$$
Using the Maclaurin series of $\sin \chi$:
$$
\begin{array}{rcl}
\psi_i &=& \lr{ \frac{\epsilon}{5} + \zeta} \sin \chi\\
&\leq& \lr{ \frac{\epsilon}{5} + \zeta} \frac{6 \zeta}{\lr{\epsilon + 5 \zeta}\sqrt{3}}\\
&=& \frac{6\zeta}{5 \sqrt{3}}
\end{array}
$$
The coordinates of the centers of the disks are close to cannonical position following the angular displacement argument above.  
That is, if $u$ is a center of a disk in a realization to $T$, $u$ lies in a ball $b_{\psi_1}\lr{u_c}$ where $u_c$ is the cannonical position of the $u$.  

\paragraph{Displacement on the stems is small.}
Without loss of generality, consider a path of vertices along a distinct stem, petiole, or leaf $\lr{v_1, v_2, \ldots, v_n}$.  
In the case that there are two paths attached to each vertex, four different angles about the vertex are formed.  
For the $\ith$ vertex, the counter clockwise order of the angles are $\lr{\alpha_{i+1}, \gamma_i, \delta_i, \beta_{i+1}}$ (See Figure \ref{fig:ch4Paralellogram.pdf} for reference).  
We want to establish an angular relationship between two consecutive vertices in a path.
\begin{lem}\label{lem:hingedPolygon27}
Let $\lr{a,b,c,d}$ be a polygonal path in the plane such that the unit disks centered at $a$, $b$, $c$, and $d$ are interior-disjoint.  Then the sum of the two angles on each side at the two interioir vertices is greater than $\pi$. Then the sum of the two angles on each side at the two interior vertices is greater than $\pi$.
\end{lem}
\begin{proof}
Without loss of generality, consider the two angles on the left side  at the two interioir vertices, $\angle abc$ and $\angle bcd$.  We have $\vlr{ab} = \vlr{bc} = \vlr{cd} = 2$, since we have a disk arrangement of unit disks.  If $\lr{a,b,c,d}$ is a rhombus, then $\vlr{ad}=2$ and $\angle abc + \angle bcd = \pi$.  Hence $\vlr{ad}>2$ implies $\angle abc + \angle bcd > \pi$. 
\end{proof}

\begin{minipage}{\linewidth}
\begin{center}
\includegraphics[width=.75\columnwidth]{graphics/lemmaHingedPolygon2.pdf}
\captionof{figure}{(a)Four disks centered along a polygonal path $\lr{a,b,c,d}$ in various drawings.}\label{fig:lemmaHingedPolygon2.pdf}
\end{center}
\end{minipage}

We can now say that for any two consecutive vertices $\lr{v_i, v_{i+1}}$ along a path, the following angular relationship from Lemma \ref{lem:hingedPolygon27}:
$$ 
\begin{array}{rcl}
\pi &<& \alpha_i + \gamma_i \\
\pi &<& \beta_i + \delta_i
\end{array}
$$
This result shows that a perturbed snowflake removes the issue of having unintended constacts that do not reflect a give tree's edge relations that a perfect snowflake has.  

In Figure \ref{fig:ch4Paralellogram.pdf}(a), we have a parallelogram with four unit disks centered, one on each of the vertices of the parallelogram.  
If $\alpha_i + \gamma_{i+1} < \pi$, One of the disks will overlap with another.  
For any arbitrary realization of a perturbed snowflake, any two consecutive angles formed between two adjacent vertices along a path must satisfy the following constraint $$\alpha_i + \gamma_{i+1} \geq \pi.$$

\begin{minipage}{\linewidth}
\begin{center}
\includegraphics[width=.75\columnwidth]{graphics/ch4Paralellogram.pdf}
\captionof{figure}{(a)Four disks of the snowflake shown where the top two disks can either be leaves off petioles or off the stems. (b)An stem depicted at the $\ith$ and $(i+1)^\text{st}$ vertex.}\label{fig:ch4Paralellogram.pdf}
\end{center}
\end{minipage}

To show that the angluar displacement along the stem is small, we extend the angular argument on the perturbed $S_1$ and by induction, show that it is small for all $i$.  
Denote the angles on the concave side of the $\ith$ vertex as $\alpha_i$ and $\beta_i$ and the convex side of the $(i+1)^\text{st}$ vertex as $\gamma_i$ and $\delta_i$ respectively (see Figure \ref{fig:ch4Paralellogram.pdf}(b) for reference). 
For any vertex, the sum of angles about the vertex is $2 \pi$, e.g.:
$$\gamma_i + \delta_i + \alpha_{i+1} + \beta_{i+1} = 2 \pi$$ 

Suppose we numbered the disks about the central disk 1 through 6.  
Without loss of generality, the angles $\alpha_0$ and $\beta_0$ correspond to the angles formed between the central angle, disks $i$ and $i+1$ and disks $i+1$ and $i+2$ respectively, for $i = 1,2,3$.  
The bounds for $\alpha_0$ and $\beta_0$ are the same as $2\chi$ in the earlier argument, i.e.:
$$
\begin{array}{rcccl}
\frac{\pi}{3} - \frac{12 \zeta}{\lr{\epsilon + 5 \zeta}\sqrt{3}} \leq \frac{\pi}{3} - 2 \chi &\leq& \alpha_0 &\leq& \frac{\pi}{3} + 2 \chi \leq \frac{\pi}{3} + \frac{12 \zeta}{\lr{\epsilon + 5 \zeta}\sqrt{3}}\\
\frac{\pi}{3} - \frac{12 \zeta}{\lr{\epsilon + 5 \zeta}\sqrt{3}}\leq \frac{\pi}{3} - 2 \chi &\leq& \beta_0  &\leq& \frac{\pi}{3} + 2 \chi \leq \frac{\pi}{3} + \frac{12 \zeta}{\lr{\epsilon + 5 \zeta}\sqrt{3}}\\
\end{array}
$$

For $i=0$ we know that $\alpha_0 + \beta_0 \leq \frac{2\pi}{3} + \frac{48 \zeta}{5\sqrt{3}}.$
For the inductive step, suppose the following:
$$
\begin{array}{rcl}
\alpha_i +\beta_i &\leq& \frac{2 \pi}{3} + \frac{24 \zeta}{\lr{\epsilon + 5 \zeta}\sqrt{3}}\\
\pi &\leq& \alpha_i + \gamma_i \\
\pi &\leq& \beta_i + \delta_i
\end{array}
$$ 

$$
\begin{array}{rcl}
\alpha_i +\beta_i &\leq& \frac{2 \pi}{3} + \frac{24 \zeta}{\lr{\epsilon + 5 \zeta}\sqrt{3}}\\
\pi &\leq& \alpha_i + \gamma_i \\
\pi &\leq& \beta_i + \delta_i
\end{array}
$$

Together, we have the following result:
\begin{eqnarray*}
2\pi &=& \alpha_i + \gamma_i + \beta_i + \delta_i\\
 &=& \alpha_i + \gamma_i + \lr{2\pi - \alpha_{i+1} - \beta_{i+1}}\\
 &\iff&\\
\alpha_{i+1} + \beta_{i+1}&=&\alpha_i + \gamma_i \\
&\leq& \frac{2\pi}{3} + \frac{24 \zeta}{\lr{\epsilon + 5 \zeta}\sqrt{3}}\\
\end{eqnarray*}
And so the error bounds on $\frac{\pi}{3}+2\chi$ hold in general for $\alpha_i$ and $\beta_i$ for all $i$.  
$$\alpha_i + \beta_i \leq \frac{2\pi}{3} + \frac{24 \zeta}{\lr{\epsilon + 5 \zeta}\sqrt{3}}$$

\begin{minipage}{\linewidth}
\begin{center}
\includegraphics[width=.75\columnwidth]{graphics/psiSegmentChange.pdf}
\captionof{figure}{This figure illustrates the change of position of the centers of a disk in an arbitrary positin and in a straight, canoncial position. }\label{fig:psiSegmentChange.pdf}
\end{center}
\end{minipage}

A perturbed snowfloke $\sigma_i$ is a realization of $T$.  
Let $v$ be a vertex in $T$.  
There exists a unique path consisting of $j$ edges from $v_0$ to $v$.  
We can compute a bound of coordinate displacement $\psi_i$.  
We want to establish a bound of coordinate displacement for every vertex in $T$.  
We extablish a bound for the path from $v_0$ to $v$ by induction on $j$, the number of edges on the path from $v_0$ to $v$.  
The base case is $v_0$ which the displacement is zero suppose $\psi_j$ is true.
% A perturbed snowflake is a realization of some tree $T$.  
% For the $\jth$ vertex along some path from $v_0$ to $v_j$ (see Figure \ref{fig:psiSegmentChange.pdf} for reference), we can compute the bound of coordinate displacement $\psi_i$ of the disk corresponding to $v_j$ using our inductive argument on the angular displacement above.  
% Without loss of generality, to translate the $\jth$ coordinate displacement, we compute the radius of the ball of radius $\psi_i$ in which the $\jth$ center may lie. 
$$\psi_j = (2j-2) \lr{\frac{\epsilon}{10} + \zeta} \sin \chi$$
Let $\zeta \leq \frac{\epsilon^2-\epsilon}{2}$.  
Using the Maclaurin series of $\sin \chi$:
$$
\begin{array}{rcl}
\psi_j &=& (2j-2) \lr{\frac{\epsilon}{10} + \zeta} \sin \chi\\
&\leq& (2j-2) \lr{\frac{\epsilon}{10} + \zeta}  \frac{6 \zeta}{\lr{\epsilon + 5 \zeta}\sqrt{3}}\\
&\leq& \lr{\frac{(j-1)\epsilon}{5} + \epsilon} \frac{6 \zeta}{\epsilon\sqrt{3}}\\
&\leq& \frac{6(i-1)\zeta + 30\zeta}{5\sqrt{3}}= \frac{\zeta(6i-29)}{5 \sqrt{3}}
\end{array}
$$

$\phi_i$ bound is true for almost all vertices but the outermost leaves of the petioles with additional freedom of movement (see Figure \ref{fig:hexagonPetiolesLeafs9LayersRotatedOutward.pdf} for reference).  
By choosing $\zeta = \frac{5\epsilon  \sqrt{3}}{(6i-29)}$, we have that $\zeta$ is polynomial in $x$ and $\epsilon$ since $i$ is polynomial in $x$ and $\epsilon$.
\paragraph{Displacement on the petioles is small.}
Note that the petioles have the same geometric structure as the stems; the exception is the number of leaves on each side of the petioles. 
Since we've shown that the geometric shape in arbirary position is already close to canonical position for any $\epsilon>0$, the same argument applies here for the petioles.

We have shown the displacements of all components of the perturbed snowflake are small for any $\epsilon > 0$.  
This shows that the structure has stability in preserving any information encoded with it.