\section{Approximating Regular Hexagons with Snowflakes}
In figure \ref{fig:hexagonOutline5LayerSmall.pdf}, we have a set of unit radius disks (circles) arranged in a manner that outlines regular, concentric hexagons.
\begin{figure}[!htbp]
\begin{center}
\includegraphics{graphics/hexagonOutline5LayerSmall.pdf}
\caption{A contact graph that resembles the shape of concentric hexagons.}\label{fig:hexagonOutline5LayerSmall.pdf}
\end{center}
\end{figure}
\begin{prob}[Appoximating Polygonal Shapes with Contact Graphs]\label{problem:ApproxShapesWithContactGraphs}
For every $\epsilon >0$ and polygon $P$, there exists a contact graph $G = (V,E)$  such that the Hausdorff distance $d(P,G) < \epsilon$
\end{prob}

Recall problems (\ref{problem:UnorderedContactGraph}) and (\ref{problem:OrderedContactGraph}): given a positive weighted tree, $T$, is $T$ the (ordered) contact graph of some disk arrangement where the radii are equal to the vertex weights.  For now, we'll focus on a particular family of this problem space where the weighted trees can be realized as a \textit{snowflake}. For $i \in \bbN$, the construction of the snowflake tree, $S_i$, is as follows:
\begin{itemize}
\item Let $v_0$ be a vertex that has six paths attached to it: $p_1$, $p_2$, $\dots$, $p_6$.  Each path has $i$ vertices.
\item For every other path $p_1$, $p_3$, and $p_5$: 
	\begin{itemize}
		\item 	Each vertex on that path has two paths attached, one path on each side of $p_k$.
		\item	The number of vertices that lie on the path attached to the $j^\text{th}$ vertex of $p_k$ is $i-j$.
	\end{itemize}
\end{itemize}
\begin{figure}[!htbp]
\begin{center}
\includegraphics{graphics/snowflakeOutline5LayerSmall.pdf}
\caption{The same contact graph as in figure \ref{fig:hexagonOutline5LayerSmall.pdf} overlayed with the a perfectly weighted snowflake tree.}\label{fig:snowflakeOutline5LayerSmall.pdf}
\end{center}
\end{figure}

A \textit{perfectly weighted snowflake tree} is a snowflake tree with all vertices having weight $\frac{1}{2}$.  A \textit{perturbed snowflake tree} is a snowflake tree with all vertices having weight of 1 with the exception of $v_0$;  in a perturbed snowflake tree, $v_0$ will have a weight of $\frac{1}{2} + \gamma$.  For our analysis, all realizations of any snowflake, perfect or perturbed, shall have $v_0$ fixed at origin.  This is said to be the canonical position under Hausdorff distance of the snowflake tree.   

\paragraph{Perfectly Weighted Snowflake Tree.}

Consider the graph of the triangular lattice with unit distant edges:
\begin{eqnarray*}
V &=& \left\lbrace a\cdot (1,0) + b \cdot \left(\frac{1}{2},\frac{\sqrt{3}}{2}\right) : a,b \in \bbZ \right\rbrace\\
E &=& \left\lbrace \left\lbrace u,v \right\rbrace : \vert\vert u-v \vert\vert = 1 \text{ and } u,v \in V\right\rbrace
\end{eqnarray*}
The following graph, $G=(V,E)$ is said to be the \textit{unit distance graph} of the triangular lattice.  We can show that no two distinct edges of this graph are non-crossing.  First suppose that there were two distinct edges that crossed, $\left\lbrace u_1,v_1 \right\rbrace $ and $\left\lbrace u_2,v_2 \right\rbrace$.  With respect to $u_1$, there are 6 possible edges corresponding to it, with each edge $\frac{\pi}{3}$ radians away from the next.  Neither edge crosses another; and so we have a contradiction that there are no edge crossings with $\left\lbrace u_1,v_1 \right\rbrace $.  


The perfectly weighted snowflake tree that is a subgraph over the \textit{unit distance graph}, $G=(V,E)$, of the triangular lattice.  To show this, for any $S_i$, fix $v_0 = 0 \cdot \cdot (1,0) + 0 \cdot \left(\frac{1}{2},\frac{\sqrt{3}}{2}\right)=\lr{0,0} \in V$ at origin.  Next consider the six paths attached from origin.  Fix each consecutive path $\frac{\pi}{3}$ radians away from the next such that the following points like on the corresponding paths: $\lr{1,0} \in p_1, \lr{\frac{1}{2} ,\frac{\sqrt{2}}{3}} \in p_2,\lr{-\frac{1}{2}\p_4,\frac{\sqrt{3}}{2}} \in p_3, \lr{-1,0} \in p4, \lr{-\frac{1}{2},-\frac{\sqrt{3}}{2}}\in p_5,\lr{\frac{1}{2},-\frac{\sqrt{3}}{2}}\in p_6$.  For $S_i$, there are $i$ vertices on each path.  

We define the six paths from origin as follows:
\begin{eqnarray*}
p_1 &=& \set{a\cdot\lr{1,0} = \vec{v}}{a \in \bbR^+}\\
p_2 &=& \set{a\cdot\lr{\frac{1}{2},\frac{\sqrt{3}}{2}} = \vec{v}}{a \in \bbR^+}\\
p_3 &=& \set{-a\cdot \lr{1,0} + a \cdot \lr{\frac{1}{2},\frac{\sqrt{3}}{2}} = a\lr{-\frac{1}{2},\frac{\sqrt{3}}{2}} = \vec{v}}{a \in \bbR^+}\\
p_4 &=& \set{a \cdot \lr{-1,0} = \vec{v}}{a \in \bbR^+}\\
p_5 &=& \set{a \cdot \lr{-\frac{1}{2},-\frac{\sqrt{3}}{2}}  = \vec{v}}{a \in \bbR^+}\\
p_6 &=& \set{ a\cdot \lr{1,0} - a \cdot \lr{\frac{1}{2},\frac{\sqrt{3}}{2}}= a \cdot \lr{\frac{1}{2}, -\frac{\sqrt{3}}{2}}}{a \in \bbR^+} 
\end{eqnarray*}
For $S_i$ there exists $i$ vertices on each path.  We shall denote the $i^\text{th}$ vertex on the $j^\text{th}$ path as $v_{j,i}$.  For each path defined above, the paths are defined as a set of vectors, $\vec{v} = a \cdot \vec{p}$  for some $a \in \bbR^+$ and $\vec{p} \in \bbR^2$.  By setting $a = 1,2,\dots, i$, we obtain points that are contained in $V$.  For $j = 1,3,5$ and $l = 1,..., i$, there exists two paths attached to each vertex $v_{j,l}$.  For $S_i$, each path attached to the $k^\text{th}$ vertex of $p_j$, there are $i-k$ vertices.  We will need to show that each of the $i-k$ vertices on each corresponding path are also in $V$.

The triangular lattice is symmetice under rotation about $v_0$ by $\frac{\pi}{3}$ radians.  For each vertex $v_{1,l}$ for $l=1,2,..., i-k$, we place two paths from it; the first path $\frac{\pi}{3}$ above $p_1$ at $v_{1,l}$ and $\frac{-\pi}{3}$ below $p_1$ at $v_{1,l}$ and call these paths $p_{1,l}^+$ and $p_{1,l}^-$ respectively.  With respect to $v_{1,l}$, one unit along $p_{1,l}^+$ is a point on the triangular lattice and similarly so on $p_{1,l}^-$.  Continuing the walk along these paths, unit distance-by-unit distance, we obtain the next point corresponding point on the the triangular lattice up to $i-k$ distance away from $v_{1,l}$.  This shows that each of the $i-k$ vertices on $p_{1,l}^-$ and $p_{1,l}^+$ are in $V$.  By rotating all of the paths along $p_1$ by $\frac{2\pi}{3}$ and $\frac{4\pi}{3}$, we obtain the the paths along $p_3$ and $p_5$ respectively, completing the construction.

\paragraph{Perturbed Snowflake Tree.}

The perturbed snowflake follows the construction of the perfect snowflake with the exception of $v_0$ having weight $\frac{1}{2} + \gamma$ where $\gamma > 0$.
A perturbed snowflake realization has some distinct qualities from perfect snowflake realizations.  
The angular relationships between adjacent vertices may vary; the distance between adjacent and neighboring vertices may vary as well.
Note that we regard snowflakes with unit weight as a weight of $\frac{1}{2}$.

%Consider a perturbed snowflake with six unit weight vertices having an edge to $v_0$, each unit weighted vertex lying on one distinct path $p_1, \ldots, p_6$.


\begin{minipage}{\linewidth}
\begin{center}
\includegraphics[width=.33\columnwidth]{graphics/modifiedContactGraph.pdf}
\captionof{figure}{A canonical disk arrangement from a perturbed snowflake with 6 unit disks around a central disk with radius $\frac{1}{2} + \gamma$.}\label{fig:modifiedContactGraph.pdf}
\end{center}
\end{minipage}

In Figure \ref{fig:modifiedContactGraph.pdf}, we have a realization of disk arrangement from a perturbed snowflake.
In a disk arrangement of a perfect snowflake, the disks around the central disk contact the adjacent disks.
The disk arrangement from the perturbed snowflake does not have this quality.  
Figure \ref{fig:modifiedContactGraph.pdf} shows a gap $\epsilon(\gamma)$ between adjacent disks around the central disk.  
This gap is formed from the perturbed weight $\frac{1}{2} + \gamma$ of the central disk.  
\begin{lem}\label{lem:cg-1}
Given any realization of a perturbed snowflake of 7 weighted vertices, with the central vertex $v_0$ weighted $\frac{1}{2} + \gamma$ and the others weighted $\frac{1}{2}$, the total additional distance between all vertices is $6 \epsilon(\gamma)$ compared to a perfect snowflake of 7 unit weight vertices.
\end{lem}
\begin{proof}
Consider a canonical disk arrangement of a perturbed snowflake of 7 weighted vertices (see Figure \ref{fig:modifiedContactGraph.pdf}).
The side length of the sides formed between the center of the central disk and two adjacent disks around the central disk is $1 + \gamma$.  
Let the distance between the two adjacent disks be $1 + \epsilon(\gamma)$.
There are a total of $6 \epsilon(\gamma)$ between adjacent centers of disks. 
The total perimeter of the hexagon formed about the centers of the disks in contact with the central disk is $6 + 6\epsilon(\gamma)$. 
Note that 1) the total perimeter of the hexagon formed on a perfect snowflake of 7 weighted vertices is 6 and 2) the canonical disk arrangement can be transformed to any other disk arrangement corresponding to the perturbed snowflake of 7 weighted vertices by pushing the the ring of disks around the central disk together such that all adjacent disks are in contact with each other with the exception of the disks at the end.
\end{proof}
As the perturbed snowflake grows outer layers, we can begin to define parts of the snowflake and the corresponding disk arrangement.

\begin{minipage}{\linewidth}
\begin{center}
\includegraphics[width=.66\columnwidth]{graphics/PerturbedContactGraphAnatomy.pdf}
\captionof{figure}{}\label{fig:PerturbedContactGraphAnatomy.pdf}
\end{center}
\end{minipage}

In Figure \ref{fig:PerturbedContactGraphAnatomy.pdf}, we show an overlay of a realization of a perturbed snowflake, a corresponding disk arrangement, and concentric hexagons about the $v_0$.

\begin{minipage}{\linewidth}
\begin{center}
\includegraphics[width=.33\columnwidth]{graphics/PerturbedSpine.pdf}
\captionof{figure}{?A?S}\label{fig:PerturbedSpine.pdf}
\end{center}
\end{minipage}

In Figure \ref{fig:PerturbedSpine.pdf}, we have a perturbed spine

\begin{minipage}{\linewidth}
\begin{center}
\includegraphics[width=.45\columnwidth]{graphics/PerturbedVertebrae.pdf}
\captionof{figure}{?A?S}\label{fig:PerturbedVertebrae.pdf}
\end{center}
\end{minipage}

In Figure \ref{fig:PerturbedVertebrae.pdf}, we have four an arrangement of disks on the snowflake, off the spine and away from the central disk.  
We call this a vertebrae.





 \section{On the Decidability of Problem \ref{problem:UnorderedContactGraph}}
\begin{proof} 
Consider a $k \times (\sqrt{3}k)$ rectangle section of a triangular lattice, and place disks of radius 1 at each grid point as in Fig. ?????. 
The contact graph of these disks contains 2-cycles. 
Consider the spanning tree $T$ of the contact graph indicated in Fig. ????. 
The tree $T$ decomposes into paths of collinear edges: $T$ contains two paths along the two main diagonals, each containing $2k - 1$ vertices; all other paths have an endpoint on a main diagonal. 
 We now modify the disk arrangement to ensure that its contact graph is $T$. 
 The disks along the main diagonal do not change. 
 We reduce the radii of all other disks by a factor of $1 - k^{-3}$ (as a result, they lose contact with other disks), and then successively translate them parallel in the direction of the shortest path in $T$ to the main diagonal until the contact with the adjacent disk is reestablished. 
 The Hausdorff distance between the union of these disks and the initial $k \times (\sqrt{3}k)$ rectangle is clearly less than 1.
 However, the contact tree $T$ with these radii no longer has a unique realization (small perturbations are possible). 
 To show stability, we argue by induction on the hop distance from the central disk. 
 There are $O(i)$ disks at i hops from the central disk, most one which have radius $(1-k^{-3}) \frac{1}{2}$.
 Since all radii are 1 or $(1-k^{-3}) \frac{1}{2}$,the six neighbors of the central disk can differ from the regular hexagon by at most $O(k )$. 
 Similarly, the disks at $i$ hops from the center be off from the triangular grid pattern by $O(i2^{k-3})$, for $i = 1,2,\dots,k$.
\end{proof}


% Recall that problem (\ref{problem:UnorderedTree}) states: given a positive weighted tree, $T$, is $T$ the contact graph of some disk arrangement where the radii are equal to the vertex weights?  

% \begin{proof}
% Suppose we are given a positive weighted tree, $T = \lr{V_1,E_1}$.  By the Disk Packing Theorem, there is a disk arrangement in the plane, $D$, whose contact graph, $G=\lr{V_2,E_2}$ is isomorphic to $T$.  We need to so that $G=T$ and the radii of the disks in $D$ are equal to the vertex weights of $T$.

% To show that $G=T$, we need to show that $V_1=V_2$ and $E_1=E_2$.  

% To show that the radii of the disks in $D$ are equal to the vertex weights of $T$, we first consider ....  
% \end{proof}

% Related Previous Work. Polygonal linkages (or body-and-joint frameworks) are a gen-
% eralization of classical linkages (bar-and-joint frameworks) in rigidity theory. A linkage
% is a graph G = (V, E) with given edge lengths. A realization of a linkage is a (crossing-
% free) straight-line embedding of G in the plane. Bhatt and Cosmadakis [3] proved that
% the realizability of linkages is NP-hard. Their “logic engine” method [11, 13, 15, 18],
% has become a powerful tool in graph drawing. The logic engine is a graph composed
% of rigid 2-connected components, connected by cut vertices (hinges). The two possi-
% ble realizations of each 2-connected component (that differ by a single reflection) rep-
% resent the truth assignment of a binary variable. This method does not applicable to
% the oriented version of the realizability, where the circular order of the neighbors of
% each vertex is part of the input. Cabello et al. [6, 14] proved that the realizability of 3-
% connected linkages (where the orientation is unique by Steinitz’s theorem) is NP-hard,
% but efficiently decidable for near-triangulations [6, 12].
% Note that every tree linkage can be realized in R2 (with almost collinear edges).
% According to the celebrated Carpenter’s Rule Theorem [9, 21], every realization of a
% path (or a cycle) linkage can be continuously moved (without self-intersection) to any
% other realization. In other words, the realization space of such a linkage is always con-
% nected. However, there are trees of maximum degree 3 with at few as 8 edges whose
% realization space is disconnected [2]; and deciding whether the realization space of a
% tree linkage is connected is PSPACE-complete [1]. (Earlier, Reif [20] showed that it is
% PSPACE-complete to decide whether a polygonal linkage can be moved from one re-
% alization to another among polygonal obstacles in R3.) Cheong et al. [7] considers the
% “inverse” problems of introducing the minimum number of point obstacles to reduce
% the configuration space of a polygonal linkage to a unique realization.
% Connelly et al. [10] showed that the Carpenter’s Rule Theorem generalizes to certain
% polygonal linkages, which are obtained by replacing the edges of a path linkage with
% special polygons called (slender adornments). Our Theorem 1 indicates that if we are
% allowed to replace the edges of a path linkage with arbitrary convex polygons, then
% deciding whether the realization space is empty or not is already NP-hard.
% Recognition problems for intersection graphs of various geometric object have a
% rich history [18]. Breu and Kirkpatrick [5] proved that it is NP-hard to decide whether
% a graph G is the contact graph of unit disks in the plane (a.k.a. recognizing coin graphs
% is NP-hard). A simpler proof was later provided via the logic engine [13]. It is also NP-
% hard to recognize the contact graphs of pseudo-disks [18] and disks of bounded radii [4]
% in the plane, and unit disks in higher dimensions [17, 18]. All these hardness reductions
% produce graphs of high genus, and do not apply to trees. Note that the contact graphs
%  of disks (of arbitrary radii) are exactly the planar graph (by Koebe’s circle packing
%  theorem), and planarity testing is polynomial. Consequently, every tree is the contact
%  graph of disks of some radii in the plane.


%My goals for next week are as follows:
%1) Clean up chapter 1.  This will happen over the next several weeks.
%2) Continue section 2.3
%3) Add graphics to epsilon-approximation definition to facilitate the explanation.
%4) Anything else  I have time for.



% $\overline{v_0 }$

%Define the snowflake graph, S_i, without pictures
%v_0 has six paths attached to it, p_1, p_2, ..., p_6.  Each path has i vertices.
%for every other path p_1, p_3, and p_5 ,
%	Each vertex on that path has two paths attached, one path on each side of p_k.
%	The number of vertices that lie on the path attached to the j^{th} vertex of p_k is $i-j$
%Let all vertices on the contact graph have weight 1 except $v_0$ whose weight is 1 + \epsilon.
%For any $i$, the weighted contact graph $S_i$, is a subgraph of the unit distance graph of the triangular lattice
%lattice be V = {a(1,0) + b(1/2, \frac{\sqrt{2}}{3} such that a,b \in \bbZ}
%		  E = {{u,v} such that u,v \in V and ||u-v||=1}
%Define this as a canonical position of the disks corresponding to the vertices; goal: show every realization of S_i's disks is close to the canonical position of the disk under hausdorff distance (see lemma)
%NTS: there exists a disk arrangement that corresponds to this weighted contact graph, S_i
%pf:For any two adjacent stems are disjoint, angular displacement and planar (point) displacement
  

%    \begin{prob}[Unordered Realizibility Problem for the Tree]\label{problem:UnorderedTree}
%    For a tree with positive weights for the verticies, it asks whether it is a contact graph of some 
%    disk arrangement where the radii are equal to the vertex weights.
%    \end{prob}
%    
%    \begin{prob}[Ordered Realizibility Problem for the Tree]\label{problem:OrderedTree}
%    For a tree with positive weights for the vertices, it asks whether its corresponding graph is the 
%    ordered contact graph of some disk arrangement where the radii equal the vertex weights.
%    \end{prob}