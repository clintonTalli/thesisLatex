\section{Weight Trees $T_k$}
In this section we describe a particular family of unit weight trees and corresponding contact graphs disk arrangements called \textit{snowflakes}.  
Note that we regard snowflakes with unit weight as a weight of $\frac{1}{2}$.  
For $i \in \bbN$, the construction of the snowflake tree, $T_i$, is as follows:
\begin{itemize}
\item Let $v_0$ be a dvertex that has six paths attached to it: $p_1$, $p_2$, $\dots$, $p_6$.  Each path has $i$ vertices.
\item For every other path $p_1$, $p_3$, and $p_5$: 
	\begin{itemize}
		\item 	Each vertex on that path has two paths attached, one path on each side of $p_k$.
		\item	The number of vertices that lie on a path attached to the $j^\text{th}$ vertex of $p_k$ is $i-j$.
	\end{itemize}
\end{itemize}

\begin{figure}[!htbp]
\begin{center}
\includegraphics{graphics/snowflakeOutline5LayerSmall.pdf}
\caption{The same contact graph as in figure \ref{fig:hexagonOutline5LayerSmall.pdf} overlayed with the a perfectly weighted snowflake tree.}\label{fig:snowflakeOutline5LayerSmall.pdf}
\end{center}
\end{figure}

A \textit{perfectly weighted snowflake tree} is a snowflake tree with all vertices having weight $\frac{1}{2}$.   
A \textit{perturbed snowflake tree} is a snowflake tree with all vertices having weight of 1 with the exception of $v_0$;  in a perturbed snowflake tree, $v_0$ will have a weight of $\frac{1}{2} + \gamma$.  
For our analysis, all realizations of any snowflake, perfect or perturbed, shall have $v_0$ fixed at origin.  
% This is said to be the canonical position under Hausdorff distance of the snowflake tree.   

\paragraph{Perfectly Weighted Snowflake Tree.}

Consider the graph of the triangular lattice with unit distant edges:
\begin{eqnarray*}
V &=& \left\lbrace a\cdot (1,0) + b \cdot \left(\frac{1}{2},\frac{\sqrt{3}}{2}\right) : a,b \in \bbZ \right\rbrace\\
E &=& \left\lbrace \left\lbrace u,v \right\rbrace : \vert\vert u-v \vert\vert = 1 \text{ and } u,v \in V\right\rbrace
\end{eqnarray*}
The following graph, $G=(V,E)$ is said to be the \textit{unit distance graph} of the triangular lattice.  
We can show that no two distinct edges of this graph are non-crossing.  
First suppose that there were two distinct edges that crossed, $\left\lbrace u_1,v_1 \right\rbrace $ and $\left\lbrace u_2,v_2 \right\rbrace$.  
With respect to $u_1$, there are 6 possible edges corresponding to it, with each edge $\frac{\pi}{3}$ radians away from the next.  
Neither edge crosses another; and so we have a contradiction that there are no edge crossings with $\left\lbrace u_1,v_1 \right\rbrace $.  


The perfectly weighted snowflake tree that is a subgraph over the \textit{unit distance graph}, $G=(V,E)$, of the triangular lattice.  
To show this, for any $S_i$, fix $v_0 = 0 \cdot \cdot (1,0) + 0 \cdot \left(\frac{1}{2},\frac{\sqrt{3}}{2}\right)=\lr{0,0} \in V$ at origin.  
Next consider the six paths attached from origin.  
Fix each consecutive path $\frac{\pi}{3}$ radians away from the next such that the following points like on the corresponding paths: $\lr{1,0} \in p_1, \lr{\frac{1}{2} ,\frac{\sqrt{2}}{3}} \in p_2,\lr{-\frac{1}{2}\p_4,\frac{\sqrt{3}}{2}} \in p_3, \lr{-1,0} \in p4, \lr{-\frac{1}{2},-\frac{\sqrt{3}}{2}}\in p_5,\lr{\frac{1}{2},-\frac{\sqrt{3}}{2}}\in p_6$.  
For $S_i$, there are $i$ vertices on each path.  

We define the six paths from origin as follows:      
\begin{eqnarray*}
p_1 &=& \set{a\cdot\lr{1,0} = \vec{v}}{a \in \bbR^+}\\
p_2 &=& \set{a\cdot\lr{\frac{1}{2},\frac{\sqrt{3}}{2}} = \vec{v}}{a \in \bbR^+}\\
p_3 &=& \set{-a\cdot \lr{1,0} + a \cdot \lr{\frac{1}{2},\frac{\sqrt{3}}{2}} = a\lr{-\frac{1}{2},\frac{\sqrt{3}}{2}} = \vec{v}}{a \in \bbR^+}\\
p_4 &=& \set{a \cdot \lr{-1,0} = \vec{v}}{a \in \bbR^+}\\
p_5 &=& \set{a \cdot \lr{-\frac{1}{2},-\frac{\sqrt{3}}{2}}  = \vec{v}}{a \in \bbR^+}\\
p_6 &=& \set{ a\cdot \lr{1,0} - a \cdot \lr{\frac{1}{2},\frac{\sqrt{3}}{2}}= a \cdot \lr{\frac{1}{2}, -\frac{\sqrt{3}}{2}}}{a \in \bbR^+} 
\end{eqnarray*}
For $S_i$ there exists $i$ vertices on each path.  We shall denote the $\ith$ vertex on the $\jth$ path as $v_{j,i}$.  
For each path defined above, the paths are defined as a set of vectors, $\vec{v} = a \cdot \vec{p}$  for some $a \in \bbR^+$ and $\vec{p} \in \bbR^2$.  
By setting $a = 1,2,\dots, i$, we obtain points that are contained in $V$.  
For $j = 1$, $3$, $5$ and $l = 3 b \leq i$ where $b \in \bbN$,  there exists two paths attached to each vertex $v_{j,l}$.  
We borrow the term \textit{petiole} from botany to describe the two paths attached to $v_{j,l}$.  
In botany, the stalk that attaches to a stem of a plant is called a petiole; petioles usually have leaves attached to their ends.  
For $S_i$, each petiole attached to the $k^\text{th}$ vertex of $p_j$, there are $i-k$ vertices. 
For each vertex $v$ on a petiole, which is not in the paths $p_1$, $p_3$, or $p_5$, there are two \textit{leafs} on either side of the vertex; each leaf is a vertex that has an edge with $v$.  
The one exception to the two leafs rule is on the first vertex of the petiole off of $p_1$, $p_3$, or $p_5$.  
In this exception, attach one leaf to the side of the vertex that is closest to center vertex $v_0$.

The triangular lattice is symmetric under rotation about $v_0$ by $\frac{\pi}{3}$ radians.  
For each vertex $v_{1,l}$ and $l = 3 b \leq i$ where $b \in \bbN$, we place two petioles from it; the first petiole $\frac{\pi}{3}$ above $p_1$ at $v_{1,l}$ and $\frac{-\pi}{3}$ below $p_1$ at $v_{1,l}$ and call these petioles $p_{1,l}^+$ and $p_{1,l}^-$ respectively.  
With respect to $v_{1,l}$, one unit along $p_{1,l}^+$ is a point on the triangular lattice and similarly so on $p_{1,l}^-$.  
Continuing the walk along these paths, unit distance-by-unit distance, we obtain the next point corresponding point on the the triangular lattice up to $i-k$ distance away from $v_{1,l}$.  
Without loss of generality, for each each vertex $v$ of the petiole which are not in $p_1$ have two associated leaf nodes $v^+$ and $v^-$; $v^+$ is placed $\frac{\pi}{3}$ and one unit above $v$ and $v^-$ is placed $\frac{-\pi}{3}$ and one unit below $v$.  
Thus all leaf nodes are in the triangular lattics.
This shows that each of the $i-k$ vertices on $p_{1,l}^-$, $p_{1,l}^+$, and leafs are in $V$.
By rotating all of the paths along $p_1$ by $\frac{2\pi}{3}$ and $\frac{4\pi}{3}$, we obtain the the paths along $p_3$ and $p_5$ respectively, completing the construction.

In Figure \ref{fig:hexagonPetiolesLeafs9Layers.pdf}, we have a set of unit radii disks arranged in a manner that outlines the perfectly weighted snowflake description above.

\begin{minipage}{\linewidth}
\begin{center}
\includegraphics{graphics/hexagonPetiolesLeafs9Layers.pdf}
\captionof{figure}{A contact graph that resembles the shape of concentric hexagons.}\label{fig:hexagonPetiolesLeafs9Layers.pdf}
\end{center}
\end{minipage}
