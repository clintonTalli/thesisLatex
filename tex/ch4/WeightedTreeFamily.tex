\section{Weighted Trees $T_i$}
In this section we describe a particular family of unit weight trees and corresponding contact graphs disk arrangements called \textit{snowflakes}.  
Note that we regard snowflakes with unit weight as a weight of $r$.  
For $i \in \bbN$, the construction of the snowflake tree, $T_i$, is as follows:
\begin{itemize}
\item Let $v_0$ be a vertex that has six paths attached to it: $p_1$, $p_2$, $\dots$, $p_6$.  Each path has $i$ vertices.
\item In botany, the stalk that attaches to a stem of a plant is called a \textit{petiole}; petioles usually have leaves attached to their ends. Our snowflake will incorporate these concepts by incorporating petioles and leaves.  Petioles will be paths that extend off the arms.  \textit{Leaves} will be paths of one vertex that extend of the petioles and arms. We will now attach paths (petioles) onto every other path $p_1$, $p_3$, and $p_5$: 
	\begin{itemize}
		\item 	Every third vertex on that path has two petioles attached, one petiole on each side of $p_k$; otherwise the vertex on $p_k$ will have two leaves, one on either side.
		\item	The number of vertices that lie on a petiole attached to the $\jth$ vertex of  $p_k$ is $i-j$.
		\item The first vertex of the $i-j$ vertices has one petiole attached; the remaining $i-j-1$ vertices contain two paths. Each of these paths contain only one vertex.  These paths are called \textit{leaves}. 
	\end{itemize}
\item Attach leaves to the vertices on paths $p_2$, $p_4$, and $p_6$.
\end{itemize}
A drawaing example of this snowflake description is shown in Figure \ref{fig:hexagonPetiolesLeafs9Layers.pdf}

\begin{minipage}{\linewidth}
\begin{center}
\includegraphics[width=.4\columnwidth]{graphics/hexagonPetiolesLeafs9Layers.pdf}
\captionof{figure}{A contact graph that resembles the shape of concentric hexagons.}\label{fig:hexagonPetiolesLeafs9Layers.pdf}
\end{center}
\end{minipage}

A \textit{perfectly weighted snowflake tree} is a snowflake tree with all vertices having weight $r$.   
A \textit{perturbed snowflake tree} is a snowflake tree with all vertices having weight of 1 with the exception of $v_0$;  in a perturbed snowflake tree, $v_0$ will have a weight of $r + \zeta$.  
The value of $\zeta $ will be determined later on in the proof of the Lemma \ref{lem:ch4IntroLemma}.  
For our analysis, all realizations of any snowflake, perfect or perturbed, shall have the disk corresponding to $v_0$ is centered at the origin.  
We can assume that a neighbor of $v_0$ is on the positive x-axis.     

\paragraph{Perfectly Weighted Snowflake Tree.}

Consider the graph of the triangular lattice with unit distant edges:
\begin{eqnarray*}
V &=& \left\lbrace a\cdot (1,0) + b \cdot \left(\frac{1}{2},\frac{\sqrt{3}}{2}\right) : a,b \in \bbZ \right\rbrace\\
E &=& \left\lbrace \left\lbrace u,v \right\rbrace : \vert\vert u-v \vert\vert = 1 \text{ and } u,v \in V\right\rbrace
\end{eqnarray*}
The graph, $G=(V,E)$ is said to be the \textit{unit distance graph} of the triangular lattice.  
We can show that no two distinct edges cross.  
First suppose that there were two distinct edges that crossed, $\left\lbrace u_1,v_1 \right\rbrace $ and $\left\lbrace u_2,v_2 \right\rbrace$. 
By strict triangle inequality of the sides and diagonals of the convex quadrilateral $\lr{u_1, u_2, v_1, v_2}$, we have the following result:
$$\vlr{u_1,u_2} + \vlr{u_2,v_1} + \vlr{v_1,v_2} + \vlr{u_1,v_2}< 4 < 2 \lr{\vlr{u_1,u_2} + \vlr{u_1,u_2 }}=4.$$ 
% The inequality implies that one of the four terms on the left side is less than 1. However, no two points of the triangular lattice are less than unit distance apart---contradiction. 
On the lefthand side, one of the four terms is less than 1.  
No two points of the triangular lattice is less than 1 which is a contradiction.

The perfectly weighted snowflake tree that is a subgraph over the \textit{unit distance graph}, $G=(V,E)$, of the triangular lattice.  
For the remainder of the thesis, a \textit{snowflake}, the ordered contact graph of $S_i$ contains the tree $T_i$.  
To show this, for any $S_i$, fix $v_0 = 0 \cdot (1,0) + 0 \cdot \left(\frac{1}{2},\frac{\sqrt{3}}{2}\right)=\lr{0,0} \in V$ at origin.  
Next consider the six paths attached from origin.  
Fix each consecutive path $\frac{\pi}{3}$ radians away from the next such that the following points like on the corresponding paths: $\lr{1,0} \in p_1, \lr{\frac{1}{2} ,\frac{\sqrt{2}}{3}} \in p_2,\lr{-\frac{1}{2}\p_4,\frac{\sqrt{3}}{2}} \in p_3, \lr{-1,0} \in p4, \lr{-\frac{1}{2},-\frac{\sqrt{3}}{2}}\in p_5,\lr{\frac{1}{2},-\frac{\sqrt{3}}{2}}\in p_6$.  
For $S_i$, there are $i$ vertices on each path.  

We define the six paths from origin as follows:      
\begin{eqnarray*}
p_1 &=& \set{a\cdot\lr{1,0}}{a = 1,2,\dots, i}\\
p_2 &=& \set{a\cdot\lr{\frac{1}{2},\frac{\sqrt{3}}{2}}}{a = 1,2,\dots, i}\\
p_3 &=& \set{a\lr{-\frac{1}{2},\frac{\sqrt{3}}{2}}}{a = 1,2,\dots, i}\\
p_4 &=& \set{a \cdot \lr{-1,0}}{a = 1,2,\dots, i}\\
p_5 &=& \set{a \cdot \lr{-\frac{1}{2},-\frac{\sqrt{3}}{2}}}{a = 1,2,\dots, i}\\
p_6 &=& \set{a \cdot \lr{\frac{1}{2}, -\frac{\sqrt{3}}{2}}}{a = 1,2,\dots, i}
\end{eqnarray*}
For $S_i$ there exists $i$ vertices on each path.  We shall denote the $\ith$ vertex on the $\jth$ path as $v_{j,i}$.  
For each path defined above, the paths are defined as a set of vectors, $\vec{v} = a \cdot \vec{p}$  for some $a \in \bbN$ and $\vec{p} \in \bbR^2$.  
By setting $a = 1,2,\dots, i$, we obtain points that are contained in $V$.  
For $j = 1$, $3$, $5$ and $\ell = 3 b \leq i$ where $b \in \bbN$,  there exists two paths attached to each vertex $v_{j,\ell}$.  
% We borrow the term \textit{petiole} from botany to describe the two paths attached to $v_{j,l}$.  
% In botany, the stalk that attaches to a stem of a plant is called a petiole; petioles usually have leaves attached to their ends.  
For $S_i$, each petiole attached to the $\ell^\text{th}$ vertex of $p_j$, there are $i-\ell$ vertices. 
For each vertex $v$ on a petiole, which is not in the paths $p_1$, $p_3$, or $p_5$, there are two \textit{leaves} on either side of the vertex; each leaf is a vertex that has an edge with $v$.  
The exceptions to the two leaves rule is on the first and last vertices of the petiole off of $p_1$, $p_3$, or $p_5$.  
In these exception, attach one leaf to the side of the vertex that is closest to center vertex $v_0$.

The triangular lattice is symmetric under rotation about $v_0$ by $\frac{\pi}{3}$ radians.  
For each vertex $v_{1,l}$ and $l = 3 b \leq i$ where $b \in \bbN$, we place two petioles from it; the first petiole $\frac{\pi}{3}$ above $p_1$ at $v_{1,l}$ and $\frac{-\pi}{3}$ below $p_1$ at $v_{1,l}$ and call these petioles $p_{1,l}^+$ and $p_{1,l}^-$ respectively.  
With respect to $v_{1,l}$, one unit along $p_{1,l}^+$ is a point on the triangular lattice and similarly so on $p_{1,l}^-$.  
Without loss of generality, for each vertex $v$ of the petiole which are not in $p_1$ has two associated leaf nodes $v^+$ and $v^-$; $v^+$ is placed $\frac{\pi}{3}$ and one unit above $v$ and $v^-$ is placed $\frac{-\pi}{3}$ and one unit below $v$.  
Thus all leaf nodes are in the triangular lattice.
This shows that each of the $i-k$ vertices on $p_{1,l}^-$, $p_{1,l}^+$, and leaves are in $V$.
By rotating all of the paths along $p_1$ by $\frac{2\pi}{3}$ and $\frac{4\pi}{3}$, we obtain the paths $p_3$ and $p_5$ respectively, completing the construction.

In Figure \ref{fig:hexagonPetiolesLeafs9Layers.pdf}, we have a set of unit radii disks arranged in a manner that outlines the perfectly weighted snowflake description above.
