\chapter{Realizability Problems for Weighted Trees}\label{chp:disk}

In this chapter our goal is to prove Theorem \ref{thm:disk}: It is NP-Hard to decide whether a given tree with positive vertex weights is the contact graph of a disk arrangements with specified radii.   
This chapter's approach to proving Theorem \ref{thm:disk} introduces an ordered weighted tree $T$, perturbed ordered weight tree $T_\epsilon$, the Hausdorff distance, and then prove a lemma which shows that hexagons can be approximated by an ordered disk contact graph corresponding to the weighted tree $T_\epsilon$.


\section{Hausdorff Distance}  
Let $A$ and $B$ be sets in the plane. The \textit{directed Hausdorff distance} is: 
\begin{equation}\label{eqn:ContactGraphV3-1}
d\lr{A,B} = \sup_{a \in A} \inf_{b \in B} \left\vert\left\vert a-b \right\vert \right\vert
\end{equation}
$d\lr{A,B}$ finds the furthest point $a \in A$ from any point in $B$.  \textit{Hausdorff distance} is
\begin{equation}\label{eqn:ContactGraphV3-2}
D\lr{A,B} = \max \left\lbrace d\lr{A,B}, d\lr{B,A} \right\rbrace
\end{equation}

In Figure \ref{fig:HausdorffDistanceExample1.pdf}, we have two sets $X$ and $Y$ and illustrate $d(X,Y)$ and $d(Y,X)$.  
From this, it is possible to calculate the Hausdorff distance between $X$ and $Y$.

\begin{minipage}{\linewidth}
\begin{center}
\includegraphics[width=.33\columnwidth]{graphics/HausdorffDistanceExample1.pdf}
\captionof{figure}{An illustrative example of $d(X,Y)$ and $d(Y,X)$ where $X$ is the inner curve, and $Y$ is the outer curve.}\label{fig:HausdorffDistanceExample1.pdf}
\end{center}
\end{minipage}

\paragraph{$\epsilon$-approximation}
The weighted graph, $G$, is an \textit{$\epsilon$-approximation} of a polygon $P$ if the Hausdorff distance between every realization of $G$ as a contact graph of disks and a congruent copy of $P$ is at most epsilon.  
A weighted graph $G$ is said to be a \textit{$\BigOh{f(x)}$-approximation} of a polygon P if there is a positive constant $M$ such that for all sufficiently large values of $x$ the Hausdorff distance between every realization such realization of $G$ as a contact graph of disks and a congruent copy of $P$ is at $M \cdot \vert f(x)\vert$. 
A weighted graph $G$ is said to be a \textit{stable} if it has the property that for every two such realizations of $G$, the distance between the centers of the corresponding disks is at most $\epsilon$ after a suitable rigid transformation.

Suppose we have a unit disk $U$ and we have a grid overlayed on the disk with side length $\delta$.  
Let $S_1(\delta)$ be the the union of squares formed by the grid found completely in the interior of the disk $U$.  
Let $S_2(\delta)$ be the union of squares formed by the grid with some point of the boundary of the square contained in the interior of disk $U$.  
The Hausdorff distance of $U$ and $S_1(\delta)$ is at most $H\lr{S_1(\delta), U}=\sqrt{2}\delta$.  
Similarly, the Hausdorff distance of $U$ and $S_2(\delta)$ is at most $H\lr{S_2(\delta), U}=\sqrt{2}\delta$.
Thus for any $\epsilon>0$  choose a $\delta$ such that $\sqrt{2}\delta \leq \epsilon$.  
Similiaryly, the Hausdorff distance of $U$ and $S_2(\delta)$ is at most $H\lr{S_2(\delta), U}=\sqrt{2}\delta$.

\begin{prob}[Appoximating Polygonal Shapes with Contact Graphs]\label{problem:ApproxShapesWithContactGraphs}
For every $\epsilon >0$ and polygon $P$, there exists a contact graph $G = (V,E)$  such that the Hausdorff distance $d(P,G) < \epsilon$
\end{prob}


\begin{lem}\label{lem:ch4IntroLemma}
for ever $\epsilon > 0$, there exists an ordered weighted tree $T_\epsilon$ such that every realization of $T_\epsilon$ as an ordered disk contact graph where the radii of the disks equal the vertex weights.
\end{lem} 
Using Lemma \ref{lem:ch4IntroLemma}, we prove Theorem \ref{thm:disk} by extending the modified auxiliary construction in Chapter \ref{chapter:polygonalLinkage}.  

We first cover the preliminary concepts of Hausdorff distance and the ordered weighted tree families of $T$ and $T_\epsilon$.  
We then continue with the proof of Lemma \ref{lem:ch4IntroLemma} and Theorem \ref{thm:disk}.  
\section{Weighted Trees $T_k$}
In this section we describe a particular family of unit weight trees and corresponding contact graphs disk arrangements called \textit{snowflakes}.  
Note that we regard snowflakes with unit weight as a weight of $r$.  
For $i \in \bbN$, the construction of the snowflake tree, $T_i$, is as follows:
\begin{itemize}
\item Let $v_0$ be a vertex that has six paths attached to it: $p_1$, $p_2$, $\dots$, $p_6$.  Each path has $i$ vertices.
\item In botany, the stalk that attaches to a stem of a plant is called a \textit{petiole}; petioles usually have leaves attached to their ends.  We will now attach paths (petioles) onto every other path $p_1$, $p_3$, and $p_5$: 
	\begin{itemize}
		\item 	Every third vertex on that path has two petioles attached, one petiole on each side of $p_k$. 
		\item	The number of vertices that lie on a petiole attached to the $\jth$ vertex of  $p_k$ is $i-j$.
		\item The first vertex of the $i-j$ vertices has one petiole attached; the remaining $i-j-1$ vertices contain two paths. Each of these paths contain only one vertex.  These paths are called \textit{leaves}. 
	\end{itemize}
\end{itemize}


\begin{minipage}{\linewidth}
\begin{center}
\includegraphics[width=.4\columnwidth]{graphics/hexagonPetiolesLeafs9Layers.pdf}
\captionof{figure}{A contact graph that resembles the shape of concentric hexagons.}\label{fig:hexagonPetiolesLeafs9Layers.pdf}
\end{center}
\end{minipage}

A \textit{perfectly weighted snowflake tree} is a snowflake tree with all vertices having weight $r$.   
A \textit{perturbed snowflake tree} is a snowflake tree with all vertices having weight of 1 with the exception of $v_0$;  in a perturbed snowflake tree, $v_0$ will have a weight of $r + \zeta(\epsilon)$.  
The value of $\zeta (\epsilon)$ will be determined later on in the proof of the Lemma \ref{lem:ch4IntroLemma}.  
For our analysis, all realizations of any snowflake, perfect or perturbed, shall have the disk corresponding to $v_0$ is centered at the origin.  We can assume one of the paths is on the x-axis.  
% This is said to be the canonical position under Hausdorff distance of the snowflake tree.   

\paragraph{Perfectly Weighted Snowflake Tree.}

Consider the graph of the triangular lattice with unit distant edges:
\begin{eqnarray*}
V &=& \left\lbrace a\cdot (1,0) + b \cdot \left(\frac{1}{2},\frac{\sqrt{3}}{2}\right) : a,b \in \bbZ \right\rbrace\\
E &=& \left\lbrace \left\lbrace u,v \right\rbrace : \vert\vert u-v \vert\vert = 1 \text{ and } u,v \in V\right\rbrace
\end{eqnarray*}
The following graph, $G=(V,E)$ is said to be the \textit{unit distance graph} of the triangular lattice.  
We can show that no two distinct edges of this graph are non-crossing.  
First suppose that there were two distinct edges that crossed, $\left\lbrace u_1,v_1 \right\rbrace $ and $\left\lbrace u_2,v_2 \right\rbrace$.  
With respect to $u_1$, there are 6 possible edges corresponding to it, with each edge $\frac{\pi}{3}$ radians away from the next.  
No two edges cross.
% Neither edge crosses another; and so we have a contradiction that there are no edge crossings with $\left\lbrace u_1,v_1 \right\rbrace $.  


The perfectly weighted snowflake tree that is a subgraph over the \textit{unit distance graph}, $G=(V,E)$, of the triangular lattice.  
For the remainder of the thesis, a \textit{snowflake}, $S_i$ is a realization of a weighted tree $T_i$.  
To show this, for any $S_i$, fix $v_0 = 0 \cdot (1,0) + 0 \cdot \left(\frac{1}{2},\frac{\sqrt{3}}{2}\right)=\lr{0,0} \in V$ at origin.  
Next consider the six paths attached from origin.  
Fix each consecutive path $\frac{\pi}{3}$ radians away from the next such that the following points like on the corresponding paths: $\lr{1,0} \in p_1, \lr{\frac{1}{2} ,\frac{\sqrt{2}}{3}} \in p_2,\lr{-\frac{1}{2}\p_4,\frac{\sqrt{3}}{2}} \in p_3, \lr{-1,0} \in p4, \lr{-\frac{1}{2},-\frac{\sqrt{3}}{2}}\in p_5,\lr{\frac{1}{2},-\frac{\sqrt{3}}{2}}\in p_6$.  
For $S_i$, there are $i$ vertices on each path.  

We define the six paths from origin as follows:      
\begin{eqnarray*}
p_1 &=& \set{a\cdot\lr{1,0} = \vec{v}}{a = 1,2,\dots, i}\\
p_2 &=& \set{a\cdot\lr{\frac{1}{2},\frac{\sqrt{3}}{2}} = \vec{v}}{a = 1,2,\dots, i}\\
p_3 &=& \set{-a\cdot \lr{1,0} + a \cdot \lr{\frac{1}{2},\frac{\sqrt{3}}{2}} = a\lr{-\frac{1}{2},\frac{\sqrt{3}}{2}} = \vec{v}}{a = 1,2,\dots, i}\\
p_4 &=& \set{a \cdot \lr{-1,0} = \vec{v}}{a = 1,2,\dots, i}\\
p_5 &=& \set{a \cdot \lr{-\frac{1}{2},-\frac{\sqrt{3}}{2}}  = \vec{v}}{a = 1,2,\dots, i}\\
p_6 &=& \set{ a\cdot \lr{1,0} - a \cdot \lr{\frac{1}{2},\frac{\sqrt{3}}{2}}= a \cdot \lr{\frac{1}{2}, -\frac{\sqrt{3}}{2}}}{a = 1,2,\dots, i}
\end{eqnarray*}
For $S_i$ there exists $i$ vertices on each path.  We shall denote the $\ith$ vertex on the $\jth$ path as $v_{j,i}$.  
For each path defined above, the paths are defined as a set of vectors, $\vec{v} = a \cdot \vec{p}$  for some $a \in \bbN$ and $\vec{p} \in \bbR^2$.  
By setting $a = 1,2,\dots, i$, we obtain points that are contained in $V$.  
For $j = 1$, $3$, $5$ and $\ell = 3 b \leq i$ where $b \in \bbN$,  there exists two paths attached to each vertex $v_{j,\ell}$.  
% We borrow the term \textit{petiole} from botany to describe the two paths attached to $v_{j,l}$.  
% In botany, the stalk that attaches to a stem of a plant is called a petiole; petioles usually have leaves attached to their ends.  
For $S_i$, each petiole attached to the $\ell^\text{th}$ vertex of $p_j$, there are $i-\ell$ vertices. 
For each vertex $v$ on a petiole, which is not in the paths $p_1$, $p_3$, or $p_5$, there are two \textit{leaves} on either side of the vertex; each leaf is a vertex that has an edge with $v$.  
The exceptions to the two leaves rule is on the first and last vertices of the petiole off of $p_1$, $p_3$, or $p_5$.  
In these exception, attach one leaf to the side of the vertex that is closest to center vertex $v_0$.

The triangular lattice is symmetric under rotation about $v_0$ by $\frac{\pi}{3}$ radians.  
For each vertex $v_{1,l}$ and $l = 3 b \leq i$ where $b \in \bbN$, we place two petioles from it; the first petiole $\frac{\pi}{3}$ above $p_1$ at $v_{1,l}$ and $\frac{-\pi}{3}$ below $p_1$ at $v_{1,l}$ and call these petioles $p_{1,l}^+$ and $p_{1,l}^-$ respectively.  
With respect to $v_{1,l}$, one unit along $p_{1,l}^+$ is a point on the triangular lattice and similarly so on $p_{1,l}^-$.  
Continuing the walk along these paths, unit distance-by-unit distance, we obtain the next point on the triangular lattice up to $i-k$ distance away from $v_{1,l}$.  
Without loss of generality, for each vertex $v$ of the petiole which are not in $p_1$ has two associated leaf nodes $v^+$ and $v^-$; $v^+$ is placed $\frac{\pi}{3}$ and one unit above $v$ and $v^-$ is placed $\frac{-\pi}{3}$ and one unit below $v$.  
Thus all leaf nodes are in the triangular lattice.
This shows that each of the $i-k$ vertices on $p_{1,l}^-$, $p_{1,l}^+$, and leaves are in $V$.
By rotating all of the paths along $p_1$ by $\frac{2\pi}{3}$ and $\frac{4\pi}{3}$, we obtain the paths $p_3$ and $p_5$ respectively, completing the construction.

In Figure \ref{fig:hexagonPetiolesleaves9Layers.pdf}, we have a set of unit radii disks arranged in a manner that outlines the perfectly weighted snowflake description above.


\subsection{Perturbed Weighted Trees $T_\epsilon$}

A perturbed weighted tree $T_\epsilon$ is a weighted unit tree with unit weight on every vertex with the exception of the root vertex having weight $r+\zeta(\epsilon)$ where $\epsilon>0$ can be realized as a disk touching graph (a disk arrangement) and $r$ is the unit length.  

The perturbed snowflake follows the construction of the perfect snowflake with the exception of $v_0$ having weight $r+ \zeta(\epsilon)$ where $\epsilon > 0$.
A perturbed snowflake realization has some distinct qualities from perfect snowflake realizations.  
The angular relationships between adjacent vertices may vary; the distance between adjacent and neighboring vertices may vary as well.


In general, the perturbation $\epsilon$ can modify the realization of a perfect snowflake $S_i$ in the following ways:

\textbf{Modification of $S_1$.} 

Given a instance of a perturbed snowflake with $v_0$ having weight $r + \zeta(\epsilon)$ where $\epsilon > 0$, vertices neighboring $v_0$ each have a range of placement on the plane when realizaed as a disk arrangement. 
Figure \ref{fig:modifiedContactGraph} shows a realization of $S_1$ and illustrates one such example of possible gaps, $\zeta(\epsilon)$, that could be created between adjacent disks of $S_1$ in a perfect snowflake.  

Note that (1) the adjacent disks in a perfect snowflake may or may not be adjacent in a given perturbed snowflake of $S_1$ and (2) $S_1 \subseteq S_i$ for any $i \in \bbN$.  



\begin{minipage}{\linewidth}
\begin{center}
\includegraphics[width=.95\columnwidth]{graphics/LineSegmentDelta.pdf}
\captionof{figure}{}\label{fig:LineSegmentDelta.pdf}
\end{center}
\end{minipage}
\begin{eqnarray*}
\sin \lr{\frac{\pi}{6} - \chi} &=& \frac{1}{2+\psi}\\
\sin \frac{\pi}{6} \cos \chi &=& \frac{1}{2+\psi} + \cos \frac{\pi}{6} \sin \chi\\
&\iff&\\
\frac{1}{2} &\geq& \frac{1}{2} \cos \chi \\
&=& \frac{1}{2+\psi} + \frac{\sqrt{3}}{2} \sin \chi\\
&\geq& \frac{1}{2+\psi} + \frac{\sqrt{3}}{2} \lr{ \chi - \frac{\chi^3}{6}}\\
&\iff&\\
\frac{1}{2}-\frac{1}{2+\psi} &\geq& \frac{\sqrt{3}}{2} \lr{ \chi - \frac{\chi^3}{6}} \qquad \text{if }\chi < 1\\
\frac{\psi}{2 ( 2 + \psi)} &\geq&  \frac{5\sqrt{3}}{12} \chi\\
\frac{3 \psi}{5\sqrt{3}}=\frac{12}{5\sqrt{3}} \frac{\psi}{4} &\geq& \chi
\end{eqnarray*}

\begin{minipage}{\linewidth}
\begin{center}
\includegraphics[width=.20\columnwidth]{graphics/part1ch4.pdf}
\captionof{figure}{}\label{fig:part1ch4.pdf}
\end{center}
\end{minipage}
\begin{eqnarray*}
\sin \lr{\frac{\pi}{6}- \chi} &=& \frac{1}{2+ \psi}\\
\sin \frac{\pi}{6} \cos \chi - \cos \frac{\pi}{6} \sin \chi &=& = \frac{1}{2+\psi}\\
\frac{1}{2} \cos \chi - \frac{\sqrt{3}}{2} \sin \chi &=& \frac{1}{2+\psi}\\
&\geq& \frac{1}{2+\psi} + \frac{\sqrt{3}}{2} \chi - \frac{\sqrt{3}}{12} \chi^3 \\
\frac{1}{2} &\geq&  \frac{1}{2+\psi} + \frac{\sqrt{3}}{4} \chi\\
\end{eqnarray*}

$$\frac{\psi}{4} \geq \frac{\psi}{2 (2 + \psi)}\geq \frac{\sqrt{3}}{4} \chi\iff \frac{\psi}{\sqrt{3}} \geq \chi$$

$$\frac{\pi}{3} - 2 \chi = \lambda_\text{min} \leq \lambda \leq \lambda_\text{max} = \frac{\pi}{3} + 10 \chi$$


\begin{minipage}{\linewidth}
\begin{center}
\includegraphics[width=.66\columnwidth]{graphics/ch4Paralellogram.pdf}
\captionof{figure}{}\label{fig:ch4Paralellogram.pdf}
\end{center}
\end{minipage}
% ----------------------------
% CHAPTER 3. (Labels refer to the file cat.pdf)

% Glossary of formulas:
% For arctan, sin, cos, and sec, I would give upper and lower bounds
% (rather than approximate values).
% For example, the Maclaurin series of tan^{-1} gives: x-\frac{x}{3} <
% \tan^{-1}x < x for all 0<x<1.
% All lowerupper bounds are derived from the first or the first two terms
% of the Maclaurin series.


% In (3.5), I would simply (5s^\kappa-1)^3  to (4s^\kamma)^3=4s^{3\kappa},
% which is true for sufficiently large values of s (above a constant
% threshold).
% This will simplify the formulae (3.5)--(3.10).

% Below Figure 3.18, the paragraph title "Vertical Displacement delta" is
% missing.

% Below Figure 3.20. You introduce omega, but you later use omega_i.
% I would define omega_i from the start.

% Below Figure 3.21, you consider the case that omega_i \leq \pi/2.
% It is unclear why the threshold is \pi/2. Shouldn't it be
% \pi/2-2arctan(1/100N), that is \pi/2 minus twice the angle of
% angle of the right triangle shown in the bottom of Figure 3.21?

% Lemma 7, and the calculation above it: I don't understand this lemma.
% I think the main argument should something line this:
% --If omega_i is close to pi, then O_{i+1} has a horizontal displacement
% of about 2 units, and it
% overlaps with another obstacle (which has small displacement by induction).
% --If omega_i is between 0 and pi (but not close to either 0 or pi), then
%   the top vertex of the small rombus is "too high,", contradicting the
% fact the
%    O_{i+1} has "small" vertical displacement (here the terms "too high" and
%    "small" can be quantified in terms of the parameter s).

% -------------------------------------
% CHAPTER 4. OUTLINE

% In chapter 4, we show that for every epsilon>0 there exists an ordered
% weighted tree T_epsilon such that every realization of T_epsilon as an
% ordered disk contact graph, where the radii of the disks equal the
% vertex weights, has Housdorff distance at most epsilon 1 from a regular
% hexagon of unit side length. Once we establish this, we can prove
% Theorem 4 by extending the construction in Section 3 and simulating
% regular hexagons with ordered trees T_epsilon.

% Section 4.1 should define Hausdorff distance and eps-approximation; and
% provide a few examples.

% Section 4.2 should define the snowflake trees. It is an infinite family
% of ordered trees, say T_k,
% where teach axis contains k vertices.

% Section 4.3 should show that for every k, the snowflake tree (with
% vertex weights 1/2+delta and 1/2
% can be realized as a contact graph of disks; and this realization has
% Hausdorff distance at most 1
% from a regular Hexagon (of side length k). Let us also define a
% "canonical position" of the disk
% centers; where each center is a vertex of the hexagonal lattice. Note,
% however, that the canonical
% position does not give a valid realization!

% Section 4.4 should prove that in  _every_ realization of the snowflake
% tree T_k, every disks is "close"
% to their canonical position (here the term "close" must be
% quantified---an upper bound on the maximal distance of a disk from its
% canonical position comes from the proof).

% Section 4.5: Proof of Theorem 4 (this should be rather short: we simply
% state that the construciton of Section 3 can be repeated).

% -----------------------------------
% Lemma 1 in Chapter 4.

% The statement of the lemma is fine, just need to quantify it in terms of
% gamma.

% In the proof of the lemma, the formula
% y = 2\pi - 5( 2 \tan^{-1} frac{1}{2(1+gamma)}
%    = 2\pi - 10 \tan^{-1} frac{1}{2(1+gamma)}
% is great. Observe that if gamma=0, then y=pi/3.
% By continuity, for small values of gamma,
% y will be close to pi/3.

% Note that 1-1/(1+gamma) < gamma.
% Since the derivative of tan^{-1}(x) is less than 1;
% we conclude that  tan^{-1} frac{1}{2(1+gamma)}
% < tan^{-1}(frac{1}{2})  + frac{1}{2}(1-frac{1}{1+ gamma}
% < pi/3 + gamma/2.
% -----------------------------------

% Caterpillar concept:

% A caterpillar can simulate a small rhombus.

% But the main point is Lemma 2 in that paper, which can be used directly
% for the proof of Lemma 2 in Chapter 4 (to show the displacement is small
% for every axis of the snowflake graph. .

% -------------------------------------