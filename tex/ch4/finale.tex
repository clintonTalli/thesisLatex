\paragraph{Proof of Lemma \ref{lem:ch4IntroLemma}}
\begin{proof}
For any $\epsilon>0$, we construct an ordered weighted tree $T_\epsilon$ and regular hexagon of side length $x$. 
 
Let the corresponding disk arrangement for $T_\epsilon$ be $S_j$ where $$j=\frac{x}{2 \epsilon + 1}.$$  
The Hausdorff distance between the regular hexagon that is the convex hull for the centers of the disks in the disk arrangement and the union of the disks themselves is $$H\lr{r(T_\epsilon),h} = \lr{\frac{2}{\sqrt{3}} - 1} \zeta(\epsilon)$$
The number of disk in the contact graph corresponding is a polynomial $d(x,\epsilon) \leq \frac{4x}{\epsilon}$.  

\begin{minipage}{\linewidth}
\begin{center}
\includegraphics[width=.5\columnwidth]{graphics/hexagonOutlineLayerSmall.pdf}
\captionof{figure}{A regular hexagon of side length $x$ as the convex hull of the centers of a disk arrangement in canonical position.}\label{fig:hexagonOutlineLayerSmall.pdf}
\end{center}
\end{minipage}
\end{proof}

\paragraph{Proof of Theorem \ref{thm:disk}}

\begin{proof}
Given an instance of a P3SAT boolean formula, we can use the snowflake reduction of the modified auxiliary construction.  
For any center of a disk in any realization the displacement of the center is in an open ball $b_{\zeta(\epsilon)}(c)$ where $c$ is the position of the center in canonical position.  
Using Lemma \ref{lem:ch4IntroLemma}, we can approximate any hexagon with a tree $T_\epsilon$.  
In the modified auxiliary construction in Chapter 3, we had four types of hexagons with different side lengths and the skinny rhombus.  
We can scale the weights (radii) of the corresponding ordered weighted disk contact graph corresponding to $T_\epsilon$ to the rigid frame, obstacle, flag, and half sized hexagons in a modified auxiliary contruction accordingly.  
The rhombus can be approximated by a chain of obstacle hexagons.

By approximating the polygons in the modified auxiliary construction with the snowflake, we show that Theorem \ref{thm:disk} is a corollary by applying Lemma \ref{lem:ch4IntroLemma}.  
\end{proof}
