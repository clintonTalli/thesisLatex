\paragraph{Proof of Lemma \ref{lem:ch4IntroLemma}}
\begin{proof}
Given an $\epsilon$, we can construct an ordered weighted tree $T_\epsilon$ and hexagon such the Hausdorff distance of the ordered disk contact graph corresponding to $T_\epsilon$ of equal radii and the hexagon is less than $\epsilon$ in the following manner:
\begin{enumerate}
	\item Let the radii of the of equally weighted disks be $\frac{1}{kd}$ where $d$ is the diameter of the hexagon and $k$ is a multiple of 6, i.e. $k = 6 j$.  
	\item The snowflake $S_j$ is centered at the center of the hexagon.
	\item The Hausdorff distance is at most $\sqrt{3}\epsilon$.
\end{enumerate}	
\end{proof}

\paragraph{Proof of Theorem \ref{thm:disk}}

\begin{proof}
Given an instance of a P3SAT boolean formula, we can use the snowflake reduction of the modified auxiliary construction.  
Using Lemma \ref{lem:ch4IntroLemma}, we can approximate any hexagon with a tree $T_\epsilon$.  In the modified auxiliary construction in Chapter 3, we had four different hexagons and the skinny rhombus.  
We can scale the weights (radii) of the corresponding ordered weighted disk contact graph corresponding to $T_\epsilon$ to the rigid frame, obstacle, flag, and half sized hexagons in a modified auxiliary contruction accordingly.  
The rhombus can be approximated by a chain of obstacle hexagons.

By approximating the polygons in the modified auxiliary construction with the snowflake, we show that Theorem \ref{thm:disk} is a corollary by applying Lemma \ref{lem:ch4IntroLemma}.  
\end{proof}
