\section{Realizability Problems for Weighted Trees}
Recall problems (\ref{problem:UnorderedTree}) and (\ref{problem:OrderedTree}): given a positive weighted tree, $T$, is $T$ the (ordered) contact graph of some disk arrangement where the radii are equal to the vertex weights.  For now, we'll focus on a particular family of this problem space where the weighted trees can realized as a \textit{snowflake}. For $i \in \bbN$, the construction of the snowflake tree, $S_i$, is as follows:
\begin{itemize}
\item Let $v_0$ be a vertex that has six paths attached to it: $p_1$, $p_2$, $\dots$, $p_6$.  Each path has $i$ vertices.
\item For every other path $p_1$, $p_3$, and $p_5$: 
	\begin{itemize}
		\item 	Each vertex on that path has two paths attached, one path on each side of $p_k$.
		\item	The number of vertices that lie on the path attached to the $j^\text{th}$ vertex of $p_k$ is $i-j$.
	\end{itemize}
\end{itemize}
A \textit{perfectly weighted snowflake tree} is a snowflake tree with all vertices having weight $\frac{1}{2}$.  A \textit{perturbed snowflake tree} is a snowflake tree with all vertices having weight of 1 with the exception of $v_0$;  in a perturbed snowflake tree, $v_0$ will have a radius of $\frac{1}{2} + \epsilon$.  For our analysis, all realizations of any snowflake, perfect or perturbed, shall have $v_0$ fixed at origin.  This is said to be the canonical position under Hausdorff distance of the snowflake tree.   

Consider the graph of the triangular lattice with unit distant edges:
\begin{eqnarray*}
V &=& \left\lbrace a\cdot (1,0) + b \cdot \left(\frac{1}{2},\frac{\sqrt{3}}{2}\right) : a,b \in \bbZ \right\rbrace\\
E &=& \left\lbrace \left\lbrace u,v \right\rbrace : \vert\vert u-v \vert\vert = 1 \text{ and } u,v \in V\right\rbrace
\end{eqnarray*}
The following graph, $G=(V,E)$ is said to be the \textit{unit distance graph} of the triangular lattice.  We can show that no two distinct edges of this graph are non-crossing.  First suppose that there were two distinct edges that crossed, $\left\lbrace u_1,v_1 \right\rbrace $ and $\left\lbrace u_2,v_2 \right\rbrace$.  With respect to $u_1$, there are 6 possible edges corresponding to it, with each edge $\frac{\pi}{3}$ radians away from the next.  Neither edge crosses another; and so we have a contradiction that there are no edge crossings with $\left\lbrace u_1,v_1 \right\rbrace $.  


The perfectly weighted snowflake tree that is a subgraph over the \textit{unit distance graph}, $G=(V,E)$, of the triangular lattice.  To show this, for any $S_i$, fix $v_0 = 0 \cdot \cdot (1,0) + 0 \cdot \left(\frac{1}{2},\frac{\sqrt{3}}{2}\right)=\lr{0,0} \in V$ at origin.  Next consider the six paths attached from origin.  Fix each consecutive path $\frac{\pi}{3}$ radians away from the next such that the following points like on the corresponding paths: $\lr{1,0} \in p_1, \lr{\frac{1}{2} ,\frac{\sqrt{2}}{3}} \in p_2,\lr{-\frac{1}{2}\p_4,\frac{\sqrt{3}}{2}} \in p_3, \lr{-1,0} \in p4, \lr{-\frac{1}{2},-\frac{\sqrt{3}}{2}}\in p_5,\lr{\frac{1}{2},-\frac{\sqrt{3}}{2}}\in p_6$.  For $S_i$, there are $i$ vertices on each path.  

We define the six paths from origin as follows:
\begin{eqnarray*}
p_1 &=& \set{a\cdot\lr{1,0} = \vec{v}}{a \in \bbR^+}\\
p_2 &=& \set{a\cdot\lr{\frac{1}{2},\frac{\sqrt{3}}{2}} = \vec{v}}{a \in \bbR^+}\\
p_3 &=& \set{-a\cdot \lr{1,0} + a \cdot \lr{\frac{1}{2},\frac{\sqrt{3}}{2}} = a\lr{-\frac{1}{2},\frac{\sqrt{3}}{2}} = \vec{v}}{a \in \bbR^+}\\
p_4 &=& \set{a \cdot \lr{-1,0} = \vec{v}}{a \in \bbR^+}\\
p_5 &=& \set{a \cdot \lr{-\frac{1}{2},-\frac{\sqrt{3}}{2}}  = \vec{v}}{a \in \bbR^+}\\
p_6 &=& \set{ a\cdot \lr{1,0} - a \cdot \lr{\frac{1}{2},\frac{\sqrt{3}}{2}}= a \cdot \lr{\frac{1}{2}, -\frac{\sqrt{3}}{2}}}{a \in \bbR^+} 
\end{eqnarray*}
For $S_i$ there exists $i$ vertices on each path.  We shall denote the $i^\text{th}$ vertex on the $j^\text{th}$ path as $v_{j,i}$.  For each path defined above, the paths are defined as a set of vectors, $\vec{v} = a \cdot \vec{p}$  for some $a \in \bbR^+$ and $\vec{p} \in \bbR^2$.  By setting $a = 1,2,\dots, i$, we obtain points that are contained in $V$.  For $j = 1,3,5$ and $l = 1,..., i$, there exists two paths attached to each vertex $v_{j,l}$.  For $S_i$, each path attached to the $k^\text{th}$ vertex of $p_j$, there are $i-k$ vertices.  We will need to show that each of the $i-k$ vertices on each corresponding path are also in $V$.

The triangular lattice is symmetice under rotation about $v_0$ by $\frac{\pi}{3}$ radians.  For each vertex $v_{1,l}$ for $l=1,2,..., i-k$, we place two paths from it; the first path $\frac{\pi}{3}$ above $p_1$ at $v_{1,l}$ and $\frac{-\pi}{3}$ below $p_1$ at $v_{1,l}$ and call these paths $p_{1,l}^+$ and $p_{1,l}^-$ respectively.  With respect to $v_{1,l}$, one unit along $p_{1,l}^+$ is a point on the triangular lattice and similarly so on $p_{1,l}^-$.  Continuing the walk along these paths, unit distance-by-unit distance, we obtain the next point corresponding point on the the triangular lattice up to $i-k$ distance away from $v_{1,l}$.  This shows that each of the $i-k$ vertices on $p_{1,l}^-$ and $p_{1,l}^+$ are in $V$.  By rotating all of the paths along $p_1$ by $\frac{2\pi}{3}$ and $\frac{4\pi}{3}$, we obtain the the paths along $p_3$ and $p_5$ respectively, completing the proof.

% The easiest approach would be to embed the path p_1 on the positive
% x-axis, at points (a,0) for a=1,...,i.
% Then note that the triangular lattice is symmetric under rotation about
% the origin by 60-degrees.
% Thus you can embed the paths p_2,...,p_6 as rotated images of p_1.

% For the paths attached to p_1, p_3, p_5, I would use a similar strategy:
% First embed the paths attached to p_1, and then rotate them about the
% origin by 120- and 240-degrees to p_3 and p_5, respectively.
% The two paths attached to vertex (a,0) of p_1, can be simply translated
% copes of p_2 and p_6, both translated by vector (a,0).

% Since no two edges cross in the the unit distance graph G of the
% triangular lattice, any subgraph of G is also crossing-free.
% Consequently, the embedding of the perfectly weighted snowflake graph
% S_i is also crossing-free. 

1) I want to use the lattice as a tessellation of an equilateral triangle in two orientations and show that the remaining vertices are really apart of a class of points (e.g. the class of points whose radian measure with repect to some "origin" is pi/3 or 2*pi/3 and so on) on the regular hexagon about the origin. How would I say this?  This is how I want to show the remaining points are in V.
2) Once I show that these remaining points are in V, I would need to show that the perfect snowflake graph does not intersect itself, correct?



%Consider the lattice points all unit distance away from $v_0$ and every $\frac{\pi}{3}$ radians away from the next consecutive point about origin: $\lr{1,0}, \lr{\frac{1}{2},\frac{\sqrt{2}}{3}},\lr{-\frac{1}{2},\frac{\sqrt{3}}{2}}, \lr{-1,0},\lr{-\frac{1}{2},-\frac{\sqrt{3}}{2}},\lr{\frac{1}{2},-\frac{\sqrt{3}}{2}}$.



%$\lr{1,0}, \lr{\frac{\sqrt{2}}{3},\frac{1}{2}},\lr{0,1},\lr{-1,0}, \lr{-\frac{\sqrt{2}}{3},\frac{1}{2}},\lr{0,-1}, \lr{-\frac{\sqrt{2}}{3},-\frac{1}{2}},\lr{\frac{1}{2}, -\frac{\sqrt{2}}{3}}$

%To show that for any $i$, any perfectly weighted snowflake tree $S_i$ is a subgraph of the triangular lattice unit distance graph, first consider the canonical position where $v_0$ is at origin. 
%Need to fix the realizations

%My goals for next week are as follows:
%1) Clean up chapter 1.  This will happen over the next several weeks.
%2) Continue section 2.3
%3) Add graphics to epsilon-approximation definition to facilitate the explanation.
%4) Anything else  I have time for.



% $\overline{v_0 }$

%Define the snowflake graph, S_i, without pictures
%v_0 has six paths attached to it, p_1, p_2, ..., p_6.  Each path has i vertices.
%for every other path p_1, p_3, and p_5 ,
%	Each vertex on that path has two paths attached, one path on each side of p_k.
%	The number of vertices that lie on the path attached to the j^{th} vertex of p_k is $i-j$
%Let all vertices on the contact graph have weight 1 except $v_0$ whose weight is 1 + \epsilon.
%For any $i$, the weighted contact graph $S_i$, is a subgraph of the unit distance graph of the triangular lattice
%lattice be V = {a(1,0) + b(1/2, \frac{\sqrt{2}}{3} such that a,b \in \bbZ}
%		  E = {{u,v} such that u,v \in V and ||u-v||=1}
%Define this as a canonical position of the disks corresponding to the vertices; goal: show every realization of S_i's disks is close to the canonical position of the disk under hausdorff distance (see lemma)
%NTS: there exists a disk arrangement that corresponds to this weighted contact graph, S_i
%pf:For any two adjacent stems are disjoint, angular displacement and planar (point) displacement
  

%    \begin{prob}[Unordered Realizibility Problem for the Tree]\label{problem:UnorderedTree}
%    For a tree with positive weights for the verticies, it asks whether it is a contact graph of some 
%    disk arrangement where the radii are equal to the vertex weights.
%    \end{prob}
%    
%    \begin{prob}[Ordered Realizibility Problem for the Tree]\label{problem:OrderedTree}
%    For a tree with positive weights for the vertices, it asks whether its corresponding graph is the 
%    ordered contact graph of some disk arrangement where the radii equal the vertex weights.
%    \end{prob}