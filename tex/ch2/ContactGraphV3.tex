\section{Contact Graphs and Disk Arrangements}
% Outline:
% 1) What is the goal of this section?
% 	a) What is needed to achieve this goal?
% 	b) Formally pose the problem.
% 	c) How is (a) used to solve (b)?
Using logic engines, it has been shown that it is NP-Hard to decide whether the graph $G$ is a contact graph of unit disks \cite{BET+99}. 
\begin{thm}\label{thm:ContactGraphV3-1}
It is NP-complete to decide whether a polygonal linkage whose hinge graph is a tree can be realized.
\end{thm} 
\begin{proof}
proof goes here
\end{proof}
We take the proof of Theorem \ref{thm:ContactGraphV3-1} and adapt it to contact graphs, i.e.:
\begin{thm}\label{thm:ContactGraphV3-2}
It is NP-complete to decide whether a contact graph can be realized.
\end{thm}
In order to show this, we will review some properties about contact graphs.
\subsection{Contact Graph Properties}
\paragraph{Hausdorff Distance}  Let $A$ and $B$ be sets in the plane. The \textit{directed Hausdorff distance} is 
\begin{equation}\label{eqn:ContactGraphV3-1}
d\lr{A,B} = \sup_{a \in A} \inf_{b \in B} \left\vert\left\vert a-b \right\vert \right\vert
\end{equation}
$h\lr{A,B}$ finds the furthest point $a in A$ from any point in $B$.  \textit{Hausdorff distance} is
\begin{equation}\label{eqn:ContactGraphV3-2}
D\lr{A,B} = \max \left\lbrace d\lr{A,B}, d\lr{B,A} \right\rbrace
\end{equation}
\begin{figure}[!htbp]
\begin{center}
\includegraphics{graphics/HausdorffDistanceExample1.pdf}
\caption{An illustrative example of $d(X,Y)$ and $d(Y,X)$ where $X$ is the inner curve, and $Y$ is the outer curve.}\label{fig:HausdorffDistanceExample1.pdf}
\end{center}
\end{figure}
\paragraph{$\epsilon$-approximation}
% Modeling the logic engine with polygonal linkages requires reflected copies of the
% rectangles. For an oriented realization, we use a different technique in Section 3. The
% above proof can be adapted to the realization of contact trees of disks by approximating
% rectangles with disk arrangements. In this context, we say that a weighted graph G is a
% ε-approximation of a polygon P if G is realizable as a contact graph of disks of given
% radii, and in every such realization, the Hausdorff distance between the union of disks
% and a congruent copy of P is at most ε. A weighted graph G is a stable ε-approximation
% if, in addition, for every two such realizations of G, the distance between the centers of
% the corresponding disks is at most ε after a suitable rigid transformation.
The weighted graph, $G$, is an \textit{$\epsilon$-approximation} of a polygon $P$ if the Hausdorff distance between every realization such realization of $G$ as a contact graph of disks and a congruent copy of $P$ is at most epsilon.  A weighted graph $G$ is said to be a \textit{$\BigOh{f(x)}$-approximation} of a polygon P if there is a positive constant $M$ such that for all sufficiently large values of $x$ the Hausdorff distance between every realization such realization of $G$ as a contact graph of disks and a congruent copy of $P$ is at $M \cdot \vert f(x)\vert$. A weighted graph $G$ is said to be a \textit{stable} if it has the property that for every two such realizations of $G$, the distance between the centers of the corresponding disks is at most $\epsilon$ after a suitable rigid transformation.  





\begin{lem}\label{lem:ContactGraphV3-1}
For any natural number $k$,  there exists a tree $T = (V,E)$ with vertex weights $1$ and $1 - k^{-3}$ such that a stable $\BigOh{k^{-1}}$-approximation of a rectangle with ratio $\sqrt{3}$.
\end{lem}
\begin{thm}\label{thm:ContactGraphV3-3}
It is NP Complete to decide whether a given tree with positive vertex weights is the contact graph of a disk arrangements with specified radii.
%1) needs to show that
\end{thm}

\begin{figure}[!htbp]
\begin{center}
\includegraphics{graphics/InsertFigure.pdf}
\caption{Insert Caption}\label{fig:insertSomething}
\end{center}
\end{figure}

\begin{pf}
\begin{itemize}
\item[(i)]  To prove \ref{thm:ContactGraphV3-3}, we need to show the following:
	\begin{itemize}
	\item[(a)] blah blah
	\item[(b)] blah blah
	\end{itemize}
\item[(ii)] Once we have (i), we then proceed with...
\end{itemize}
\end{pf}
