

\section{Contact Graphs and Disk Arrangements}

\begin{figure}[!htbp]
\centering
\begin{minipage}{0.45\textwidth}
\centering
\includegraphics{graphics/snowflakeOutline5LayerSmallNoCircles.pdf}
\caption{A snowflake.}\label{fig:snowflakeOutline5LayerSmallNoCircles}
\end{minipage}\hfill
\begin{minipage}{0.45\textwidth}
\centering
\includegraphics{graphics/snowflakeOutline5LayerSmall.pdf}
\caption{A configuration of disks about a snowflake.}\label{fig:snowflakeOutline5Layer}
\end{minipage}
\end{figure}
In Figure \ref{fig:snowflakeOutline5LayerSmallNoCircles}, we have a realization of a snowflake graph whose branches are placed $\frac{2 \pi}{3}$, $\frac{4 \pi}{3}$, and $2 \pi$ radians about the origin. Each branch contains five stems on its right side and on its left side.  Each consecutive stem is equidistant to its neighbor.  The $i^\text{th}$ stem of the right side is the same length as the $i^\text{th}$ stem of the left side.  We will denote these snowflake realizations as $S_i$ where $i$ is the number of stems on any side of any branch.  Figure \ref{fig:snowflakeOutline5LayerSmallNoCircles} is $S_5$.  In Figure \ref{fig:snowflakeOutline5Layer}, we have $S_5$ and corresponding set of outlined disks, $D_5$.  Each of the disks are of unit radius.  The disks are arranged in a disk packing that form concentric hexagonal rings with respect to the disk centered at the origin of the snowflake.  We will denote the set of disks as $D_i$, where $i$ is the number of concentric, hexagonal rings of disks.  The number disks in $D_i$ is $1+3i+3i^2$.  The number of stems in $S_i$ is $6i$.  (needs to discuss the disk adjacency and its relation to $S_i$; the number of edges in $D_i$ is $-6 +12i$).

\begin{prob}
Let $\epsilon > 0$.  Given $D_i$ and $S_i$ such that the disk centered at the origin has radius $r_0 = 1+\epsilon$, does $D_i$ approximate a hexagonal packing?
\end{prob}
\begin{figure}[!htbp]
\centering
\begin{minipage}{0.45\textwidth}
\centering
\includegraphics{graphics/D1.pdf}
\caption{$D_1$ illustrated.  All disks have unit radius.}\label{fig:D1}
\end{minipage}\hfill
\begin{minipage}{0.45\textwidth}
\centering
\includegraphics{graphics/epsilonCenter.pdf}
\caption{A modified $D_1$ such that the center disk has radius $1+\epsilon$ whereas the other disks have unit radius.  Let this modified $D_1$ be called $D_{1,\epsilon}$.}\label{fig:epsilonCenter}
\end{minipage}
\end{figure}
In order to answer this question be must define some terminology.  \textit{Stability} is the condition that every realization of $D_{i,\epsilon}$ is close to the hexagonal graph with weights on vertices.  Let $G$ be an ordered tree. With any two realizations $R_1$, $R_2$.  $G$ is \textit{$\epsilon$-stable} with any two realizations the vertices are close.

\begin{minipage}{\linewidth}
\begin{center}
\includegraphics[width=.9\columnwidth]{graphics/epsilonCenterWithHexagon.pdf}
\captionof{figure}{$D_{1,\epsilon}$ with a hexagon whose vertices are the centers of the disks about the center disk.}\label{fig:epsilonCenterWithHexagon}
\end{center}
\end{minipage}

% \begin{figure}[!htbp]
% \begin{center}
% \includegraphics{graphics/epsilonCenterWithHexagon.pdf}
% \caption{$D_{1,\epsilon}$ with a hexagon whose vertices are the centers of the disks about the center disk.}\label{fig:epsilonCenterWithHexagon}
% \end{center}
% \end{figure}

\begin{thm}\label{thm:ContactGraphV3-1}
It is NP-complete to decide whether a polygonal linkage whose hinge graph is a tree can be realized.
\end{thm}
 
\begin{lem}\label{lem:ContactGraphV3-1}
For any natural number $k$,  there exists a tree $T = (V,E)$ with vertex weights $1$ and $1 - k^{-3}$ such that a stable $\BigOh{k^{-1}}$-approximation of a rectangle with ratio $\sqrt{3}$.
\end{lem}

\begin{thm}
 Let $G$ be an $\epsilon$-stable ordered tree. For any two realizations $R_1$, $R_2$, there exists a rigid transformation $\gamma$ such that the $$\text{Vertices}\left(\gamma\left(R_1 \right)\right) - \text{Vertices}\left(\gamma\left(R_2\right)\right) < \epsilon$$
\end{thm}

%2) state theorem: Hexagon of side length 1, for any \epsilon > 0, there exists an oriented vertex weighted tree that satisfies the following:
%	a) |V| is defined polynomially as a function of epsilon
%	b) all weights are bounded by some polynomial as a function of epsilon
%3) prove step by step:
%	a) define the tree
%	b) show that any realization of the tree is close to T_i, i.e. for any realization of a tree there exists a T_i such that H(T_i,P)<\epsilon
%4) proof by induction
%	a) one for the angles
%	b) one for the centers
%	c) fuzzy hexagons approximate regular hexagons are approximal from the same proof

%Define the snowflake graph, S_i, without pictures
%v_0 has six paths attached to it, p_1, p_2, ..., p_6.  Each path has i vertices.
%for every other path p_1, p_3, and p_5 ,
%	Each vertex on that path has two paths attached, one path on each side of p_k.
%	The number of vertices that lie on the path attached to the j^{th} vertex of p_k is $i-j$
%Let all vertices on the contact graph have weight 1 except $v_0$ whose weight is 1 + \epsilon.
%For any $i$, the weighted contact graph $S_i$, is a subgraph of the unit distance graph of the triangular lattice
%lattice be V = {a(1,0) + b(1/2, \frac{\sqrt{2}}{3} such that a,b \in \bbZ}
%		  E = {{u,v} such that u,v \in V and ||u-v||=1}
%Define this as a canonical position of the disks corresponding to the vertices; goal: show every realization of S_i's disks is close to the canonical position of the disk under hausdorff distance (see lemma)
%NTS: there exists a disk arrangement that corresponds to this weighted contact graph, S_i
%pf:For any two adjacent stems are disjoint, angular displacement and planar (point) displacement



