% CHAPTER 2 OUTLINE
% 2.1 lOGIC eNGINE
%       Intro, give reference to logic engine
%       Explanation of how it works
%       Theorem 2.1.1.
% 2.2 Theorem and Proof to hinged polygons/Squares
% 2.3 Unit Disk Contact Graph


% \section{Logic Engines Represented as Polygonal Linkages}

\section{Logic Engines Represented as Polygonal Linkages}         
\begin{figure}[!htbp]
\begin{center}
\includegraphics{graphics/HingedLogicEngineSmall.pdf}
\caption{A logic engine realized as a polygonal linkage.}\label{fig:HingedLogicEngineSmall.pdf}
\end{center}
\end{figure}
Suppose we are given an instance of a NAE3SAT with $m$ clauses and $n$ variables in 3-CNF form.  
The polygonal linkage logic engine that corresponds to this boolean formula has the following 
dimensions:
$$\begin{array}{|l|c|c|c|}%TRY TO ADD SHADING COLUMN TO TABLE AND GRAPHIC:
 \hline
 \text{Component} & \text{Height} & \text{Width} & \text{Quantity}\\\hline
 \text{Shaft Subcomponent} & 1 & 3 & n\\\hline
 \text{Armature Subcomponent} & 2 & 1 & 2\cdot m\\\hline
 \text{Flag} & 1 & 1.5 & -\\\hline
 \text{Large Frame Subcomponent} & 2\cdot m & 1.75 & 4\\\hline
 \text{Small Shaft Subcomponent} & 1 & 1 & 2\\\hline
\end{array}$$
The shaft subcomponents correspond to an armature and its subcomponents. The shaft subcomponents 
have two orientations.  The armature subcomponents may have rectangular flags. Flagging arragement 
indicates the relationship of the boolean literal's existence within a clause.   An armature 
subcomponent is flagged according to the following rules:
\begin{enumerate}
 \item If the literal $x_j$ is found in clause $C_k$, then the armature subcomponent $l_{j,k}$ is 
unflagged.
 \item If the literal $\bar{x}_j$ is found in clause $C_k$, then the armature 
subcomponent $\bar{l}_{j,k}$ is unflagged.
\end{enumerate}
Both frame subcomponents are static.  
\begin{thm}\label{thm:Satisfiability-1}
 Given an instance of a $NAE3SAT$,  it is a ``yes'' instance if and only if the 
corresponding polygonal linkage logic engine has a collision-free configuration.  
\end{thm}

