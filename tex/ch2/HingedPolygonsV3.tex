													]\section{Logic Engines Represented as Polygonal Linkages}   
In the previous section, we introduced the logic engine.  
This section builds an analogous structure that is formed from a polygonal linkage and we may interchangeably say subcomponent for polygon.
We can modify the mechanical structure of the logic engine to form a polygonal linkage.  
For a given a boolean formula, $\Phi$, in 3-CNF with $n$ variables and $m$ clauses,
the rigid frame is broken into two polygons, each polygon on the extremity of the structure.
The shaft is broken into $n$ polygons.
Each armature is broken into two parts, each part containing $m$ subcomponents.
In figure \ref{fig:HingedLogicEngineSmall.pdf}, each flag becomes a rectangle.

\begin{figure}[!htbp]
\begin{center}
\includegraphics{graphics/HingedLogicEngineSmall.pdf}
\caption{A logic engine realized as a polygonal linkage.}\label{fig:HingedLogicEngineSmall.pdf}
\end{center}
\end{figure}
\subsection{Construction of the Polygonal Linkage Logic Engine}
Suppose we are given an boolean formula with $m$ clauses and $n$ variables in 3-CNF form, $\Phi$, we construct the polygonal linkage similarly to the logic engine.
The corresponding polygonal linkage $P_\ell = (\PP,\HH)$ has the following polygon components:
\begin{center}
	\begin{table}\label{TBL:HingedPolygonsV3-1}
		$$\begin{array}{|l|c|c|c|}%TRY TO ADD SHADING COLUMN TO TABLE AND GRAPHIC:
		 \hline
		 \text{Component} & \text{Height} & \text{Width} & \text{Quantity}\\\hline
		 \text{Large Frame Subcomponent} & 2\cdot m & 1 & 2\\\hline
		 \text{Shaft Subcomponent} & 1 & 3 & n\\\hline
		 \text{Armature Subcomponent} & 2 & 1 & 2\cdot m\\\hline
		 \text{Flag} & 1 & 1.5 & 2mn-3m\\\hline
		\end{array}$$
		\caption{The components of $\PP$ specified polynomially in terms of the size of the boolean formula $\Phi$.}
	\end{table}
\end{center}

The large frame subcomponents are hinged on the left most and right most shaft subcomponents. 
Each adjacent shaft subcomponents are hinged and each shaft subcomponent has two orientations, a reflecion up and a reflection down about the shaft hinge points.  
On each shaft subcomponent there are two armature shaft subcomponents, one above the shaft subcomponent and one below the shaft subcomponent.  
By the two orientations of the shaft subcomponent, each armature subcomponent has to possible positions.  
Each armature comprises of $m$ armature subcomponents that are hinged together; in total there are $2n$ armatures.  
Each armature subcomponent has two orientations, a reflection left and a reflection right about the armature hinge points.  
Label the armature subcomponents on the $j^\text{th}$ armature starting from the shaft by $\ell_{j,1},\ldots,\ell_{j,n}$ on one side and  $\bar{\ell}_{j,1},\ldots,\bar{\ell}_{j,n}$ on the other side of the shaft.  
Attach a rectangular flag specified in Table \ref{TBL:HingedPolygonsV3-1}, to some of these segments. 
Each segment is either flagged one or zero flags.

\begin{enumerate}
	 \item If the literal $x_j$ is found in clause $C_k$, then $\ell_{j,k}$ is unflagged.
	 \item If the literal $\bar{x}_j$ is found in clause $C_k$, then $\bar{l}_{j,k}$ is unflagged.
\end{enumerate}

Each flag has two orientations with respect to armature it is attached to.  Each flag has four potential positions, the flag can reflect left or right about the armature and the armature can reflect up or down about the shaft.

\begin{figure}[!htbp]
\begin{center}
\includegraphics{graphics/HingedLogicEngineSmallEnumerated.pdf}
\caption{A polygonal linkage logic engine that corresponds to the boolean formula $\Phi = C_1 \cap C_2 \cap C_3$.}\label{fig:HingedLogicEngineSmallEnumerated.pdf}
\end{center}
\end{figure}

\begin{thm}\label{thm:chp2-HingedPolygons-1}
 Given an instance of a $NAE3SAT$,  it is a ``yes'' instance if and only if the 
corresponding polygonal linkage logic engine has a collision-free configuration.  
\end{thm}
\begin{proof}
Suppose we have an instance of a $NAE3SAT$ that is a ``yes'' instance. This implies that there is a 
truth assignment such that each clause contains a true and a false literal. Now consider the polygonal linkage logic 
engine corresponding to this instance. We now 
show that it has a collision free configuration.

For variables that are true, configure the armatures such that the flags corresponding to the 
non-negated literals reside above the 
shaft and the flags that correspond to the negated literals reside below this shaft.  For variables 
that are false, configure the 
armatures in the opposite orientation.  Each clause corresponds to a pair of rows in 
the polygonal linkage logic engine, one row for non-negated literals and one for negated literals.  Because the 
$NAE3SAT$ is a yes instance, every row contains at least one unflagged armature.  
By Lemma \ref{lem:logicEngine1}, every row  has a collision-free configuration.

Suppose we have an instance of a $NAE3SAT$ such that the corresponding polygonal linkage logic engine has a 
collision-free configuration. By Lemma \ref{lem:logicEngine1} every row at least one unflagged 
armature.  The $k^{th}$ clause is represented by the $k^{th}$ rows above and below the shaft. If the 
literal $x_j$ is found in clause $C_k$, then the armature is unflagged in that row. If the literal 
$\bar{x}_j$ is found in clause $C_k$, then $\bar{l}_{j,k}$ is unflagged.  All flags 
corresponding to negated literals reside below the shaft and flags corresponding to non-negated 
literals reside above the shaft.  All together we have that every clause has a true literal and a 
false literal.  Thus, we have a 'yes' instance of the $NAE3SAT$.
\end{proof}