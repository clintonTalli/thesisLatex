\paragraph{Vertical Displacement $\delta$}

When an obstacle hexagon is rotated by $\alpha_i$, the height of the obstacle hexagon becomes $h \sec \alpha_i$ where $h$ is the canonical height of the obstacle hexagon (see Figure \ref{fig:hexagonNonCanonical.pdf} for reference).
Figure \ref{fig:hexagonNonCanonical.pdf} shows the geometry of a rotated obstacle hexagon.

\begin{minipage}{\linewidth}
\begin{center}
\includegraphics[width=.33\columnwidth]{graphics/hexagonNonCanonical2.pdf}
\captionof{figure}{
This figure shows a right triangle with angle $\alpha_i$ and sides of length $h$ and $\frac{h}{\cos \alpha_i}$.
}\label{fig:hexagonNonCanonical.pdf}
\end{center}
\end{minipage}

To show that the vertical displacement from canonical position is small, we first consider a column of obstacle hexagons in canonical position (see Figure \ref{fig:dualSmallHexagonalGrid.pdf} for illustration).  
For canonical position, the $\jth$ obstacle has $\delta_j = 0$.

\begin{minipage}{\linewidth}
\begin{center}
\includegraphics[width=.33\columnwidth]{graphics/verticalDisplacementArgument.pdf}
\captionof{figure}{This illustration is of a column of obstacle hexagons in canonical position along a vertical line segment $\ell$.}\label{fig:verticalDisplacementArgument.pdf}
\end{center}
\end{minipage}

From Equation \ref{eqn:Hnm}, we know the exact height of $\ell$ in terms of the heights of the corridors and obstacle hexagons in canonical position.  
Consider the first $j$ terms for the height of the column of obstacle hexagons and corridors for an arbitrary construction with angular rotation and vertical displacement for \textit{one} obstacle hexagon $\vert \delta_v \vert > 0$, where $j=2$, $\cdots$, $u+1$ and $1 < v \leq j$.
\begin{eqnarray*}
\sum_{i=1}^j \lr{2 \sqrt{3} N \sec \lr{ \alpha_i}} + \delta_v  + (j-1) \lr{\frac{1}{100N}+\sqrt{3}} &\leq& j \cdot 2 N \sqrt{3} + (j-1) \cdot \lr{\frac{1}{100N}+\sqrt{3}}\\
2 \sqrt{3} N \sum_{i=1}^j \sec \lr{\alpha_i} + \delta_ v &\leq& j \cdot 2 \sqrt{3} N \\
\sum_{i=1}^j \sec \lr{\alpha_i} + \delta_v &\leq& j\\
 \delta_v &\leq& j- \sum_{i=1}^j \sec \lr{\alpha_i}\\
 \delta_v &\leq& j - \lr{j - \sum_{i=1}^j \frac{\alpha_i^2}{2}}\\
\delta_v &\leq&  \sum_{i=1}^j \frac{\alpha_i^2}{2}
\end{eqnarray*}

Using Inequalities \ref{eqn:angularSumBound} and \ref{eqn:angularMaxBound}, we derive the following result:
$$
\begin{array}{rcl}
\sum_{i=1}^j \frac{\alpha_i^2}{2} &\leq& \frac{1}{2}\sum_{i=1}^j  \lr{\frac{288s^2+24s}{125 s^{3\kappa}-75 s^{2\kappa}+15 s^\kappa-5}}^2\\
&\leq&\frac{1}{2}\cdot  \lr{\frac{288s^2+24s}{125 s^{3\kappa}-75 s^{2\kappa}+15 s^\kappa-5}}^2 \cdot j\\
&\leq&\frac{1}{2}\cdot  \lr{\frac{288s^2+24s}{125 s^{3\kappa}-75 s^{2\kappa}+15 s^\kappa-5}}^2 \cdot u\\
&\leq&\frac{1}{2}\cdot  \lr{\frac{288s^2+24s}{125 s^{3\kappa}-75 s^{2\kappa}+15 s^\kappa-5}}^2 \cdot 12s\\
&\leq&  \frac{6s\lr{288s^2+24s}^2}{\lr{125 s^{3\kappa}-75 s^{2\kappa}+15 s^\kappa-5}^2}
\end{array}
$$

Thus we finally say that the bound for $\delta_v$, where $1<v\leq j\leq u$, is small:
\begin{equation}\label{eqn:verticalBound}
\delta_v \leq \frac{6s\lr{288s^2+24s}^2}{\lr{125 s^{3\kappa}-75 s^{2\kappa}+15 s^\kappa-5}^2}
\end{equation}



